\chapter{Discontinuous Galerkin for Maxwells Equations} % Write in your own chapter title
\label{Chapter3}
\lhead{Chapter 3. \emph{Discontinuous Galerkin}} % Write in your own chapter title to set the page header

\subsection{Formulation}

The Discontinuous Galerkin method was first introduced to solve the neutron transport problem by Reed and Hill \cite{} in 1973.

** Some background/literature review here Cockburn, Shui for solving hyperbolic equations etc ***
% PhDJesusAlvarez good for motivation - not so much for the method

As seen in Chapter \ref{PhysicalProblemChapter}, given a suitable choice of initial conditions the evolution of the system in time can be described by Maxwell's curl equations \eqref{maxwell-curl-equations-conservation-form}, given again for convenience

$$
\ut + \fk  = \mathbf{S(U)}
\label{strong-form-DG}
$$

Consider that the problem is defined on a physical domain, $\Omega$, which can be discretised by an unstructured mesh of $K$ element such that

\begin{equation}
  \Omega \approx \Omega_h = \mathop{\bigcup}_{k=1}^{K} \Omega_e^k
\end{equation}
where $\Omega_e^k$ are the elements in the discretisation.  

% discuss the discretisation more - discontinuous elements etc

Following the method of weighted residuals the \eqref{strong-form-DG} is multiplied by a vector of test functions $\mathbf{W}$ and integrated over an element $\Omega_e^k$. The following weak form is obtained by integration by parts

$$
\int_{\Omega_e^k} \mathbf{W} \ut d\Omega_e^k  - \int_{\Omega_e^k} \frac{\partial \mathbf{W}}{ \partial x_k} \mathbf{F}_k(\mathbf{U}) d\Omega + \int_{\partial \Omega_e^k} \mathbf{W} \cdot \mathbf{F_n}(\mathbf{U_e}) d\Gamma = \int_{\Omega_e^k} \mathbf{W} \cdot \mathbf{S}(\mathbf{U_e}) d\Omega
\label{maxwell-DG-weak-form}
$$

where $n_k$ is the $k$th component of the element outward normal to $\partial \Omega_e^k$ and $\mathbf{F_n}$, the normal flux, is given by

$$
\mathbf{F_n}(\mathbf{U}) = \mathbf{F}_k(\mathbf{U}) n_k
$$

This weak form is specified on an element, however this is not yet a scheme suitable for solving the global problem. In order to recover the global solution the continuity of the solution between elements is weakly enforced by replacing the physical normal flux $\mathbf{F_n}(\mathbf{U_e})$ with a consistent numerical flux $\mathbf{\tilde{F}}_n(\mathbf{U}_e,\mathbf{U}_e^{out})$, evaluated in terms of the solution on an element, $\mathbf{U}_e$, and the is the value of the solution along a given face in adjoining element sharing that face, $\mathbf{U}^{out}$.

*** conditions to be satisfied by numerical flux + form of numerical flux ***

The system is discretised by choosing the solution approximated by
$$
U_e \simeq \sum_{i=1}^{n} u_{i} N_{i}
$$
where $N_{i}$ are Lagrangian shape functions and $u_{i}$ are nodal solution values. Following the Galerkin method test functions $W$ are then chosen with the same basis of shape functions:

$$
W = \sum_{i=1}^{n} N_{i}
$$

The resulting discretised system of equations can be written as a system of ordinary differentail equations

$$
\mathbf{M} \frac{d \mathbf{U}} {dt} + \mathbf{R}(\mathbf{U}) = 0
$$

where $\mathbf{U}$ is a vector of the coefficients $u_{i}$, $M$ is the mass matrix which is block diagonal and $\mathbf{R}$ is the residual vector.

*** RK4 Stability condition

\subsubsection{Local Element Equations}
\begin{itemize}
  \item problem in conservation form -> GMWR -> weak form
  \item discretised versions of relevant maxwells equations
	\item local matrix + broken space
	\item Nodal (or Modal) representation
\end{itemize}
\subsubsection{Numerical Flux}

\begin{itemize}
	\item recover solution with numerical flux
  \item numerical flux used
	\item upwind flux: effect for wave domination problems - flow of information, upwind flux
\end{itemize}

\subsection{Time Integration}
\begin{itemize}
	\item explicit RK4
	\item breif mention explicit justification in terms of delta t limit (and refer to chapter)
	\item justify high-order time integration
\end{itemize}

The solution is advanced in time with an explicit fourth order Runge-Kutta (RK4) method. Implicit schemes which allow larger time steps may be employed to solve the system of equations in a shorter computational time. However as detailed in section [ ***section ref *** ] the transient system behaviour at closely spaced intervals is required in order to reduce the accuracy in time. Thus an upper limit on timestep length is imposed by the frequencies of interest and the computation accuracy required.

\subsection{Implementing Boundary Conditions}
\begin{itemize}
	\item material interfaces (PEC,ABC)
  \item will need to mention PML -> another section
\end{itemize}

For interfaces which intersect the domain boundary, $\partial \Omega$, the not all components of $\mathbf{U}^{out}$ are determined by the boundary conditions on the interface. For a system of conservation laws Rankine-Hugoniot jump conditions of the form

$$
\left[[ \mathbf{F}_n \right]] = \lambda_j \left[[ \mathbf{U} \right]]
$$

where $\lambda_j$ are the eigenvalues of the jacobian matrix $\mathbf{A}_n$. For the 3-dimensional case

$$
\lambda_1 = \lambda_2 = \frac{1}{\sqrt{\epsilon_L \mu_L}}
$$

$$
\lambda_3 = \lambda_4 = 0
$$

and

$$
\lambda_5 = \lambda_6 = \frac{1}{\epsilon_R \mu_R}
$$
where the subscripts  $L$ and $R$ on the material parameters $\epsilon$ and $\mu$ denote the left and right side of the interface respectively.

are applied at the interfaces

\subsubsection{Spatial Discretisation}
\begin{itemize}
	\item discuss types of element - (non-)affine, planar, different shapes etc.
	\item condition number of a matrix - moving nodes around %reference!
\end{itemize}

\subsection{Errors and convergence}
\begin{itemize}
  \item expected rates of convergence for time-domain (interpolation error) and freq domain (dispersion error)
\end{itemize}
