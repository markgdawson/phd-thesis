% Chapter 1

\chapter{Analytical Investigations} % Write in your own chapter title
\label{Chapter 4}
\lhead{Chapter 4. \emph{Analytical Investigations}} % Write in your own chapter title to set the page header

\section{Mesh Refinement Study for Analytical Solutions}
A study of FFT and FDM methods using an analytical wave shape were performed to try and determine how the method is affected by signal length and mesh interpolation. The waves studied are of the form:
$$
E(x,t) = \sum_i sin(kx - \omega_i t)
$$
Take two spatial points $x_1$ and $x_2$, which are a distance $\Delta x$ appart. Choose a point $x_0$ which is half way between these two points, and represent the value at that point as a linear interpolation of the values at $x_1$ and $x_2$.
$$
E(x_0,t) = \frac{1}{2} \left[ E(x_1,t) + E(x_2,t) \right]
$$

The studies in this section were performed with a single and multiple frequencies ($\omega_i$) however no major differences were found and the results presented here are for f=100 ($\omega = 2 \pi f$);

The idea is then to cary $\Delta_x$ and investigate the effect of the mesh on the signal achieved. This is illustrated in figure \ref{MonitorPointMeshRefinement_waves_in_space} which shows the spatial, analytical waves in space at time $t=0$ and a line showing the linear approximation being used to find the value at $x_0$ for 9 different values of $\Delta x$. These 9 different values of $\Delta x$ will be used throughout the study. Clearly depending on the size of $\Delta x$ there will be a phase difference in the signals between the sampled points at $x_0$, $x_1$ and $x_2$. A very small $\Delta x$ will give a signal shape very close to signal both at $x_1$ and $x_2$. The larger values of $\Delta x$ however give a linear interpolation with a larger phase difference and the phase difference between the wave at point $x_0$ and $x_1$ in space determines the amplitude. Figure \ref{MonitorPointMeshRefinement_waves_in_time} shows analytical values for signals at $x_0$, $x_1$ and $x_2$ while Figure \ref{MonitorPointMeshRefinement_waves_approx_in_time} shows the analytical values at $x_0$ (the same as those in Figure \ref{MonitorPointMeshRefinement_waves_in_time}) compared to the linear interpolations obtained at $x_0$. Figure \ref{MonitorPointMeshRefinement_waves_approx_in_time} shows that the linear interpolations for the selected $\Delta x$ values vary visibly from the analytical values of the waves sampled at the same point. Values of $\Delta x$ in which the points selected are in phase show an interpolated wave which is in phase. Whilst points out of phase show waves with a very different phase and amplitude.

\begin{figure}
    \subfigure[Waves for t=0 in space showing the linear approximation being used for each $\Delta x$.]{
        \includegraphics[width=0.48\textwidth]{Figures/1D_Analytical_Study/MonitorPointMeshRefinement/waves_in_space.png}
        \label{MonitorPointMeshRefinement_waves_in_space}
    }
    \subfigure[This plot shows the exact (analytical) values of the waves sampled in time at points $x_0$ (green), $x_1$ (blue) and $x_2$ (red).] {
        \label{MonitorPointMeshRefinement_waves_in_time}
        \includegraphics[width=0.48\textwidth]{Figures/1D_Analytical_Study/MonitorPointMeshRefinement/waves_in_time.png}
    }
    
    \subfigure[Values obtained at $x_0$ by linear interpolation (blue) and the corrisponding analytical values at $x_0$ (green) for different $\Delta x$.]{
        \label{MonitorPointMeshRefinement_waves_approx_in_time}
        \includegraphics[width=0.48\textwidth]{Figures/1D_Analytical_Study/MonitorPointMeshRefinement/waves_approx_in_time.png}
    }
    \caption{Analytical and interpolated waves in space and time}
    
    \label{MonitorPointMeshRefinement}
\end{figure}

\begin{figure}
    \subfigure[FFT spectrum from waves sampled for 1000 cycles. Waves with different mesh sizes appear to be have different spectral amplitudes.]{
        \includegraphics[width=0.66\textwidth]{Figures/1D_Analytical_Study/MonitorPointMeshRefinement/waves_in_time_fft.png}
    }
    
    \subfigure[Peak frequency error in the two frequency components from the analytical frequency]{
        \includegraphics[width=0.66\textwidth]{Figures/1D_Analytical_Study/MonitorPointMeshRefinement/PeakFreqError.png}
    }
    \caption{Frequencies Calculated from FFT of signal in time obtained from interpolation at various $\Delta x$s}
    \label{MonitorPointMeshRefinement_FFT}
\end{figure}

\section{Timestep Refinement Study for Analytical Solutions}

A the same analytical wave shapes are used
$$
E(x,t) = \sum_f sin(kx - \omega_f t)
$$
where
$$
\omega_f = [ 100, 120, 1000, 500, 300 ]
$$

\begin{figure}
\includegraphics[width=\textwidth]{Figures/1D_Analytical_Study/MonitorPointDtRefinement/waves_in_time.png}
\caption{Original signal in time with red line showing points selected for approximation}
\end{figure}

\begin{figure}
\includegraphics[width=\textwidth]{Figures/1D_Analytical_Study/MonitorPointDtRefinement/waves_in_time_fft.png}
\caption{Spectrum FFT - cut off point visible}
\end{figure}

\begin{figure}
\includegraphics[width=\textwidth]{Figures/1D_Analytical_Study/MonitorPointDtRefinement/PeakFreqError.png}
\caption{Frequencies plotted up to cutoff point}
\end{figure}

\begin{figure}
\includegraphics[width=\textwidth]{Figures/1D_Analytical_Study/MonitorPointDtRefinement_byDoubling/PeakFreqError.png}
\caption{Frequencies plotted against cutoff (more points)}
\end{figure}