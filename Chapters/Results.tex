
%-- Simulation  --
%
%PCB VCSEL
%
%-- Dispersive Photonic Crystals --
%* Photonic Crystal Vertical Cavity Surface Emitting Lasers -> 3 weeks
%  * Photonic Band-Gap Defect cavity - single cell
%  * Photonic Band-Gap hexagonal, square and stick waveguide resonators
%* Add-drop photonic band gap filters  -> 3 weeks
%  - square, circular, hex shapes
%  - double
%  - with/without "scatterers"
%* maybe a comparison to FDTD again -> 2 weeks
%
%-- 2D Examples --
%* Rectangular cavity with convergence curves (write up) -> done
%* PEC Circle cavity - comparison of NEFEM and high-order -> 2 week
%* Equilateral triangle cavity (or polygon) -> 1 week
%* Infinite 2D dielectric circle cavity (in a PML) with add-drop filter - Niegemann -> 2 weeks
%* Cylindrically symmetric cavity example (Bangor example) -> 2 weeks
% 
%-- Parallel Examples --
%* Aeroplane Scattering (comparison to low-order code) - might require some optimisations (faces etc) -> 2 weeks
%* Comparison of code speed for large 3D meshes -> 1 week
%* Study of partitioning for element orders and types (2D + 3D) -> 2 weeks


\label{Chapter3}
\chapter{Free space rectangular PEC cavity} % Write in your own chapter title

\begin{itemize}
  \item Problem setup
  \item Convergence of error with simulation period
  \item dt study
  \item Scheme convergence results
\end{itemize}

In this chapter we validate our results using a 2-dimensional rectangular free space ($\epsilon=\mu=1$) cavity of dimensions $a \times b$. Discrete resonant frequencies of the system are given by

$$
f_{nm} = \frac{1}{2 \pi \sqrt{\epsilon_r \mu_r}} \sqrt{ k_x^2 + k_y^2 }
$$
$$
k_{x} = \frac{m \pi}{a}^2
k_{y} = \frac{n \pi}{b}^2
$$

where $n$ and $m$ are the mode numbers.

\section{Time Domain Signals}

\figHere{Chapters/Results/2D_FreeSpaceCavity/signal/p1_signal}{The signal obtained for the $E_1$ component using $p=1$ and a mesh of 4 and 8 quadrilaterals}{2DPECCavityFreqConv}
\figHere{Chapters/Results/2D_FreeSpaceCavity/signal/p3_signal}{The signal obtained for the $E_1$ component using $p=3$ and a mesh of 8 and 128 quadrilaterals. Signal can be seeon to contain a higher frequency content than p1 signals.}{2DPECCavityFreqConv}
\section{Spectra}
Spectra for comparison of $p=1$ and $p=3$ for small (1x2) and large (8x16) meshes.
\figHere{Chapters/Results/2D_FreeSpaceCavity/spectrum/spectrum_1x2_p1}{The spectrum obtained for the TE mode using $p=1$ and a mesh of 1x2 quadrilaterals}{2DPECCavityFreqConv}
\figHere{Chapters/Results/2D_FreeSpaceCavity/spectrum/spectrum_2x4_p1}{The spectrum obtained for the TE mode using $p=1$ and a mesh of 2x4 quadrilaterals}{2DPECCavityFreqConv}
\figHere{Chapters/Results/2D_FreeSpaceCavity/spectrum/spectrum_2x4_p3}{The spectrum obtained for the TE mode using $p=3$ and a mesh of 2x4 quadrilaterals}{2DPECCavityFreqConv}
\figHere{Chapters/Results/2D_FreeSpaceCavity/spectrum/spectrum_8x16_p3}{The spectrum obtained for the TE mode using $p=3$ and a mesh of 8x16 quadrilaterals}{2DPECCavityFreqConv}

\clearpage
\subsection{h-Convergence of Frequency}
Figure \ref{2DPECCavityFreqConv} shows the convergence of frequency obtained for a cavity with $a=0.5$ and $b=1.0$ for three differents meshes of quadrilateral elements.
%% NOTE::: HAVE I TRIED THIS WITH TRIANGLES

\figHere{Chapters/Results/2D_FreeSpaceCavity/freqConvergence/U2F1}{The convergence of absolute error in the frequency is shown for three different meshes using shape functions of order 1,2 and 3}{2DPECCavityFreqConv}
\figHere{Chapters/Results/2D_FreeSpaceCavity/periodConvergence/U2F1}{Convergence of absolute error in the frequency with the final time, $T$, of the simulation for a number of meshes and $p$ for \textbf{Frequency 1} for a rectangular PEC cavity}{2DPECCavityFreqConv}
\figHere{Chapters/Results/2D_FreeSpaceCavity/freqConvergence/U2F2}{The convergence of absolute error in the frequency is shown for three different meshes using shape functions of order 1,2 and 3}{2DPECCavityFreqConv}
\figHere{Chapters/Results/2D_FreeSpaceCavity/periodConvergence/U2F2}{Convergence of absol    ute error in the frequency with the final time, $T$, of the simulation for a number of     meshes and $p$ for \textbf{Frequency 2} for a rectangular PEC cavity}{2DPECCavityFreqConv}
\figHere{Chapters/Results/2D_FreeSpaceCavity/freqConvergence/U2F3}{The convergence of absolute error in the frequency is shown for three different meshes using shape functions of order 1,2 and 3}{2DPECCavityFreqConv}
\figHere{Chapters/Results/2D_FreeSpaceCavity/periodConvergence/U2F3}{Convergence of absol    ute error in the frequency with the final time, $T$, of the simulation for a number of     meshes and $p$ for \textbf{Frequency 3} for a rectangular PEC cavity}{2DPECCavityFreqConv}
\clearpage
\figHere{Chapters/Results/2D_FreeSpaceCavity/freqConvergence/U2F4}{The convergence of absolute error in the frequency is shown for three different meshes using shape functions of order 1,2 and 3}{2DPECCavityFreqConv}
\figHere{Chapters/Results/2D_FreeSpaceCavity/periodConvergence/U2F4}{Convergence of absol    ute error in the frequency with the final time, $T$, of the simulation for a number of     meshes and $p$ for \textbf{Frequency 4} for a rectangular PEC cavity}{2DPECCavityFreqConv}
\figHere{Chapters/Results/2D_FreeSpaceCavity/freqConvergence/U2F5}{The convergence of absolute error in the frequency is shown for three different meshes using shape functions of order 1,2 and 3}{2DPECCavityFreqConv}
\figHere{Chapters/Results/2D_FreeSpaceCavity/periodConvergence/U2F5}{Convergence of absol    ute error in the frequency with the final time, $T$, of the simulation for a number of     meshes and $p$ for \textbf{Frequency 5} for a rectangular PEC cavity}{2DPECCavityFreqConv}
\figHere{Chapters/Results/2D_FreeSpaceCavity/freqConvergence/U2F6}{The convergence of absolute error in the frequency is shown for three different meshes using shape functions of order 1,2 and 3}{2DPECCavityFreqConv}
\figHere{Chapters/Results/2D_FreeSpaceCavity/periodConvergence/U2F6}{Convergence of absol    ute error in the frequency with the final time, $T$, of the simulation for a number of     meshes and $p$ for \textbf{Frequency 6} for a rectangular PEC cavity}{2DPECCavityFreqConv}
\clearpage
\figHere{Chapters/Results/2D_FreeSpaceCavity/freqConvergence/U2F7}{The convergence of absolute error in the frequency is shown for three different meshes using shape functions of order 1,2 and 3}{2DPECCavityFreqConv}
\figHere{Chapters/Results/2D_FreeSpaceCavity/periodConvergence/U2F7}{Convergence of absol    ute error in the frequency with the final time, $T$, of the simulation for a number of     meshes and $p$ for \textbf{Frequency 7} for a rectangular PEC cavity}{2DPECCavityFreqConv}
\figHere{Chapters/Results/2D_FreeSpaceCavity/freqConvergence/U2F8}{The convergence of absolute error in the frequency is shown for three different meshes using shape functions of order 1,2 and 3}{2DPECCavityFreqConv}
\figHere{Chapters/Results/2D_FreeSpaceCavity/periodConvergence/U2F8}{Convergence of absol    ute error in the frequency with the final time, $T$, of the simulation for a number of     meshes and $p$ for \textbf{Frequency 8} for a rectangular PEC cavity}{2DPECCavityFreqConv}
\figHere{Chapters/Results/2D_FreeSpaceCavity/freqConvergence/U2F9}{The convergence of absolute error in the frequency is shown for three different meshes using shape functions of order 1,2 and 3}{2DPECCavityFreqConv}
\figHere{Chapters/Results/2D_FreeSpaceCavity/periodConvergence/U2F9}{Convergence of absol    ute error in the frequency with the final time, $T$, of the simulation for a number of     meshes and $p$ for \textbf{Frequency 9} for a rectangular PEC cavity}{2DPECCavityFreqConv}
\clearpage
\section{p3 Convergence Comparison}
4 runs have been performed for different meshes with $p=3$

Run 1 =  monitor at [ -1, 0.5 ] ctt=0.45.

Runs 2 + 3 = monitor at [ 0.5,0.5] ctt=0.1 (different final times).

Run 4 = mointor at [ 0.5,0.5 ] ctt=0.4

Error due to determining the peak?

\figHere{Chapters/Results/2D_FreeSpaceCavity/convergenceComparisonP3/ConvergenceforU2F5}{Convergence with p3 only - h=1.2 not converged for this frequency, but h=0.9 is}{label}
\figHere{Chapters/Results/2D_FreeSpaceCavity/convergenceComparisonP3/ConvergenceforU2F6}{Convergence with p3 only - h=1.2 not converged for this frequency, but h=0.9 is}{label}
\figHere{Chapters/Results/2D_FreeSpaceCavity/convergenceComparisonP3/ConvergenceforU2F7}{Convergence with p3 only - h=1.2 may not be converged}{label}
\figHere{Chapters/Results/2D_FreeSpaceCavity/convergenceComparisonP3/ConvergenceforU2F8}{Convergence with p3 only - h=1.2 may not be converged}{label}
\figHere{Chapters/Results/2D_FreeSpaceCavity/convergenceComparisonP3/ConvergenceforU2F9}{Convergence with p3 only - h=1.2 may not be converged}{label}
\figHere{Chapters/Results/2D_FreeSpaceCavity/convergenceComparisonP3/ConvergenceforU2F10}{Convergence with p3 only - h=1.2 may not be converged}{label}


\label{Chapter3}
\chapter{Free space circular PEC cavity} % Write in your own chapter title

\begin{itemize}
  \item Problem setup
  \item Convergence of error with simulation period
  \item dt study
  \item Scheme convergence results
\end{itemize}

In this chapter we validate our results using a 2-dimensional rectangular free space ($\epsilon=\mu=1$) cavity of dimensions $a \times b$. Discrete resonant frequencies of the system are given by

$$
f{nm} = \frac{1}{2 \pi \sqrt{\epsilon_r \mu_r}} k_{mp}
$$

$$
k_{mp}=\sqrt{ k_x^2 + k_y^2 }
$$
$$
k_{x} = \frac{m \pi}{a}^2
k_{y} = \frac{n \pi}{b}^2
$$

where $n$ and $m$ are the mode numbers.

\subsection{Spectra}
\figHere{Chapters/Results/2D_FreeSpacePECCircle/spectrums/U1}{Spectrum for $E_1$ component with $p=1$ elements with planar edges for different mesh refinements}{2DPECCavityFreqConv}
\subsection{h-Convergence of frequency}
Figure \ref{2DPECCavityFreqConv} shows the convergence of frequency obtained for a cavity with $a=0.5$ and $b=1.0$ for three differents meshes of quadrilateral elements.
%% NOTE::: HAVE I TRIED THIS WITH TRIANGLES

\figHere{Chapters/Results/2D_FreeSpacePECCircle/hConvergencePlots/combined}{Convergence of absolute error in the frequency with characteristic mesh size using exact geometry representation and planar faced elements with $p=1$ for a PEC circle. The expected gradient of 4 is obtained with the NEFEM mesh and a gradient of 2 is obtained with FEM.}{2DPECCavityFreqConv}

\figHere{Chapters/Results/2D_FreeSpacePECCircle/convPlotsAll/convAll_iComp2}{Convergence of absolute error in the frequency with p-refinement using NEFEM, planar edges and curved edge elements}{2DPECCavityFreqConv}

\figHere{Chapters/Results/2D_FreeSpacePECCircle/ndofConvergencePlots/p_vs_h_ndof/eps/F4} {Convergence of error in the frequency with number of degrees of freedom comparison between h-refinement with planar elements and p-refinement for a PEC circle. Both types of refinement are shown both with exact geometry representation and with curved high-order elements.}{2DPECCavityFreqConv}


\chapter{Dispersive 2D Cavity} % Write in your own chapter title
\label{Chapter3}
\lhead{Chapter 3. \emph{Dispersive 2D Cavity}} % Write in your own chapter title to set the page header

\begin{itemize}
	\item set up problem with volumetric source
	\item show convergence of wave shapes
\end{itemize}

\section{h-convergence with volumetric source}
\figHere{Chapters/Results/2D_DispersivePlate/convergenceAllComponents/eps/L2vsH_TE_single_comp}{L2 Norm convergence in time domain waveform for TE mode of dispersive plate}{label}
\figHere{Chapters/Results/2D_DispersivePlate/convergenceAllComponents/eps/L2vsH_TM_single_comp}{L2 Norm convergence in time domain waveform for TM mode of dispersive plate}{label}

\section{h-Convergence with frequency}

\chapter{3D Cube PEC Example} % Write in your own chapter title
\label{Chapter3}
\lhead{Chapter 3. \emph{3D Cube Example}} % Write in your own chapter title to set the page header

\figHere{Chapters/Results/3D_Cube/spectrums/spectrum_p3_HEX_H3}{Spectrum for $p=3$ H3 with hexahedral elements showing a component of $E$ and $H$}{2DPECCavityFreqConv}

\figHere{Chapters/Results/3D_Cube/convPlotsAll/HEXConvergenceforU1F2}{Hex convergence for $E_1$}{label}
\figHere{Chapters/Results/3D_Cube/convPlotsAll/HEXConvergenceforU1F5}{Hex convergence for $E_1$}{label}
\figHere{Chapters/Results/3D_Cube/convPlotsAll/HEXConvergenceforU1F6}{Hex convergence for $E_1$}{label}
\figHere{Chapters/Results/3D_Cube/convPlotsAll/HEXConvergenceforU1F7}{Hex convergence for $E_1$}{label}
\figHere{Chapters/Results/3D_Cube/convPlotsAll/HEXConvergenceforU1F8}{Hex convergence for $E_1$}{label}
\figHere{Chapters/Results/3D_Cube/convPlotsAll/HEXConvergenceforU1F9}{Hex convergence for $E_1$}{label}

\figHere{Chapters/Results/3D_Cube/convPlotsAll/TETConvergenceforU1F2}{Tet convergence for $E_1$}{label}
\figHere{Chapters/Results/3D_Cube/convPlotsAll/TETConvergenceforU1F5}{Tet convergence for $E_1$}{label}
\figHere{Chapters/Results/3D_Cube/convPlotsAll/TETConvergenceforU1F6}{Tet convergence for $E_1$}{label}
\figHere{Chapters/Results/3D_Cube/convPlotsAll/TETConvergenceforU1F7}{Tet convergence for $E_1$}{label}
\figHere{Chapters/Results/3D_Cube/convPlotsAll/TETConvergenceforU1F8}{Tet convergence for $E_1$}{label}
\figHere{Chapters/Results/3D_Cube/convPlotsAll/TETConvergenceforU1F9}{Tet convergence for $E_1$}{label}

Parallel: benchmark times for communication and calculation for different meshes + compare to overall performance plots should appear somewhere - should appear somewhere in results.
