% Chapter 1
\chapter{Numerical Method Review}
\label{NumericalMethodsChapter}
\lhead{Chapter 3. \emph{Numerical Method Review}}

Today there are several poular methods for solving maxwells equations. The one which remains in widespread use due to its simplicity and low operation count is the Yee Scheme \cite{YeeScheme} which uses a Finite Difference method. However the use of cartesian structured grids means that geometrical boundaries are subject to staircasing effects.

% jumping in too early with this...

Improved approximation of boundaries requires global refinement of the entire grid. It has been shown that small changes in the boundary representation for some electromagnetic problems can cause global changes in the solution \cite{RubenNEFEM}.

\subsection{Unstructured Mesh}
\begin{itemize}
  \item justify using FEM + limitations of FDTD
  \item unstructured mesh vs structured grid - stair-casing/geometric flexibility
  \item better represent boundaries with mesh refinement
  \item refine away from boundaries to capture solution not geometry
  \item hybrid meshes
\end{itemize}

Several techniques have been propsed to resolve these issues - notably the finite element and finite volume methods. These method allow the problem to be discretised on an unstructured mesh. One key advantage is that this allows local refinement near geometrical boundaries whilst still performing computations on a course mesh away from these boundaries. Refinement may also be used to capture the solution in regions where it is more complex without global refinement of the mesh.

However, a number of issues still remain with this approach. Boundaries are approximated by a linear approximation, which may include discontinuities in the boundaries or derivatives of the boundaries and can lead to non-physical effects []. Also the linear approximation of the solution means that these methods are subject to numerical disperion and dissipation which can cause significant errors in wave solutions propagated for long periods of time [].

\subsection{High Order FEM}
\begin{itemize}
	\item Capture geometric boundaries with elements with polynomial edges/faces
  \item interpolate soltn with polynomials
	\item still have non-physical effects due to inexact boundary representation - example global effect small change
	\item options for high order (FEM)
  \item numerical dispersion/dissipation - problem for long propagation
  \item limitations: sparse + global matrix, geometry representation, parallelisation
\end{itemize}

High-order methods allow boundaries to be approximated with polynomials by allowing elements to have curved faces - which greatly reduces staircasing effects []. Additionally the solution if approximated with a polynomial basis, whose order can be changed per element for local refinement. Higher order shape function have been shown to reduce dispersion and dissipation effects and allows, for smooth solutions, an improved accuracy for the same number of degrees of freedom. However this results in a sparse global mass matrix which needs to be inverted at each time step to solve the problem. This can be be very computationally expensive both in terms of CPU time required to invert the matrix and memory requirements.

Additionally for applications discussed in this thesis a long-running simulations are required [Chapter reference], therefore a method which can be parallelized would be desirable.

\subsection{Advantages Of Discontinuous Galerkin}

\begin{itemize}
  \item introduce method
	\item how does DG address the issues with FEM
  \item advantages for wave-dominated problems
	\item Capturing wave solutions + numerical dissipation/dispersion
	\item Influence of numerical flux
\end{itemize}

The Discontinuous Galerkin method uses a high-order polynomial basis for shape functions and and unstructured mesh - which for accurate approimation of boundaries. The method introduces additional degrees of freedom when compared with the finite element method, however this results in a block diagonal mass matrix, which can easily be inverted. Additionally the block diagonal structure means that computation can be distributed between a number of processors and performed in parallel.

% can I change p in different elements in normal finite element....?


% -------------
  

%The linear approximation of boundaries in low-order schemes has been shown to cause non-physical issues with the solution as a whole []. These low-order approximations of boundaries can include discontinuities in the boundaries themselves or derivaties of the boundaries (staircasing) which can lead to non-physical effects. High-order schemes allow us to approximate boundaries and interfaces with polynomials.

% \begin{itemize}
% \item Finite Difference, Finite Volume, Finite Element, Other High Order competitors to DG
% \end{itemize}
  
% OPTIONS LEFT: FINITE DIFFERENCE, FINITE VOLUME, FINITE ELEMENT
% Finite difference schemes seem very promising and are very popular however in many cases these have many limitations. In FDTD the entire computational domain is discretized using a cartesian structured grid. Complex geometrical boundaries or interfaces become hard to capture with structured grids - and improving the approximation of the boundary requires refiniment of the entire grid. Furthermore capturing solutions which are more complex in certain regions again require refinement over the whole domain. Clearly methods which use an unstructured mesh have a clear advantage since the geometry can be captured accurately with only local refinement of the mesh.

% OPTIONS LEFT: FINITE VOLUME (or other low order meshes), FINITE ELEMENT


% OPTIONS LEFT: FINITE ELEMENT (HIGH-ORDER)
% High-order FEM scheme have several issues having a sparse global matrix which needs to be inverted. The Discontinuous Galerkin method on the other hand has elemental matricies with elements connected by a numerical flux term. This means there are no large global matricies to invert. Also the approximations in high order are better however geometries are approximated by polynomials. This can still cause issues capturing geometries with a high curvature.


%The Discontinuous Galerkin method was first introduced in 1973 by Reed and Hill [W.H. Reed, T.R. Hill, Triangular mesh methods for the neutron transport equation, Los Alamos Scientific Laboratory, 1973 Tech. Rep. LA-UR-73-479] to solve the neutron transport equation.