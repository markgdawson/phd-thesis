% Chapter 1

\chapter{Numerical FDTD 1D Simulations} % Write in your own chapter title
\label{Chapter5}
\lhead{Chapter 5. \emph{Numerical FDTD 1D Simulations}} % Write in your own chapter title to set the page header

Numerical simulations were performed using a simple finite difference 1D time domain solver by propagating a wave of the form:

$$
sin(2 \pi f (t - x))
$$

The code used was a simple FDTD 1D implementation written in matlab.

The domain used for all simulations was [0,20] and the waves obtained were compared to the analytical solutions to produce an L2 norm of the resulting wave. The initial conditions used were zero in the whole domain and dirichlet boundary conditions were introduced on the left boundary - however as can be noted from the analytical form of the wave this does not introduce a discontinuity in the solution (at least not to first order). No boundary condition is set on the right hand side of the domain - but the simulation was stopped before the wave reached the far side.
All analysis was done on the 3rd component of the TE mode (E3) which is continuous.
%TODO - verify this...!
Unless otherwise specified the frequency used was f=2, the domain used is of length L=100 with final time at 0.9L with the signal captured at x=20.
% TODO - verify this...!

\begin{figure}
\includegraphics[width=\textwidth]{Figures/1D_FD_propagation/L2_norm_only_dt_refinement/aggregate.png}
\caption{Various mesh size were selected and kept constant while dt was refined. The lower bound on dt is set purely by computational time with no stability constraints. The L2 norm error both in the wave shape over the domain at the final time and in a signal sampled at a selected point at intervals of dt is shown.}
\label{dt_refinement}
\end{figure}

\begin{figure}
\includegraphics[width=\textwidth]{Figures/1D_FD_propagation/L2_norm_only_dx_refinement/aggregate_2.png}
\caption{A similar analysis to figure \ref{dt_refinement} shows dx refinement for fixed dt values. In this case the stability condition and the choice of dt sets a lower bound of dt (dt=dx).}
\end{figure}

% \begin{figure}
% \includegraphics[width=\textwidth]{Figures/1D_FD_propagation/L2_norm_only_dx_refinement/aggregate_10.png}
% \caption{dx refinement for various fixed dt values for frequency f=10. Note there is a lower bound of dt=dx in this case for stability.}
% \end{figure}
% 
% \begin{figure}
% \includegraphics[width=\textwidth]{Figures/1D_FD_propagation/L2_norm_only_dx_refinement/aggregate_20.png}
% \caption{dx refinement for various fixed dt values for frequency f=20. Note there is a lower bound of dt=dx in this case for stability.}
% \end{figure}

\begin{figure}
\includegraphics[width=\textwidth]{Figures/1D_FD_propagation/L2_norm_vs_dx_C_0_8/aggregate_2.png}
\caption{The L2 norm error in the wave and signal shapes with uniform dx and dt (by a factor of 2) refinement using a consisten value of C=0.8. The plots are shown against dx however there is also an implied refinement of dt.}
\end{figure}

\begin{figure}
    \begin{center}
        \subfigure[FFT calculations]{
            \label{FTTC}
            \includegraphics[width=0.6\textwidth]{Figures/1D_FD_propagation/Convergence_plots_with_C/error_from_spectrum_vs_c.png}
        }
        
        \subfigure[FDM calculations]{
            \includegraphics[width=0.6\textwidth]{Figures/1D_FD_propagation/Convergence_plots_with_C/error_from_fdm_vs_c.png}
            \label{FDMC}
        }
        \caption{Error in frequency calculation for different values of C. In this case the refinement was performed by setting dx=0.1 and choosing a dt for the selected C. Then allowing uniform refinement of dx and dt simultaneously by a factor of 2 to preserve C.}
    \end{center}
    
\end{figure}


\begin{figure}
\includegraphics[width=\textwidth]{Figures/1D_FD_propagation/Convergence_plots_with_dx/aggregate_2.png}
\caption{This plots show the relative error in frequency converging with dx for selected constant values of dt for a frequency f=2}
\end{figure}

\begin{figure}
\includegraphics[width=\textwidth]{Figures/1D_FD_propagation/Convergence_plots_with_dx/aggregate_20.png}
\caption{This plots show the relative error in frequency converging with dx for selected constant values of dt for a frequency f=20}
\end{figure}

\begin{figure}
\includegraphics[width=\textwidth]{Figures/1D_FD_propagation/Convergence_plots_with_dx/aggregate_10.png}
\caption{This plots show the relative error in frequency converging with dx for selected constant values of dt for a frequency f=10}
\end{figure}

\begin{figure}
\includegraphics[width=\textwidth]{Figures/1D_FD_propagation/PeriodStudy_long_run/FDM_and_FFT_errors.png}
\caption{A brief period study to select the time required for sufficient accuracy of FDM and FFT methods}
\end{figure}
