\chapter{Discontinuous Galerkin Method for Maxwells Equations} % Write in your own chapter title
\label{Chapter3}
\lhead{Chapter 3. \emph{Discontinuous Galerkin}} % Write in your own chapter title to set the page header

The Discontinuous Galerkin method was first introduced to solve the neutron transport problem by Reed and Hill \cite{} in 1973.
** Some background/literature review here Cockburn, Shui for solving hyperbolic equations etc ***
% PhDJesusAlvarez good for motivation - not so much for the method

As seen in Chapter~\ref{PhysicalProblemChapter}, given a suitable choice of
initial conditions the evolution of the system in time can be described by
Maxwell's curl equations in strong
form,~\eqref{eq:maxwell-curl-equations-conservation-form}. Consider that the problem is defined on an open bounded physical domain, $\physicalDomain \in \Real^{\nsd}$, with a boundary $\physicalDomainBoundary$,
% TODO - should I specify that its a PEC boundary (Ruben does)
which is partitioned in an unstructured mesh of $\nel$ nonoverlapping, body-conforming simplices, $\elemIndexed$, such that
$$
\bar{\physicalDomain} = \mathop{\bigcup}_{\elemindex=1}^{\nel} \barelemindexed
$$
such that $\elemIndexed \cap \anotherelem = \emptyset $ for $\elemindex \neq \anotherelemindex$, as shown in~\ref{fig:meshPartitionFigure}.
\begin{figure}
  \centering
  
  \caption{Partition of a two dimensional domain $\physicalDomain$ in $\nel$ disjoin triangular element}
  \label{fig:meshPartitionFigure}.
\end{figure}
% TODO - no approx here...!? Understand why this is...

% discuss the discretisation more - duplication of nodes?
Following the method of weighted residuals for a single element we seek solutions to the strong form problem,~\eqref{eq:maxwell-curl-equations-conservation-form}, over a generic element, $\genericElement$, in the approximation space $\left( \E,\H,\J \right) \in \approximationSpaceTotal \left( \left[ 0, \periodOfApproximation \right], \approximationSpaceSpatial \right)$
% TODO - check WHICH J I'm using here
where $\approximationSpaceSpatial = \spaceOfSpatialApproximationFunction $ with
% TODO - more space approx stuff here.
Multiplied by a vector of test functions $\TF \in \approximationSpaceSpatial $ and integrated over the generic element $\elemIndexed$ results in,
$$
\int_{\genericElement} \TF \cdot \uet \delem  + \int_{\genericElement} \TF \cdot \dpart{\Flux_{k}}{x_{k}}= \int_{\genericElement} \TF \cdot \maxwellSource \delem,
$$
where $\Ue$ denotes the restriction of $\USoltn$ to the element $\genericElement$ and einsteins summation notation has been employed.
% TODO - is this technically true
Integration by parts results in

\begin{equation}
\int_{\genericElement} \TF \cdot \uet \delem  - \int_{\genericElement} \dpart{\TF}{\xk} \cdot
\Flux_{k}(\Ue) \delem + \int_{\genericElementtrace} \TF \cdot \NormalFlux(\Ue) \delemtrace
= \int_{\genericElement} \TF \cdot \maxwellSource(\Ue) \delem,
\label{eq:weak-form-with-physical-flux}
\end{equation}

where $\outnormalvector$ is outward unit normal vector to the boundary $\elemtracegenericelement$ of $\genericElement$, which has directional cosines $\outnormalcoeffk$, and $\mathbf{F_n}$ is the outward normal physical flux, given by $ \mathbf{F_n}(\mathbf{U}) = \outnormalcoeffk \mathbf{F}_k(\mathbf{U}) $.

Since the weak form stated in~\eqref{eq:weak-form-with-physical-flux} is specified on the element $\genericElement$, this does not constitute a scheme suitable for solving the global problem. As is usual in a DG context, continuity of the solution between elements is weakly enforcing by replacing the physical normal flux, $\NormalFlux(\Ue)$, with a consistent numerical flux, $\NumFlux(\Ue,\Uout)$. This numerical flux depends on both the trace of the solution on the element $\genericElement$ and the trace of the solution on the neighbouring element, $\mathbf{U}^{out}$.
% TODO: What is the definition of Uout in mathematical language ??
For a linear hyperbolic system a natural choice is to employ a flux splitting technique~\cite{donea2003finite}, corresponding to an upwind approximation\cite{chen2005high}.
% TODO: this is a lot like Rubens/Obays paper, rewrite it. QUOTE: A natural choice, for the linear hyperbolic system of interest here, is to employ a flux splitting technique ~\cite{donea2003finite}, which corresponds to an upwind approximation \cite{chen2005high}
The physical normal flux is written as
$$
\NormalFlux(\USoltn) = \An \USoltn,
$$
where $\An = \outnormalcoeffk \Ak$, and decomposed into incoming (superscript -) flux and outgoing flux (superscript +) as
% VIVA: An is known as the jacobian matrix, and apparently is also \An = \frac{ \partial \NumFlux }{ \partial \U }
\begin{align}
\NormalFlux(\USoltn) = \NormalFluxPositiveEigenvalues(\USoltn) + \NormalFluxNegativeEigenvalues(\USoltn),
\label{eq:phys-flux-splitting}
\end{align}
where $ \NormalFluxPositiveEigenvalues = \AnPlus \USoltn$, $\NormalFluxNegativeEigenvalues = \AnMinus \USoltn, $ and the matrices $\AnMinus$ and $\AnPlus$ denote respectively the matrices of the positive and negative eigenvalues of $\An$. These can be written conveniently as
\begin{align}
  \label{eq:AnMinus-AnPlus-Definition}
\AnPlus &= \left( \An + \AnMod \right) / 2   \\
\AnMinus &= \left( \An - \AnMod \right) / 2
\end{align}
where %TODO: definition of AnMod and change AnMod to another symbol...Add the definition of An...
The choice of an upwind approximation results in
\begin{align}
\NumFlux(\USoltn) = \NormalFluxPositiveEigenvalues(\USoltn) + \NormalFluxNegativeEigenvalues(\Uout).
\label{eq:num-flux-splitting}
\end{align}
After substitution of $\NumFlux$ and integration by parts,~\eqref{eq:weak-form-with-physical-flux} becomes
\begin{align*}
\int_{\genericElement} \TF \cdot \uet \delem  + \int_{\genericElement} \TF \cdot \dpart{\Flux_{k}(\Ue)}{\xk} \delem + \int_{\genericElement} \TF \cdot \left[ \NumFlux(\Ue,\Uout) - \NormalFlux(\Ue) \right] \delemtrace \\
= \int_{\genericElement} \TF  \cdot \maxwellSource(\Ue) \delem, \label{eq:weak-form-upwind-splitting-fluxes}
% the second term changes from + to - from prev weak form
\end{align*}
We note that % TODO - I've lost some text here...!
\begin{equation}
  \NumFlux(\Ue,\Uout) - \NormalFlux(\Ue) = \AnMinus \Uout - \AnMinus \USoltn = \AnMinusU, \label{eq:AnMinusUderivation}
\end{equation}
where $\JumpU = \Uout - \USoltn$ denotes the jump in the solution across $\elemtracegenericelement$. By substitution into~\eqref{eq:weak-form-upwind-splitting-fluxes}, the weak form with upwind flux splitting is written as
\begin{equation}
\int_{\genericElement} \TF \cdot \uet \delem  + \int_{\genericElement} \TF \cdot \dpart{\Flux_{k}(\Ue)}{\xk} \delem + \int_{\elemtracegenericelement} \TF \cdot \AnMinusU \delemtrace = \int_{\genericElement} \TF  \cdot \maxwellSource(\Ue) \delem,
\label{eq:weak-form-final}
\end{equation}

% TODO: missing some stuff on diagonalisation of A here....is it necessary, also the form of the
% positive and negative eigenvalues.
% TODO: Missing the form of the numerical flux for DG!!
% *** conditions to be satisfied by numerical flux + form of numerical flux ***
% TODO: upwind flux: effect for wave domination problems - flow of information, upwind flux

\section{Internal Element Boundaries}
We consider the boundary conditions at an internal boundary between elements. From~\ref{sec:conservation-form}, this expression may be written as
$$
  \An =
  \begin{pmatrix}
 & \zerom , & \mu^{-1} \RTotNorm, & \zerom \\
 & - \eps^{-1} \RTotNorm & \zerom & \zerom \\
 & \zerom & \zerom & \zerom 
 & \end{pmatrix} ,
$$
where
$$
  \RTotNorm =
  \begin{pmatrix}
 & 0 & n_3 & -n_2 \\
 & -n_3 & 0 & n_1 \\
& n_2 & -n_1 & 0 
 & \end{pmatrix} .
$$
The modulus of $\An$ is then given by
\begin{align*}
\AnMod = \speedoflight
\begin{pmatrix}
  \modAnSubMatrix & \zerom & \zerom \\
  \zerom  & \modAnSubMatrix & \zerom \\
   \zerom & \zerom & \zerom 
\end{pmatrix}
\end{align*}
where $\speedoflight = \left( \eps \mu  \right)^{-\frac{1}{2}}$ is the speed of light in the medium and
\begin{align*}
  \modAnSubMatrix = 
\begin{pmatrix}
\outnormalcoeff_2^2 + \outnormalcoeff_3^2 &      -\outnormalcoeff_1 \outnormalcoeff_2 &      -\outnormalcoeff_1 \outnormalcoeff_3 \\
-\outnormalcoeff_1 \outnormalcoeff_2 & \outnormalcoeff_1^2 + \outnormalcoeff_3^2 &      -\outnormalcoeff_2 \outnormalcoeff_3 \\
-\outnormalcoeff_1 \outnormalcoeff_3 &      -\outnormalcoeff_2 \outnormalcoeff_3 & \outnormalcoeff_1^2 + \outnormalcoeff_2^2 \\
\end{pmatrix} ,
\end{align*}
where the identity $\sqrt{\sum_{\outnormalcoeffcomp} \outnormalcoeffk^2} = 1$, for the unit vector $\outnormalvector$, has been used. We therefore write
\begin{align*}
\AnMinus = \speedoflight
\begin{pmatrix}
  -\modAnSubMatrix & \RTotNorm & \zerom \\
  -\RTotNorm  & -\modAnSubMatrix & \zerom \\
   \zerom & \zerom & \zerom 
\end{pmatrix} .
\end{align*}

Noting that $\RTotNorm \anyVector = \outnormalvector \times \anyVector$ and $\modAnSubMatrix \anyVector = \outnormalvector \times \left(  \outnormalvector
  \times \anyVector \right)$,for any vector $\anyVector$, results in the expression
\begin{align}
\AnMinusU = \frac{1}{2}
\begin{pmatrix}
  -\nvect \times \left( \JumpH + \sqrt{\frac{\eps}{\mu}} \nvect \times \JumpE \right) \\
   \nvect \times \left( \JumpE + \sqrt{\frac{\mu}{\eps}} \nvect \times \JumpH \right) \\
  \zerov
\end{pmatrix} .
  \label{eq:AnMinuU-expression-3D}
\end{align}
% Should I write out A_n for the TEz and TMz modes?
A similar procedure results in
\begin{align}
\AnMinusU =
  \frac{1}{2}
  \left[
    \Jump{H_3} - \sqrt{\frac{\eps}{\mu}} \Jump{\alphaGeneral}
  \right]
\begin{pmatrix}
   -\outnormalcoeff_2 \\
   \outnormalcoeff_1 \\
   - \sqrt{ \frac{\mu}{\eps} } \\
   0  \\
   0 
\end{pmatrix} . \label{eq:AnMinuU-expression-TE}
\end{align}
with $ \alphaGeneral = n_1 E_2 - n_2 E_1. $ for the $\TEz$ mode and
\begin{align}
\AnMinusU =
  \frac{1}{2}
  \left[
    \Jump{E_3} - \sqrt{\frac{\mu}{\eps}} \Jump{\alphaGeneral}
  \right]
\begin{pmatrix}
   \outnormalcoeff_2 \\
   -\outnormalcoeff_1 \\
   -\sqrt{ \frac{\eps}{\mu} } \\
   0 
\end{pmatrix} . \label{eq:AnMinuU-expression-TM}
\end{align}
for the $\TMz$ mode with $ \alphaGeneral = - n_1 H_2 + n_2 H_1. $

% VIVA: calculated with matlab script:
% modAn = sqrtm(An*An') and knowing that
% make sure I can do sqrtm by hand
\section{Spatial Discretisation}
This section describes two approaches to spatial discretisation of the weak formulation~\ref{eq:weak-form-final}, namely the traditional isoparametric finite element formulation and the recently proposed NURBS-enhanced finite element method (NEFEM).

\section{Isoparametric Finite Element Method}
\label{sec:isoparametric-elements}
A nodal interpolation of the solution, $\USoltn$, is defined in a reference element $\refelem$, with local coordinates $\refElemCoords$, as
% TODO: do this directly in the 
\begin{align}
\USoltn_e(\xbf,\t) \simeq \sum_{j=1}^{\nen} \SF_{j} (\xbf) \UVect_{j}(\t) ,
\label{eq:nodal-basis-defn}
\end{align}
% TODO - copy this directly from Ruben Paper!! COPYRUBEN
% VIVA: the second U_j is the vector of coefficients
where $N_{j}$ are polynomial, Lagrangian shape functions of order $p$ in $\refElemCoords$ and $\nen$ are the number of nodes of the element $\meshelem$. An isoparametric transformation, $\IsoMapping$, is introduced between the reference element, $\refelem$ and a generic mesh element $\meshelem$, namely
% TODO - COPYRUBEN -> iso mapping - add comma to the end
where $\nodalCoordinatesOfElement$ are the nodal coordinates of the element $\meshelem$. Note that for an element with a face or edge on the boundary of the computational domain, the boundary of the element $\meshelem$ will be a polynomial approximation of the real boundary\cite{}.
% TODO - COPY RUBEN for citation
 Following the Galerkin method, the vector of test functions, $\TF$, is chosen from the same basis as the shape
functions.
Substitution of~\eqref{eq:nodal-basis-defn} into~\eqref{eq:weak-form-final}, and selecting the space of test functions to be the same as the space spanned by the approximation functions, results in the system
% NOTE - the second term actually also has an implicit sum over k in (Einstein notation)
$$
\discretisedWeakForm{\nfn}
\label{eq:discretised-weak-form-fem}
$$
% TODO - what is the ij subscript, and also why is the An taken out of the face matrix if it isn't constant over the face?
of $\nen$ ordinary differential equations, for each node $i$, where $\UVect_j$ is a vector of the solution coefficients at the $j$th node, $\MassMatrix$ is the elemental mass matrix, $\IdentityMatrix$ is the identity matrix, $\ConvectionMatrix$ is the convection matrix in the direction $x^{k}$, $\MassMatrixFace$ is the face mass matrix and $\nfn$ denotes the number of face nodes. Note that a choice of Lagrangian, nodal shape functions results in a block diagonal elemental mass matrix. Additionally, an isoparametric mapping results in the restriction of the index $\faceindex$ to face nodes only, since other terms are zero. These matrices are defined by
\begin{align*}
\MassMatrix &= \int_{\meshelem} \SF_i \SF_j \delem \\
\ConvectionMatrix &= \int_{\meshelem} \SF_i \dpart{\SF_j}{\xk}\delem \\
\MassMatrixFace &= \int_{\meshelemtrace} \SF_i \SF_j \delemtrace
\end{align*}
% TODO - why is Ruben as a captial here? And why is M_{ij} captial...none of this make sense...!
% TODO - why dont I change the integration element here?

Using the isoparametric mapping~\eqref{eq:isoparametric-mapping}, the integrals over $\meshelem$, are transformed to the reference element, $\refelem$, as
\begin{align*}
\MassMatrix &= \int_{\refelem} \SF_i \SF_j |\Jacobian| \delem \\
\ConvectionMatrix &= 
                               \sum_{l=1}^{\nsd}
                                \int_{\refelem} \SF_i
                               \Jacobian_{lk}^{-1}
                               \dpart{\SF_j}{\xi_{l}}
                             |\Jacobian|
                             \delem
\end{align*}
where $\Jacobian = \frac{\partial \phi}{ \partial \xi}$, is the Jacobian of the mapping $\IsoMapping$. Similarly, the face mass matrix is evaluated as
\begin{align*}
\MassMatrixFace &= \int_{\refelemtrace} \SF_i \SF_j || \JacobianFace || \delemtrace
\end{align*}
% TODO - double lines here around HJacobian Face A) insert them properly B) what are they??
where $\JacobianFace$ is the Jacobian of the restriction of the isoparametric mapping to the element face.

% Wikipedia: 'In geometry, an affine transformation, affine map[1] or an affinity (from the Latin, affinis, "connected with") is a function between affine spaces which preserves points, straight lines and planes.'
For elements for which the transformation $\IsoMapping$ is affine, both $\Jacobian$ and $\JacobianFace$ are scalar constants over the element, and therefore the elemental matrices simplify to
\begin{align*}
\MassMatrix &= |\Jacobian| \MassMatrixAffine \\
\ConvectionMatrix &= |\Jacobian|
                               \sum_{l=1}^{\nsd}
                               \Jacobian_{kl}^{-1}
                               \ConvectionMatrixAffine \\
\MassMatrixFace &= || \JacobianFace || \MassMatrixFaceAffine
,
\end{align*}
% TODO - double lines here around Jacobian Face
with the reference elemental matrices
\begin{align*}
\MassMatrixAffine &= \int_{\refelem} \SF_i \SF_j \drefelem \\
\ConvectionMatrixAffine &= \int_{\refelem} \SF_i
                             \left(
                               \dpart{\SF_j}{\xi_{l}}
                             \right)
                             \drefelem \\
\MassMatrixFaceAffine &= \int_{\refelemtrace} \SF_i \SF_j \delemtrace
,
\end{align*}
which depend only on the shape functions and the geometry of $\refelem$. The reference elemental matrices $\MassMatrixAffine$, $\ConvectionMatrixAffine$ and $\MassMatrixFaceAffine$ can be computed \textit{a priori} in the reference element, inverted if necessary and stored. Computation of $\ConvectionMatrix$, $\MassMatrixFace$ and the inverse of $\MassMatrix$ for each element therefore reduces to a multiplication of the reference element matrix by a scalar. Whenever possible meshes are constructed in order to maximise the number of elements for which an affine mapping can be established between $\refelem$ and $\meshelem$. For curved elements, since the $|\Jacobian|$ and $|\JacobianFace|$ are not constant, it is not possible to precompute reference element matrices. In practical applications however, meshes are constructed where with an extremely low number of curved elements. In these cases the elemental matrices $\MassMatrix$, $\ConvectionMatrix$ and $\MassMatrixFace$ are computed and stored once per element.
The resulting implementation, in which numerical integration is performed \textit{a priori}, is known as the \textit{quadrature-free} implementation\cite{DGPaper:42}, and can reduce the cost of a high-order DG method by a factor of 100\cite{DGPaper:41}.
% VIVA : no loop on gauss points now!!
% TODO / VIVA : what about factoring \AnMinu out of the integrand because n is constant over a planar face????
% TODO - also cite Rubens paper here: The use of hybrid meshes to improve the efficiency of a discontinuous Galerkin
% TODO - what is this whole thing with triangular/tetrahedral meshes -> always constant jacobian unless curved, Quads -> not always constant jacobian unless curved, should I elabourate?


% TODO - 'isoparametric mapping given by the coords of the vertices of Omega
%%%  In order to perform the integration over the element interior an isoparametric
%%% mapping, $\IsoMapping$, is introduced from a reference element $\refelem$ to the
%%% physical element $\genericElement$. The integrals in~\eqref{}, once transformed to the
%%% reference element, become
% TODO - ruben has lk not lk on Jacobian matrix
% VIVA - make sure I know what this definition of J means
% TODO - easier way of saying this
% TODO - comment (remark) on affine elements, and how I can remove the jcaobian etc etc




\section{NURBS-enhanced finite element method (NEFEM)}
% TODO - polynomial approximation of the boundary stuff here (take it from the paper maybe)
In the NEFEM approach, the approximation is defined directly in the physical space in Cartesian coordinates, as
\begin{align}
\USoltn_e(\xbf,\t) \simeq \USoltnApprox = \sum_{j=1}^{\nen} \SF_{j} (\xbf) \UVect_{j}(\t) ,
\label{eq:nodal-basis-defn}
\end{align}
% TODO - I haven't defined U_h here...!
% TODO - duplicate definition of xbf? maybe?
% VIVA: the second U_j is the vector of coefficients
% TODO / VIVA - do these have to be Lagrangian at this stage? Can they be modal?
where $N_{j}$ are polynomial, Lagrangian shape functions of order $p$, defined in $\xbf$, and $\nen$ are the number of nodes of the element $\meshelem$. Substitution of \eqref{eq:nodal-basis-defn} into the weak form,~\eqref{eq:weak-form-final}, and again selecting the space of test functions to be the same as the space spanned by the approximation functions, results in the system
% TODO...
Note that for an element with a face or edge on the boundary of the computational domain, the boundary of the element $\meshelem$ will be a polynomial approximation of the real boundary\cite{}.
% TODO - COPY RUBEN for citation
 Following the Galerkin method, the vector of test functions, $\TF$, is chosen from the same basis as the shape
functions.
Substitution of~\eqref{eq:nodal-basis-defn} into~\eqref{eq:weak-form-final}, and selecting the space of test functions to be the same as the space spanned by the approximation functions, results in the system
% NOTE - the second term actually also has an implicit sum over k in (Einstein notation)
$$
\discretisedWeakForm{\nen}
\label{eq:discretised-weak-form-nefem}
$$
of ordinary differential equations for each node $i$ of the element $\genericElement$. Due to the definition of the approximation in physical space, the summation over the index $\faceindex$ in~\eqref{eq:discretised-weak-form-nefem} is no longer restricted to face nodes as it is in~\eqref{eq:discretised-weak-form-fem}. The elemental matrices are defined as
\begin{align*}
\MassMatrix &= \int_{\genericElement} \SF_i \SF_j \delem \\
\ConvectionMatrix &= 
               \int_{\genericElement}
               \SF_i
               \dpart{\SF_j}{\xi_{l}}
               \delem \\
\MassMatrixFace &= \int_{\elemtracegenericelement} \SF_i \SF_j \delemtrace
\end{align*}
% TODO - what is the ij subscript, and also why is the An taken out of the face matrix if it isn't constant over the face?
A detailed discussion and comparison of different strategies for computing integrals whilst accounting for the exact boundary representation are presented in\cite{DGPaper:38}. In this work it was concluded that composite Gaussian quadratures are the most efficient option. In addition, for the same mesh and degree of approximation NEFEM only requires one or two more integration points to obtain its maximum accuracy, which is significantly higher than the obtained by standard FEM with curved elements. 
% TODO - why is Ruben as a captial here? And why is M_{ij} captial...none of this make sense...!
% TODO - why dont I change the integration element here?
% RUBEN - says 'the element face f'

%% \begin{align*}
%% \sum_{j=1}^{\nen} \left[
%%   % term 1
%%   \left(
%%     \int_{\genericElement} \SF_i \SF_j \delem
%%   \right)
%%   \dodet{\UVect_j}
%% +
%%   % term 2
%%   \left(
%%     \int_{\genericElement} \SF_i \dpart{\SF_j}{x^k} \delem
%%   \right)
%%   \Ak \UVect_j
%% +
%%   % term 3
%%   \left(
%%   \int_{\elemtrace} \SF_i \SF_j \delemtrace 
%%   \right)
%%   \AnMinus \JumpUCoeffVectUnknownsWithIndex{j}
%%   % term 4
%% -
%%   \left(
%%   \int_{\genericElement} \SF_i \SF_j \delem
%%   \right)
%%   \Asource 
%%   \right]  = 0
%% %
%%   \; \; \forall \; i \; \in \; 1...\nen,
%% \end{align*}
% NOTE - again, second has implicit sum
% NOTE - in Ruben/Oubay paper MI is used instead of M
% TODO - A_s has not been defined....! Also am I missing a U to multiply A_s?

% TODO - this happens because by definition other face nodes are zero on all nodes
% except the ones not on a face...

% TODO: what is all this crap about M_{ij} and how does it correspond to residual?
% TODO: how should I write residual vector now?

\section{Time Integration}
The solution is advanced in time with an explicit fourth order Runge-Kutta (RK4)
method. The time step is selected to be sufficiently small that the numerical
error is dominated by the error in spatial discretisation. Implicit schemes which allow larger time steps may be employed to obtain the final solution of the system of equations in a shorter computational time. However, as shown in~\autoref{Ch:SignalAnalysis}, methods to extract frequency domain information from a time domain signal have an upper frequency limit inversely proportional to time step.
% TODO: RK4 Stability condition 
% TODO - justify high order time integration....!

% TODO: R/O Quote: 'Triangles and quadrilaterals are employed to provide a consistent
%% discretisation of the spatial solution domain, X, for two
%% dimensional problems. In three dimensions, consistent meshes
%% consisting of tetrahedra, hexahedra, prisms and pyramids are used.
%% Apart from the pyramid, which requires special attention, optimal
%% nodal finite elements of arbitrary order are readily defined for all
%% these shapes. For the pyramid, a recently proposed approximation
%% space [28] is adopted. This space is well suited for both continuous
%% and discontinuous approximations and is optimal, i.e. the a priori
%% error estimate is Oðhpþ1 Þ in the L2ðXÞ norm, where p denotes the
%% order of the approximation. The approximation spaces that are employed
%% are summarised in Table 1' <----


\section{Quadrature}

-> quatrature
  -> optimal interpolation points proposed in \cite{DGPaper:39} and the technique proposed in \cite{DGPaper:20} for constructing high-order polynomial basis functions and their derivatives
  -> The quadratures employed to integratet the terms of the weak form correspond to the integration rules recently proposed in \cite{DGPaper:40}
  -> The number of integration points is seected so that exact integreation is achieved for polynomials of order less than or equal to 2p + 1 
%TODO - Another thing to mention now is about the choices of quadrature (i.e. summations over gauss points)
%%  R/O Quote:
%%   For quadrilateral and hexahedral elements, quadrature
%%  based on the tensor product of well known one–dimensional Gauss–Legendre
%%  rules is readily implemented for any order of approximation. Note, however,
%%  that other quadrature formulae, with fewer integration points, exist [31, 32].
%%  For triangles, specific quadrature rules, such as the symmetric quadrature
%%  proposed in [33, 34], are used. Analogously, efficient specific quadrature
%%  rules are used for tetrahedra, prisms and pyramids [34, 35].


%  * isoparametric only capture *roughly* the geometry....
%  * curved elements used both for higher accuracy or for curved boundaries...
%     -> can use planar high order elements...dont confuse the two
%  * p-extension of FEM reference (Ruben has Szabo and Babuska,1991)
%  
%  planar mesh -> 'poly order of the approximation is increased' -> to get to the desired error
%  
%  This is great...but in some cases...
%  but... geometric accuracy 'deteriorates the solution'...
%  isoparametric causes this...i.e the nodes are correct but between them is interpolation...
%  LOADS OF REFERENCES HERE (Sevilla stuff)
% Fekette + matrix condition number
%  higher geom accuracy
%  There is a lot of stuff in Rubens thesis about curved elements...what is he on about??
% VIVA: parallel is also cool for storage...possibly...


% TODO: Fekette nodal distribution: Show some figures of Fekette distributions in reference element?
%% R/O Quote: In two dimensions, a Fekete nodal distribution is adopted for the triangle
%% [29] and a tensor product of one dimensiona In three dimensions, the nodal distributions proposed in [30] for the tetrahedron and
%% in [28] for the pyramid are used. A tensor product of one dimensional Fekete
%% nodal distributions is used for the hexahedron and a tensor product of triangular
%% and one dimensional Fekete nodal distributions is used for the prism.l Fekete nodal distributions for the quadrilateral.

\subsection{Jump Conditions}
% TODO - references -> LeVeque/Donea + Huerta(2005)
For interfaces which intersect the domain boundary, $\partial \Omega$, the not all components of $\mathbf{U}^{out}$ are determined by the boundary conditions on the interface. For a system of conservation laws Rankine-Hugoniot jump conditions of the form
\begin{align}
\Jump{ \mathbf{F}_n } = \lambda_j \Jump{ \mathbf{U} } \label{eq:rankine-hugoniot}
\end{align}
where $\lambda_j$ are the eigenvalues of the jacobian matrix $\mathbf{A}_n$.
This condition should be satisfied along the characteristics in the phase plane.
For the 3 dimensions these are $ \lambda_{ 1,2 } = - \speedoflightleft $, $
\lambda_{ 3,4 } = \speedoflightright $ and $\lambda_{5..9} = 0 $, where the
$\speedoflightleft$ and $\speedoflightright$ are the velocities of the
electromagnetic wave in media on the left and right side of the interface
respectively. This condition should be satisfied along the phase plane
characteristics, as shown in \ref{fig:phase-plane-characteristics}.

\begin{figure}[h]
  \centering
  
  \caption{Phase plane diagram showing the characteristics for Maxwells' equations}
  \label{fig:phase-plane-characteristics}
\end{figure}

\subsection{Rankine-Hugoniot Jump Conditions in 3D}
For the three dimensional case the normal flux can be written as
\begin{align*}
\NormalFlux =
\begin{pmatrix}
- \outnormalvector \times \H \\
\outnormalvector \times \E
\end{pmatrix}
\end{align*}
Applying the condition~\eqref{eq:rankine-hugoniot} and solving the resulting linear system results in
\begin{align}
  \mu^{out} \H^{out} - \mu_L \H^{L} = - \frac{1}{\speedoflightleft} \outnormalvector \times \left( \E^{out} - \E^{L} \right)
  \label{eq:jump-condition-resulting-equation-system-3D-1} \\
  \mu^{R} \H^{R} - \mu_{out} \H^{out} = \frac{1}{\speedoflightleft} \outnormalvector \times \left( \E^{R} - \E^{*} \right)
  \label{eq:jump-condition-resulting-equation-system-3D-2} \\
\end{align}
% TODO - I got this from Mar thesis, have not verified it
\subsection{Rankine-Hugoniot Jump Conditions in 2D}
For the $\TEz$ and $\TMz$ modes, by setting..., the normal physical flux can be
written in the form
$$
\NormalFlux = 
\begin{pmatrix}
  - n_2 \UFieldComp_3 \\
  n_1 \UFieldComp_3 \\
  \alphaGeneral
\end{pmatrix}
$$
where $ \alphaGeneral = n_1 \UFieldComp_2 - n_2 \UFieldComp_1 $, and the vector $\UField$ is given by $
\UField = 
  \begin{pmatrix} 
    E_1 \; E_2 \; H_3
  \end{pmatrix}^T
$ for the $\TEz$ mode, and $
\UField = 
  -
  \begin{pmatrix} 
    H_1 \; H_2 \; E_3
  \end{pmatrix}^T
  $ for the $\TMz$ mode.
For dispersive media this will be expanded with zeros...
% TODO - CHECK OUT THE A_n vectors and check that these F_n come nicely from there...
% TODO - should I have a numbered lambda here? and would that be different for
% TE and TM modes
By applying the Rankine-Hugoniot condition along $\dpart{x}{t} = 0$, with $ \Jump{ \mathbf{F}_n } = 0 $, the normal flux in region $\LStar$
and $\RStar$ are equal, we denote the flux in these region as
$$
\mathbf{F}_n^{out} = 
\begin{pmatrix}
  - n_2 \UFieldComp_3^{out} \\
  n_1 \UFieldComp_3^{out} \\
  \alphaGeneral^{out}
\end{pmatrix} .
$$

Along $\dpart{x}{t} = \speedoflightright$, the Rankine-Hugoniot condition takes the form
$$
\Jump{ \mathbf{F}_n } = \speedoflightright \Jump{ \mathbf{U} }
$$
which results in
\begin{align}
-n_2 \left(V_3^{R} - V_3^{out} \right) &= \micoeff_R \left( V_1^{R} - V_1^{out} \right) \label{RHR-sys-1} \\
n_1 \left(V_3^{R} - V_3^{out} \right) &= \micoeff_R  \left( V_2^{R} - V_2^{out} \right) \label{RHR-sys-2} \\
\micoeff_R \left(\alphaGeneral^{R} - \alphaGeneral^{out} \right) &= \left( V_3^{R} - V_3^{out} \right) \label{RHR-sys-3}
\end{align}
where $\micoeff = \sqrt{\eps/\mu}$ for $\TEz$ mode and $\micoeff =
\sqrt{\mu/\eps}$ for $\TMz$ mode. It can be easily verified that
\eqref{RHR-sys-1} and~\eqref{RHR-sys-2} imply~\eqref{RHR-sys-3}.

Along the $\dpart{x}{t} = - \speedoflightleft$, the Rankine-Hugoniot condition
takes the form
$$
\Jump{ \mathbf{F}_n } = - \speedoflightleft \Jump{ \mathbf{U} }
$$
which results in
\begin{align}
-n_2 \left(V_3^{L} - V_3^{out} \right) &= - \micoeff_L \left( V_1^{L} - V_1^{out} \right) \label{RHL-sys-1} \\
n_1 \left(V_3^{L} - V_3^{out} \right) &= - \micoeff_L  \left( V_2^{L} - V_2^{out} \right) \label{RHL-sys-2} \\
- \micoeff_L \left( \alphaGeneral^{L} - \alphaGeneral^{out} \right) &= \left( V_3^{L} - V_3^{out} \right) \label{RHL-sys-3}
\end{align}
As above, \eqref{RHL-sys-1} and \eqref{RHL-sys-2} imply \eqref{RHL-sys-3}.
For the $\TEz$ and $\TMz$ modes solving the Ranking-Hugoniot condition is equivalent to solving the system
\begin{align}
\micoeff_R \left(\alphaGeneral^{R} - \alphaGeneral^{out} \right) &= \left( V_3^{R} - V_3^{out} \right) \\
- \micoeff_L \left( \alphaGeneral^{L} - \alphaGeneral^{out} \right) &= \left( V_3^{L} - V_3^{out} \right)
\end{align}
Solving the linear system results given by \eqref{RHL-sys-3} and
\eqref{RHR-sys-3} results in
% divide by c* coeff then 1-2 again
% 1 - 2 and rearrange
\begin{equation}
\alphaGeneral^{out} = \frac{\UFieldComp_3^R - \UFieldComp_3^L - \micoeff_R \alphaGeneral^R - \micoeff_L \alphaGeneral^L}{\micoeff_L + \micoeff_R}\label{eq:interfce-bc-alphaGeneral}
\end{equation}
and
\begin{equation}
  \UFieldComp_3^{out} =
  \frac{
    \micoeffinv_R \UFieldComp_3^R + \micoeffinv_L \UFieldComp_3^L - \left(\alphaGeneral^R - \alphaGeneral^L \right)
  }{
    \micoeffinv_L + \micoeffinv_R
    } , \label{eq:interface-bc-V3}
\end{equation}
where $\micoeffinv = 1 / \micoeff$.

\subsection{Material Interfaces}
The expressions for the jump of the solution at a material interface are given by $\Jump{\E} = \E^{out} - \E^{L}$ and $\Jump{\H} = \H^{out} - \H^{L}$ for the left element and $\Jump{\E} = \E^{out} - \E^{R}$ and $\Jump{\H} = \H^{out} - \H^{R}$ for the right element.
% TODO - do I need a jump J? as well
Solving the linear system given by~\eqref{eq:jump-condition-resulting-equation-system-3D-2} and~\eqref{eq:jump-condition-resulting-equation-system-3D-2} results in
\begin{align}
 \outnormalvector \times \E^{out} = \outnormalvector \times \frac{
  \left( 
\speedoflightleft \eps_L \E^{L} - \outnormalvector \times \H^{L}
 \right)
  +
  \left( 
\speedoflightright \eps_R \E^{R} + \outnormalvector \times \H^{R}
 \right)
}{
  \speedoflightright \eps_R + \speedoflightleft \eps_L
} \label{eq:material-interfaces-1} \\
 \outnormalvector \times \H^{out} = \outnormalvector \times \frac{
  \left( 
\speedoflightleft \eps_L \H^{L} + \outnormalvector \times \E^{L}
 \right)
  +
  \left( 
\speedoflightright \eps_R \H^{R} - \outnormalvector \times \E^{R}
 \right)
}{
  \speedoflightright \eps_R + \speedoflightleft \eps_L
} \label{eq:material-interfaces-1}
\end{align}
The resulting expression for $\outnormalvector \times \Jump{E}$ and $\outnormalvector \times \Jump{H}$ are substituted into~\eqref{eq:AnMinuU-expression-3D} in order obtain an expression for numerical flux on the interface.
For the $\TEz$ mode solving the system of equations given by \eqref{eq:interfce-bc-alphaGeneral} and~\eqref{eq:interface-bc-V3} results in the conditions
% divide by c* coeff then 1-2 again
\begin{equation}
H_3^{out} = \frac{\speedoflightright \mu_R H_3^R + \speedoflightleft \mu_L H_3^L - \left(\alphaGeneral^R - \alphaGeneral^L \right)}{\speedoflightright \mu_R + \speedoflightleft \mu_L } \label{eq:interface-bc-H3-TE}
\end{equation}
and
\begin{equation}
\alphaGeneral^{out} = \frac{\speedoflightright \eps_R \alphaGeneral^R + \speedoflightleft \eps_L \alphaGeneral^L - \left( H_3^R - H_3^L \right) }{\speedoflightright \eps_R + \speedoflightleft \eps_L} \label{eq:interface-bc-alpha-TE}
\end{equation}
with
$$ \alphaGeneral = n_1 E_2 - n_2 E_1. $$
An expression for the numerical flux is obtained by substitution of the expressions for $\Jump{H_3}$ and $\Jump{\alphaGeneral}$ resulting from~\eqref{eq:interface-bc-H3-TE} and ~\eqref{eq:interface-bc-alpha-TE} into~\eqref{eq:AnMinuU-expression-TE}
Similarly for the $\TMz$ this results in 
% divide by c* coeff then 1-2 again
\begin{equation}
E_3^{out} = \frac{\speedoflightright \eps_R E_3^R + \speedoflightleft \eps_L E_3^L - \left(\alphaGeneral^R - \alphaGeneral^L \right)}{\speedoflightright \eps_R + \speedoflightleft \eps_L }
\end{equation}
and
\begin{equation}
\alphaGeneral^{out} = \frac{\speedoflightright \mu_R \alphaGeneral^R + \speedoflightleft \mu_L \alphaGeneral^L - \left( E_3^R - E_3^L \right) }{\speedoflightright \mu_R + \speedoflightleft \mu_L}
\end{equation}
with
$$ \alphaGeneral = - n_1 H_2 + n_2 H_1. $$

Again expressions for $\Jump{E_3}$ and $\Jump{\alphaGeneral}$ resulting from~\eqref{eq:interface-bc-H3-TE} and ~\eqref{eq:interface-bc-alpha-TE} are substituted into~\eqref{eq:AnMinuU-expression-TM} to obtain the numerical flux.
% TODO - do I need a jump J? as well



\subsection{Absorbing boundary condition}
Many problems are posed on infinite domains, however this presents computational difficulties. In practice computation is done on a truncated domain with boundary conditions set in such a way as to approximate the infinite domain. This is done by introducing an artificial outer boundary condition which absorbs incident radiation known as an absorbing boundary condition (ABC)\cite[].
% TODO - what about PML
This is achieved by a modified numerical flux which contains outgoing flux only,
\begin{align}
\NumFlux(\Ue,\Uout) = \NormalFluxPositiveEigenvalues(\USoltn) = \AnPlus \USoltn
\end{align}
% TODO - U here shouldn't have element subscript
in which case~\eqref{eq:AnMinusUderivation} becomes
\begin{align}
  \NumFlux(\Ue,\Uout) - \NormalFlux(\Ue) = \AnPlus \Uout - \NormalFlux(\Ue) = - \AnMinus \Ue ,
\end{align}
or equivalently
$$
\AnMinus \JumpU = - \AnMinus \USoltn,
$$
which corresponds to a first order approximation of the \SilverMuller condition.
% TODO - This can be used without need for a PML, to dissipate waves as they get to boundary.
In practice this is often used in conjunction with a coarsening of the mesh around the truncated boundary to further dissipate outgoing waves\cite{Hall2004140}.

\subsection{PML}
....possibly haven't actually used this yet...!


\section{Meshing}

high-order curvilinear meshe for isoparametric FEM -> solid mechanics analogy -> \cite{DGPaper:43}
meshes for NEFEM, technique proposed in -> \cite{DGPaper:44}

\section{Errors and convergence}
MENTION IN RESULTS
\begin{itemize}
  \item expected rates of convergence for time-domain (interpolation error) and freq domain (dispersion error)
\end{itemize}

%%% Local Variables:
%%% mode: latex
%%% TeX-master: "../Thesis"
%%% End: