\chapter{Discontinuous Galerkin Method for Maxwells Equations} % Write in your own chapter title
\label{Chapter3}
\lhead{Chapter 3. \emph{Discontinuous Galerkin}} % Write in your own chapter title to set the page header

The Discontinuous Galerkin method was first introduced to solve the neutron transport problem by Reed and Hill \cite{reed1973triangularmesh} in 1973 to solve the neutron transport problem. [ Motivation/State of the art - here on in the introduction?]
%** Some background/literature review here Cockburn, Shui for solving hyperbolic equations etc ***
% PhDJesusAlvarez good for motivation - not so much for the method

As seen in Chapter~\ref{PhysicalProblemChapter}, given a suitable choice of
initial conditions the evolution of the system in time can be described by
Maxwell's curl equations in strong
form,~\eqref{eq:maxwell-curl-equations-conservation-form}. Consider that the problem is defined on an open bounded physical domain, $\physicalDomain \in \Real^{\nsd}$, with a boundary $\physicalDomainBoundary$,
% TODO - should I specify that its a PEC boundary (Ruben does)
which is partitioned in an unstructured mesh of $\nel$ nonoverlapping, body-conforming simplices, $\elemIndexed$, such that
$$
\bardomain = \mathop{\bigcup}_{\elemindex=1}^{\nel} \barelemindexed
$$
such that $\elemIndexed \cap \anotherelem = \emptyset $ for $\elemindex \neq \anotherelemindex$, as shown in~\ref{fig:meshPartitionFigure}.
\begin{figure}
  \centering
  
  \caption{Partition of a two dimensional domain $\physicalDomain$ in $\nel$ disjoin triangular element}
  \label{fig:meshPartitionFigure}.
\end{figure}
% TODO - no approx here...!? Understand why this is...

% discuss the discretisation more - duplication of nodes?
As is usual in a DG context, solutions to the strong form problem,~\eqref{eq:maxwell-curl-equations-conservation-form}, over a generic
element $\genericElement$, are sought in the approximation space $\left(
  \E,\H,\J \right) \in \approximationSpaceTotal \left( \left[ 0,
    \periodOfApproximation \right], \approximationSpaceSpatial \right)$ where
$ \approximationSpaceSpatial = \HcurlZero \times \Hcurl \times \LTwoV $
% TODO - check WHICH J I'm using here
with
\begin{align*}
\Hcurl = \{ \bm{v} \in \LTwoV  \; | \; \nabla \times \bm{v} \in \LTwoV \},  \\
\HcurlZero = \{ \bm{v} \in \Hcurl  \; | \; \bm{v} \times \mathbf{n}  = \0  \} 
\end{align*}
and $\LTwoV = [ \LTwo ]^{\nsd}$. [$\mathbf{n}$ not defined]
Following the method of weighted residuals for a single element the strong form,~\eqref{eq:maxwell-curl-equations-conservation-form}, the strong form 
is multiplied by a vector of test functions $\TF \in \approximationSpaceSpatial
$ and integrated over the generic element $\elemIndexed$, resulting in

\begin{equation*}
\int_{\genericElement} \TF \cdot \uet \delem  + \int_{\genericElement} \TF \cdot \dpart{\Flux_{k}(\Ue)}{\xk}= \int_{\genericElement} \TF \cdot \maxwellSource(\Ue) \, \delem,
\end{equation*}
where $\Ue$ denotes the restriction of $\USoltn$ to the element $\genericElement$. Integration by parts then results in
% TODO - am I technically using Einstein notation here...!????
\begin{equation}
\int_{\genericElement} \TF \cdot \uet \delem  - \int_{\genericElement} \dpart{\TF}{\xk} \cdot
\Flux_{k}(\Ue) \delem + \int_{\elemtracegenericelement} \TF \cdot \NormalFlux(\Ue) \delemtrace
= \int_{\genericElement} \TF \cdot \maxwellSource(\Ue) \delem,
\label{eq:weak-form-with-physical-flux}
\end{equation}
where $\outnormalvector$ is outward unit normal vector to the boundary $\elemtracegenericelement$ of $\genericElement$, with directional cosines $\outnormalcoeffk$, and $\mathbf{F_n}$ is the outward normal physical flux, given by $ \mathbf{F_n}(\mathbf{U}) = \outnormalcoeffk \mathbf{F}_k(\mathbf{U}) $.

Since the weak form stated in~\eqref{eq:weak-form-with-physical-flux} is specified on the element $\genericElement$, this does not constitute a scheme suitable for solving the global problem. As is usual in a DG context, continuity of the solution between elements is weakly enforcing by replacing the physical normal flux, $\NormalFlux(\Ue)$, with a consistent numerical flux, $\NumFlux(\Ue,\Uout)$. This numerical flux depends on both the trace of the solution on the element $\genericElement$ and the trace of the solution on the neighbouring element, $\mathbf{U}^{out}$.
% TODO: What is the definition of Uout in mathematical language ??
For a linear hyperbolic system a natural choice is to employ a flux splitting technique~\cite{DoneaHuerta:05}, corresponding to an upwind approximation\cite{CockburnCEM:05}.
% TODO: this is a lot like Rubens/Obays paper, rewrite it. QUOTE: A natural choice, for the linear hyperbolic system of interest here, is to employ a flux splitting technique ~\cite{donea2003finite}, which corresponds to an upwind approximation \cite{chen2005high}
The physical normal flux is written as
$$
\NormalFlux(\USoltn) = \An \USoltn,
$$
where $\An = \outnormalcoeffk \Ak$, and decomposed into incoming (superscript -) flux and outgoing flux (superscript +) as
% VIVA: An is known as the jacobian matrix, and apparently is also \An = \frac{ \partial \NumFlux }{ \partial \U }
\begin{align}
\NormalFlux(\USoltn) = \NormalFluxPositiveEigenvalues(\USoltn) + \NormalFluxNegativeEigenvalues(\USoltn),
\label{eq:phys-flux-splitting}
\end{align}
where $ \NormalFluxPositiveEigenvalues = \AnPlus \USoltn$, $\NormalFluxNegativeEigenvalues = \AnMinus \USoltn, $ and the matrices $\AnMinus$ and $\AnPlus$ denote respectively the matrices of the positive and negative eigenvalues of $\An$, which are obtained, after diagonalisation of $\An = \AnDiagP \AnDiag \AnDiagPInv$, as
\begin{align*}
\AnPlus &= \left( \An + \AnMod \right) / 2   \\
\AnMinus &= \left( \An - \AnMod \right) / 2
\end{align*}
where $\AnMod$ is the absolute value of $\An$, given by $\AnMod \equiv \AnDiagP \AnDiagStar \AnDiagPInv$ and $\AnDiagStar$ is a diagonal matrix which contains the absolute value of the eigenvalues of $\An$.

The choice of an upwind flux splitting approach results in
\begin{align}
\NumFlux(\USoltn,\Uout) = \NormalFluxPositiveEigenvalues(\USoltn) + \NormalFluxNegativeEigenvalues(\Uout),
\label{eq:num-flux-splitting}
\end{align}
Substitution of~\eqref{eq:num-flux-splitting} into~\eqref{eq:weak-form-with-physical-flux} and integration by parts leads to
\begin{align*}
\int_{\genericElement} \TF \cdot \uet \delem  + \int_{\genericElement} \TF \cdot \dpart{\Flux_{k}(\Ue)}{\xk} \delem + \int_{\genericElement} \TF \cdot \left[ \NumFlux(\Ue,\Uout) - \NormalFlux(\Ue) \right] \delemtrace \\
= \int_{\genericElement} \TF  \cdot \maxwellSource(\Ue) \delem \label{eq:weak-form-upwind-splitting-fluxes}
.
% the second term changes from + to - from prev weak form
\end{align*}
The term in square brackets can be expressed as
\begin{equation}
  \NumFlux(\Ue,\Uout) - \NormalFlux(\Ue) = \AnMinus \Uout - \AnMinus \USoltn = \AnMinusU, \label{eq:AnMinusUderivation}
\end{equation}
where the jump operator, defined as $\JumpU = \Uout - \USoltn$, denotes the jump in the solution across $\elemtracegenericelement$. By substitution of~\label{eq:AnMinusUderivation}, the weak form of~\eqref{eq:maxwell-curl-equations-conservation-form} is thus stated as:
find $\left( \E,\H,\J \right) \in \approximationSpaceTotal \left( \left[ 0, \periodOfApproximation \right], \approximationSpaceSpatial \right)$, such that
\begin{equation}
\int_{\genericElement} \TF \cdot \uet \delem  + \int_{\genericElement} \TF \cdot \dpart{\Flux_{k}(\Ue)}{\xk} \delem + \int_{\elemtracegenericelement} \TF \cdot \AnMinusU \delemtrace = \int_{\genericElement} \TF  \cdot \maxwellSource(\Ue) \delem,
\label{eq:weak-form-final}
\end{equation}
for all $\TF \in \approximationSpaceSpatial$.
% TODO: upwind flux: effect for wave domination problems - flow of information, upwind flux

\section{Internal Element Boundaries}
We consider the boundary conditions at an internal boundary between elements.
From~\ref{sec:conservation-form}, this expression the normal Jacobian matrix can
be written as
$$
  \An =
  \begin{pmatrix}
 & \zerom  & \mu^{-1} \RTotNorm & \zerom \\
 & - \eps^{-1} \RTotNorm & \zerom & \zerom \\
 & \zerom & \zerom & \zerom 
 & \end{pmatrix} 
$$
where
$$
  \RTotNorm =
  \begin{pmatrix}
 & 0 & n_3 & -n_2 \\
 & -n_3 & 0 & n_1 \\
& n_2 & -n_1 & 0 
 & \end{pmatrix} .
$$
The modulus of $\An$ is then given by
\begin{align*}
\AnMod = \speedoflight
\begin{pmatrix}
  \modAnSubMatrix & \zerom & \zerom \\
  \zerom  & \modAnSubMatrix & \zerom \\
   \zerom & \zerom & \zerom 
\end{pmatrix}
\end{align*}
where $\speedoflight = \left( \eps \mu  \right)^{-\frac{1}{2}}$ is the speed of light in the medium and
\begin{align*}
  \modAnSubMatrix = 
\begin{pmatrix}
\outnormalcoeff_2^2 + \outnormalcoeff_3^2 &      -\outnormalcoeff_1 \outnormalcoeff_2 &      -\outnormalcoeff_1 \outnormalcoeff_3 \\
-\outnormalcoeff_1 \outnormalcoeff_2 & \outnormalcoeff_1^2 + \outnormalcoeff_3^2 &      -\outnormalcoeff_2 \outnormalcoeff_3 \\
-\outnormalcoeff_1 \outnormalcoeff_3 &      -\outnormalcoeff_2 \outnormalcoeff_3 & \outnormalcoeff_1^2 + \outnormalcoeff_2^2 \\
\end{pmatrix} ,
\end{align*}
where the identity $\sqrt{\sum_{\outnormalcoeffcomp} \outnormalcoeffk^2} = 1$, for the unit vector $\outnormalvector$, has been used. We therefore write
\begin{align*}
\AnMinus = \speedoflight
\begin{pmatrix}
  -\modAnSubMatrix & \RTotNorm & \zerom \\
  -\RTotNorm  & -\modAnSubMatrix & \zerom \\
   \zerom & \zerom & \zerom 
\end{pmatrix} .
\end{align*}

Noting that $\RTotNorm \anyVector = \outnormalvector \times \anyVector$ and $\modAnSubMatrix \anyVector = \outnormalvector \times \left(  \outnormalvector
  \times \anyVector \right)$,for any vector $\anyVector$, results in the expression
\begin{align}
\AnMinusU = \frac{1}{2}
\begin{pmatrix}
  -\nvect \times \left( \JumpH + \sqrt{\frac{\eps}{\mu}} \nvect \times \JumpE \right) \\
   \nvect \times \left( \JumpE + \sqrt{\frac{\mu}{\eps}} \nvect \times \JumpH \right) \\
  \zerov
\end{pmatrix} .
  \label{eq:AnMinuU-expression-3D}
\end{align}
% Should I write out A_n for the TEz and TMz modes?
Similarly, in two dimensions, the normal Jacobian matrix is written as
$$
  \An =
  \begin{pmatrix}
 & 0  & 0  & - \outnormalcoeff_2 \mu^{-1} & 0 & 0 \\
 & 0  & 0  & \outnormalcoeff_1 \mu^{-1} & 0 & 0 \\
 & - \outnormalcoeff_2 \eps^{-1}  & \outnormalcoeff_1 \eps^{-1}  & 0 & 0 & 0 \\
 & 0  & 0  & 0  & 0  & 0 \\
 & 0  & 0  & 0 & 0 & 0
 \end{pmatrix} 
$$
for the $\TEz$ mode and
$$
  \An =
  \begin{pmatrix}
 & 0  & 0  & \outnormalcoeff_2 \eps^{-1} & 0 \\
 & 0  & 0  & - \outnormalcoeff_1 \eps^{-1} & 0 \\
 & \outnormalcoeff_2 \mu^{-1}  & - \outnormalcoeff_1 \mu^{-1}  & 0 & 0 \\
 & 0  & 0  & 0  & 0
 \end{pmatrix} 
$$
for the $\TMz$ mode, which following single value decomposition, results in
$$
  \AnMod =
  \speedoflight
  \begin{pmatrix}
 & \outnormalcoeff_2^2  &  - \outnormalcoeff_1 \outnormalcoeff_2 & 0 & 0 & 0 \\
 & - \outnormalcoeff_1 \outnormalcoeff_2 & \outnormalcoeff_1^2 & 0 & 0 & 0 \\
 & 0  & 0  & 1  & 0  & 0 \\
 & 0  & 0  & 0 & 0 & 0 \\
 & 0  & 0  & 0 & 0 & 0
 \end{pmatrix} 
$$
for the $\TEz$ mode and
$$
  \AnMod =
  \speedoflight
  \begin{pmatrix}
 & \outnormalcoeff_2^2  &  - \outnormalcoeff_1 \outnormalcoeff_2 & 0 & 0  \\
 & - \outnormalcoeff_1 \outnormalcoeff_2 & \outnormalcoeff_1^2 & 0 & 0  \\
 & 0  & 0  & 1  & 0   \\
 & 0  & 0  & 0 & 0  
  \end{pmatrix} 
$$
for the $\TMz$ mode. [ I need to check these]
% TODO - Taken from Mar - I need to compute and check these!]
The the boundary term in~\eqref{eq:weak-form-final} in two dimensions is thus
given as
\begin{align}
\AnMinusU =
  \frac{1}{2}
  \left[
    \Jump{H_3} - \sqrt{\frac{\eps}{\mu}} \Jump{\alphaGeneral}
  \right]
\begin{pmatrix}
   -\outnormalcoeff_2 \\
   \outnormalcoeff_1 \\
   - \sqrt{ \frac{\mu}{\eps} } \\
   0  \\
   0 
\end{pmatrix} .
  \label{eq:AnMinuU-expression-TE}
\end{align}
with $ \alphaGeneral = n_1 E_2 - n_2 E_1. $ for the $\TEz$ mode and
\begin{align}
\AnMinusU =
  \frac{1}{2}
  \left[
    \Jump{E_3} - \sqrt{\frac{\mu}{\eps}} \Jump{\alphaGeneral}
  \right]
\begin{pmatrix}
   \outnormalcoeff_2 \\
   -\outnormalcoeff_1 \\
   -\sqrt{ \frac{\eps}{\mu} } \\
   0 
\end{pmatrix} . \label{eq:AnMinuU-expression-TM}
\end{align}
with $ \alphaGeneral = - n_1 H_2 + n_2 H_1 $ for the $\TMz$ mode.
% VIVA: calculated with matlab script:
% modAn = sqrtm(An*An') and knowing that
% make sure I can do sqrtm by hand
\section{Spatial Discretisation}
This section details two approaches to spatial discretisation of the weak formulation~\ref{eq:weak-form-final}, namely the traditional isoparametric finite element formulation and the recently proposed NURBS-enhanced finite element method (NEFEM).

\section{Isoparametric Finite Element Method}
\label{sec:isoparametric-elements}
A nodal interpolation of the solution $\USoltn$ is defined in a reference element $\refelem$, with local coordinates $\refElemCoords$, as
\begin{equation}
\USoltn(\xiV) \simeq \Uh(\xiV) = \sumNen
  \UVect_{j}(\t)
  \SF_{j} (\xiV)
  ,
\label{eq:nodal-basis-defn}
\end{equation}
% VIVA: the second U_j is the vector of coefficients
where $N_{j}$ are polynomial, Lagrangian shape functions of order $p$ in $\refElemCoords$, $\UVect_j(t) = \USoltn(\nodalCoordinatesOfElement,\t)$ are the time-dependent values of the solution at the nodal point $\nodalCoordinatesOfElement$, and $\nen$ are the number of nodes of the element $\meshelem$.

An isoparametric transformation, $\IsoMapping$, is introduced between the reference element, $\refelem$ and a generic mesh element $\meshelem$, namely
\begin{equation} \label{eq:isoparametricMapping}
\IsoMapping(\xiV) = \sumNen \xCoord_j N_j(\xiV)
\end{equation}
where $\nodalCoordinatesOfElement$ are the nodal coordinates of the element $\meshelem$. Note that for an element with a face or edge on the boundary of the computational domain, the boundary of the element $\meshelem$ will be a polynomial approximation of the real boundary\cite{ComparisonCurvedFEs}.

 Following the Galerkin method, the vector of test functions, $\TF$, is chosen from the same basis as the shape functions. Substitution of~\eqref{eq:nodal-basis-defn} into~\eqref{eq:weak-form-final}, and selecting the space of test functions to be the same as the space spanned by the approximation functions, results in the system
 \begin{equation}
\discretisedWeakForm{\nfn}
\label{eq:discretised-weak-form-fem}
\end{equation}
% TODO - what is the ij subscript, and also why is the An taken out of the face matrix if it isn't constant over the face? Also why is M_{ij} capital below (this is just one matrix entry, right?)
% NOTE - the second term actually also has an implicit sum over k in (Einstein notation)
of $\nen$ ordinary differential equations, for each node $i$, where $\UVect_j$ is a vector of the solution coefficients at the $j$th node, $\MassMatrix$ is the elemental mass matrix, $\IdentityMatrix$ is the identity matrix, $\ConvectionMatrix$ is the convection matrix in the direction $x^{k}$, $\MassMatrixFace$ is the face mass matrix and $\nfn$ denotes the number of face nodes. Note that a choice of Lagrangian, nodal shape functions results in a block diagonal elemental mass matrix. Additionally, an isoparametric mapping results in the restriction of the index $\faceindex$ to face nodes only, since other terms are zero. These matrices are defined by
\begin{equation}
\MassMatrix = \int_{\meshelem} \SF_i \SF_j \delem, \quad
\ConvectionMatrix = \int_{\meshelem} \SF_i \dpart{\SF_j}{\xk}\delem \quad \text{and} \quad
\MassMatrixFace = \int_{\meshelemtrace} \SF_i \SF_j \delemtrace .
\end{equation}
% TODO - why is Ruben as a captial here? And why is M_{ij} captial...none of this make sense...!
% TODO - why dont I change the integration element here?
Using the isoparametric mapping~\eqref{eq:isoparametric-mapping}, the integrals over $\meshelem$, are evaluated on the reference element, $\refelem$, as
\begin{align*}
\MassMatrix = \int_{\refelem} \SF_i \SF_j |\Jacobian| \delem \quad \text{and} \quad
\ConvectionMatrix = 
                               \sum_{l=1}^{\nsd}
                                \int_{\refelem} \SF_i
                               \left( 
                                   \Jacobian^{-1}
                               \right)_{kl}
                               \dpart{\SF_j}{\xi_{l}}
                             |\Jacobian|
                             \delem
\end{align*}
where $\Jacobian = \frac{\partial \phi}{ \partial \xi}$, is the Jacobian of the mapping $\IsoMapping$. Similarly, the face mass matrix is evaluated as
\begin{align*}
\MassMatrixFace &= \int_{\refelemtrace} \SF_i \SF_j || \JacobianFace || \delemtrace
\end{align*}
% TODO - double lines here around HJacobian Face A) insert them properly B) what are they??
where $\JacobianFace$ is the Jacobian of the restriction of the isoparametric mapping to the element face.

% Wikipedia: 'In geometry, an affine transformation, affine map[1] or an affinity (from the Latin, affinis, "connected with") is a function between affine spaces which preserves points, straight lines and planes.'
For elements for which the transformation $\IsoMapping$ is affine, both $\Jacobian$ and $\JacobianFace$ are scalar constants over the element, and therefore the elemental matrices simplify to
\begin{align*}
\MassMatrix &= |\Jacobian| \MassMatrixAffine \\
\ConvectionMatrix &= |\Jacobian|
                               \sum_{l=1}^{\nsd}
                               \left( 
                               \Jacobian^{-1}\
                                \right)_{kl}
                               \ConvectionMatrixAffine \\
\MassMatrixFace &= || \JacobianFace || \MassMatrixFaceAffine
,
\end{align*}
% TODO - double lines here around Jacobian Face
with the reference elemental matrices
\begin{align*}
\MassMatrixAffine &= \int_{\refelem} \SF_i \SF_j \drefelem \\
\ConvectionMatrixAffine &= \int_{\refelem} \SF_i
                             \left(
                               \dpart{\SF_j}{\xi_{l}}
                             \right)
                             \drefelem \\
\MassMatrixFaceAffine &= \int_{\refelemtrace} \SF_i \SF_j \delemtrace
,
\end{align*}
which depend only on the shape functions and the geometry of $\refelem$. The reference elemental matrices $\MassMatrixAffine$, $\ConvectionMatrixAffine$ and $\MassMatrixFaceAffine$ can be computed \textit{a priori} in the reference element, inverted if necessary and stored. Computation of $\ConvectionMatrix$, $\MassMatrixFace$ and the inverse of $\MassMatrix$ for each element therefore reduces to a multiplication of the reference element matrix by a scalar. Whenever possible meshes are constructed in order to maximise the number of elements for which an affine mapping can be established between $\refelem$ and $\meshelem$. For curved elements, since the $|\Jacobian|$ and $|\JacobianFace|$ are not constant, it is not possible to precompute reference element matrices. In practical applications however, meshes are constructed where with an extremely low number of curved elements. In these cases the elemental matrices $\MassMatrix$, $\ConvectionMatrix$ and $\MassMatrixFace$ are computed and stored once per element.
The resulting implementation, in which numerical integration is performed \textit{a priori}, is known as the \textit{quadrature-free} implementation\cite{AJ-Atkins-98}, and can reduce the cost of a high-order DG method by a factor of 100\cite{HybridMeshesCEM}.
% VIVA : no loop on gauss points now!!
% TODO / VIVA : what about factoring \AnMinu out of the integrand because n is constant over a planar face????
% TODO - also cite Rubens paper here: The use of hybrid meshes to improve the efficiency of a discontinuous Galerkin
% TODO - what is this whole thing with triangular/tetrahedral meshes -> always constant jacobian unless curved, Quads -> not always constant jacobian unless curved, should I elabourate?


% TODO - 'isoparametric mapping given by the coords of the vertices of Omega
%%%  In order to perform the integration over the element interior an isoparametric
%%% mapping, $\IsoMapping$, is introduced from a reference element $\refelem$ to the
%%% physical element $\genericElement$. The integrals in~\eqref{}, once transformed to the
%%% reference element, become
% TODO - ruben has lk not lk on Jacobian matrix
% VIVA - make sure I know what this definition of J means
% TODO - easier way of saying this
% TODO - comment (remark) on affine elements, and how I can remove the jcaobian etc etc




\section{NURBS-enhanced finite element method (NEFEM)}
% TODO - polynomial approximation of the boundary stuff here (take it from the paper maybe)
In the NEFEM approach, the approximation is defined directly in the physical space in Cartesian coordinates, as
\begin{align}
\USoltn_e(\xbf,\t) \simeq \USoltnApprox = \sum_{j=1}^{\nen} \SF_{j} (\xbf) \UVect_{j}(\t) ,
\label{eq:nodal-basis-defn}
\end{align}
% TODO - I haven't defined U_h here...!
% TODO - duplicate definition of xbf? maybe?
% VIVA: the second U_j is the vector of coefficients
% TODO / VIVA - do these have to be Lagrangian at this stage? Can they be modal?
where $N_{j}$ are polynomial, Lagrangian shape functions of order $p$, defined in $\xbf$, and $\nen$ are the number of nodes of the element $\meshelem$. Substitution of \eqref{eq:nodal-basis-defn} into the weak form,~\eqref{eq:weak-form-final}, and again selecting the space of test functions to be the same as the space spanned by the approximation functions, results in the system
% TODO...
Note that for an element with a face or edge on the boundary of the computational domain, the boundary of the element $\meshelem$ will be a polynomial approximation of the real boundary\cite{}.
% TODO - COPY RUBEN for citation
 Following the Galerkin method, the vector of test functions, $\TF$, is chosen from the same basis as the shape
functions.
Substitution of~\eqref{eq:nodal-basis-defn} into~\eqref{eq:weak-form-final}, and selecting the space of test functions to be the same as the space spanned by the approximation functions, results in the system
% NOTE - the second term actually also has an implicit sum over k in (Einstein notation)
\begin{equation}
\discretisedWeakForm{\nen}
\label{eq:discretised-weak-form-nefem}
\end{equation}
of ordinary differential equations for each node $i$ of the element $\genericElement$. Due to the definition of the approximation in physical space, the summation over the index $\faceindex$ in~\eqref{eq:discretised-weak-form-nefem} is no longer restricted to face nodes as it is in~\eqref{eq:discretised-weak-form-fem}. The elemental matrices are defined as
\begin{align*}
\MassMatrix &= \int_{\genericElement} \SF_i \SF_j \delem \\
\ConvectionMatrix &= 
               \int_{\genericElement}
               \SF_i
               \dpart{\SF_j}{\xi_{l}}
               \delem \\
\MassMatrixFace &= \int_{\elemtracegenericelement} \SF_i \SF_j \delemtrace
\end{align*}
% TODO - what is the ij subscript, and also why is the An taken out of the face matrix if it isn't constant over the face?
A detailed discussion and comparison of different strategies for computing integrals whilst accounting for the exact boundary representation are presented in\cite{NEFEM-Integration}. In this work it was concluded that composite Gaussian quadratures are the most efficient option. In addition, for the same mesh and degree of approximation NEFEM only requires one or two more integration points to obtain its maximum accuracy, which is significantly higher than the obtained by standard FEM with curved elements. 
% TODO - why is Ruben as a captial here? And why is M_{ij} captial...none of this make sense...!
% TODO - why dont I change the integration element here?
% RUBEN - says 'the element face f'

%% \begin{align*}
%% \sum_{j=1}^{\nen} \left[
%%   % term 1
%%   \left(
%%     \int_{\genericElement} \SF_i \SF_j \delem
%%   \right)
%%   \dodet{\UVect_j}
%% +
%%   % term 2
%%   \left(
%%     \int_{\genericElement} \SF_i \dpart{\SF_j}{x^k} \delem
%%   \right)
%%   \Ak \UVect_j
%% +
%%   % term 3
%%   \left(
%%   \int_{\elemtrace} \SF_i \SF_j \delemtrace 
%%   \right)
%%   \AnMinus \JumpUCoeffVectUnknownsWithIndex{j}
%%   % term 4
%% -
%%   \left(
%%   \int_{\genericElement} \SF_i \SF_j \delem
%%   \right)
%%   \Asource 
%%   \right]  = 0
%% %
%%   \; \; \forall \; i \; \in \; 1...\nen,
%% \end{align*}
% NOTE - again, second has implicit sum
% NOTE - in Ruben/Oubay paper MI is used instead of M
% TODO - A_s has not been defined....! Also am I missing a U to multiply A_s?

% TODO - this happens because by definition other face nodes are zero on all nodes
% except the ones not on a face...

% TODO: what is all this crap about M_{ij} and how does it correspond to residual?
% TODO: how should I write residual vector now?

\section{Time Integration}
The semi-discrete system, ..., can be written in the form
$$
\MassMatrix \dodet{\UVect} + \Residual = 0 ,
$$
where the residual vector, $\Residual$, contains contributions arising from the divergence, boundary and source terms from either \eqref{eq:discretised-weak-form-fem} or \eqref{eq:discretised-weak-form-nefem}. Multiplying by the inverse of the mass matrix results in
$$
\dodet{\UVect} = \ResidualInv(\UVect,\t) ,
$$
where $\ResidualInv = \MassMatrixInv \Residual(\UVect,\t)$.
This system of ordinary differential equations is integrated in time using an explicit fourth order Runge-Kutta method, as given by
$$
\UVect^{n+1} = \UVect^{n} + \frac{\Delta t}{6}
\left( 
  \ResidualInvContrib{1}
+  \ResidualInvContrib{2}
+  \ResidualInvContrib{3}
+  \ResidualInvContrib{4}
\right)
$$
where
\begin{align}
\ResidualInvContrib{1} &= \ResidualInv(\UVect^n,\t^n) \\
\ResidualInvContrib{2} &= \ResidualInv
  \left( 
    \UVect^n + \ResidualInvContrib{1} \frac{\Delta \t}{2},\t^{n} + \frac{\Delta \t}{2}
 \right) \\
\ResidualInvContrib{3} &= \ResidualInv
  \left( 
    \UVect^n + \ResidualInvContrib{2} \frac{\Delta \t}{2},\t^{n} + \frac{\Delta \t}{2}
 \right) \\
\ResidualInvContrib{4} &= \ResidualInv(\UVect^n + \ResidualInvContrib{3} \Delta \t,\t^{n} + \Delta \t) \\
\label{eq:3}
\end{align}
The method is known to be conditionally stable, with a stability condition given by
$$
\Delta t \le  \frac{ \CFLCondition \minElemSize }{ \speedoflightCFL p^2}
$$
where $\minElemSize$ is the minimum element size, $p$ is the degree of polynomial approximation and $\CFLCondition$ is a constant for a given $p$.
% different constant from speed of light?
In practice the time step had been taken to be sufficiently small that the numerical error is dominated by error due to spatial discretisation. Implicit time integration schemes, which are unconditionally stable, steps may be employed, however this is undesirable since larger timesteps limit the highest frequencies which are resolvable in the signal processing applications, as shown in~\autoref{Ch:SignalAnalysis}.
% TODO - justify high order time integration....!

% TODO: R/O Quote: 'Triangles and quadrilaterals are employed to provide a consistent
%% discretisation of the spatial solution domain, X, for two
%% dimensional problems. In three dimensions, consistent meshes
%% consisting of tetrahedra, hexahedra, prisms and pyramids are used.
%% Apart from the pyramid, which requires special attention, optimal
%% nodal finite elements of arbitrary order are readily defined for all
%% these shapes. For the pyramid, a recently proposed approximation
%% space [28] is adopted. This space is well suited for both continuous
%% and discontinuous approximations and is optimal, i.e. the a priori
%% error estimate is Oðhpþ1 Þ in the L2ðXÞ norm, where p denotes the
%% order of the approximation. The approximation spaces that are employed
%% are summarised in Table 1' <----


\subsection{Quadrature and numerical integration}
For numerical integration of the mass matrix, the optimal interpolation points proposed in \cite{Babuska-CB:95} are utilised, together with the technique proposed in \cite{JOCP-Hesthaven-02} for constructing high-order polynomial basis functions and their derivatives. The quadratures employed to integrate the terms of the weak form correspond to the integration rules recently proposed in \cite{Witherden20151232}. The number of integration points is selected so that exact integreation is achieved for polynomials of order less than or equal to $2p + 1$.
%TODO - Another thing to mention now is about the choices of quadrature (i.e. summations over gauss points)
%%  R/O Quote:
%%   For quadrilateral and hexahedral elements, quadrature
%%  based on the tensor product of well known one–dimensional Gauss–Legendre
%%  rules is readily implemented for any order of approximation. Note, however,
%%  that other quadrature formulae, with fewer integration points, exist [31, 32].
%%  For triangles, specific quadrature rules, such as the symmetric quadrature
%%  proposed in [33, 34], are used. Analogously, efficient specific quadrature
%%  rules are used for tetrahedra, prisms and pyramids [34, 35].


%  * isoparametric only capture *roughly* the geometry....
%  * curved elements used both for higher accuracy or for curved boundaries...
%     -> can use planar high order elements...dont confuse the two
%  * p-extension of FEM reference (Ruben has Szabo and Babuska,1991)
%  
%  planar mesh -> 'poly order of the approximation is increased' -> to get to the desired error
%  
%  This is great...but in some cases...
%  but... geometric accuracy 'deteriorates the solution'...
%  isoparametric causes this...i.e the nodes are correct but between them is interpolation...
%  LOADS OF REFERENCES HERE (Sevilla stuff)
% Fekette + matrix condition number
%  higher geom accuracy
%  There is a lot of stuff in Rubens thesis about curved elements...what is he on about??
% VIVA: parallel is also cool for storage...possibly...


% TODO: Fekette nodal distribution: Show some figures of Fekette distributions in reference element?
%% R/O Quote: In two dimensions, a Fekete nodal distribution is adopted for the triangle
%% [29] and a tensor product of one dimensiona In three dimensions, the nodal distributions proposed in [30] for the tetrahedron and
%% in [28] for the pyramid are used. A tensor product of one dimensional Fekete
%% nodal distributions is used for the hexahedron and a tensor product of triangular
%% and one dimensional Fekete nodal distributions is used for the prism.l Fekete nodal distributions for the quadrilateral.

\subsection{Jump Conditions}
% TODO - references -> LeVeque/Donea + Huerta(2005)
For interfaces which intersect $\partial \Omega$ and material interfaces not all components of $\mathbf{U}^{out}$ are completely determined. For a system of conservation laws the solution at the interface can be determined by applying  Rankine-Hugoniot jump conditions along characteristics in the phase plane, given by
\begin{align}
\Jump{ \mathbf{F}_n } = \lambda_j \Jump{ \mathbf{U} } \label{eq:rankine-hugoniot}
\end{align}
where $\lambda_j$ are the eigenvalues of the jacobian matrix $\mathbf{A}_n$.
For the 3 dimensions these are $ \lambda_{ 1,2 } = - \speedoflightleft $, $
\lambda_{ 3,4 } = \speedoflightright $ and $\lambda_{5..9} = 0 $, where the
$\speedoflightleft$ and $\speedoflightright$ are the velocities of the
electromagnetic wave in media on the left and right side of the interface
respectively. This condition should be satisfied along the phase plane
characteristics, as shown in \ref{fig:phase-plane-characteristics}.

\begin{figure}[h]
  \centering
  
  \caption{Phase plane diagram showing the characteristics for Maxwells' equations}
  \label{fig:phase-plane-characteristics}
\end{figure}

\subsection{Rankine-Hugoniot Jump Conditions in 3D}
For the three dimensional case the normal flux can be written as
\begin{align*}
\NormalFlux =
\begin{pmatrix}
- \outnormalvector \times \H \\
\outnormalvector \times \E
\end{pmatrix}
\end{align*}
Applying the condition~\eqref{eq:rankine-hugoniot} and solving the resulting linear system results in
\begin{align}
  \LStarMat{\mu} \LStarMat{H} - \muL \LMat{\H} = - \frac{1}{\LMat{\speedoflight}} \outnormalvector \times \left( \LStarMat{\E} - \LMat{\E} \right)
  \label{eq:jump-condition-resulting-equation-system-3D-1} \\
  \muR \RMat{\H} - \LStarMat{\mu} \LStarMat{\H} = \frac{1}{\LMat{\speedoflight}} \outnormalvector \times \left( \RMat{\E} - \LStarMat{\E} \right)
  \label{eq:jump-condition-resulting-equation-system-3D-2} \\
\end{align}
% TODO - I got this from Mar thesis, have not verified it
\subsection{Rankine-Hugoniot Jump Conditions in 2D}
For the $\TEz$ and $\TMz$ modes, the non-zero components of the normal physical flux can be
written in the form
$$
\NormalFlux = 
\begin{pmatrix}
  - n_2 \UFieldComp_3 \\
  n_1 \UFieldComp_3 \\
  \alphaGeneral
\end{pmatrix}
$$
where $ \alphaGeneral = n_1 \UFieldComp_2 - n_2 \UFieldComp_1 $, and the vector $\UField$ is given by $
\UField = 
  \begin{pmatrix} 
    E_1 \; E_2 \; H_3
  \end{pmatrix}^T
$ for the $\TEz$ mode, and $
\UField = 
  -
  \begin{pmatrix} 
    H_1 \; H_2 \; E_3
  \end{pmatrix}^T
  $ for the $\TMz$ mode.
% TODO - CHECK OUT THE A_n vectors and check that these F_n come nicely from there...
% TODO - should I have a numbered lambda here? and would that be different for
% TE and TM modes
By applying the Rankine-Hugoniot condition along $\dpart{x}{t} = 0$, with $ \Jump{ \mathbf{F}_n } = 0 $, the normal flux in regions $\LStarSymbol$ and $\RStarSymbol$ are equal, and
\begin{equation*}
  \LStarMat{\UFieldComp_3} = \RStarMat{\UFieldComp_3} \quad \text{and} \quad \LStarMat{\alphaGeneral} = \RStarMat{\alphaGeneral},
\end{equation*}
therefore only $\LStarMat{\Box}$ is used to represent both regions.
Along $\dpart{x}{t} = \speedoflightright$, the Rankine-Hugoniot condition takes the form
$$
\Jump{ \mathbf{F}_n } = \speedoflightright \Jump{ \mathbf{U} }
$$
which results in
\begin{align}
-n_2 \left(\RMat{V_3} - \LStarMat{V_3} \right) &= \micoeffR \left( \RMat{V_1} - \LStarMat{V_1} \right) \label{RHR-sys-1} \\
n_1 \left(\RMat{V_3} - \LStarMat{V_3} \right) &= \micoeffR  \left( \RMat{V_2} - \LStarMat{V_2} \right) \label{RHR-sys-2} \\
\micoeffR \left(\RMat{\alphaGeneral} - \LStarMat{\alphaGeneral} \right) &= \left( \RMat{V_3} - \LStarMat{V_3} \right) \label{RHR-sys-3}
\end{align}
where $\micoeff = \sqrt{\eps/\mu}$ for $\TEz$ mode and $\micoeff =
\sqrt{\mu/\eps}$ for $\TMz$ mode. It can be easily verified that
\eqref{RHR-sys-1} and~\eqref{RHR-sys-2} imply~\eqref{RHR-sys-3}.

Along the $\dpart{x}{t} = - \speedoflightleft$, the Rankine-Hugoniot condition takes the form
$$
\Jump{ \mathbf{F}_n } = - \speedoflightleft \Jump{ \mathbf{U} }
$$
which results in
\begin{align}
-n_2 \left(\LMat{V_3} - \LStarMat{V_3} \right) &= - \micoeffL \left( \LMat{V_1} - \LStarMat{V_1} \right) \label{RHL-sys-1} \\
n_1 \left(\LMat{V_3} - \LStarMat{V_3} \right) &= - \micoeffL  \left( \LMat{V_2} - \LStarMat{V_2} \right) \label{RHL-sys-2} \\
- \micoeffL \left( \LMat{\alphaGeneral} - \LStarMat{\alphaGeneral} \right) &= \left( \LMat{V_3} - \LStarMat{V_3} \right) \label{RHL-sys-3}
\end{align}
As above, \eqref{RHL-sys-1} and \eqref{RHL-sys-2} imply \eqref{RHL-sys-3}.
Therefore, for the $\TEz$ and $\TMz$ modes, the Rankine-Hugoniot constraints are
imposed by solving the linear system given by \eqref{RHL-sys-3} and
\eqref{RHR-sys-3}, resulting in
% divide by c* coeff then 1-2 again
% 1 - 2 and rearrange
\begin{equation}
\LStarMat{\alphaGeneral} = \frac{\RMat{\UFieldComp_3} - \RMat{\UFieldComp_3} - \micoeffR \RMat{\alphaGeneral} - \micoeffL \LMat{\alphaGeneral}}{\micoeffL + \micoeffR}\label{eq:interfce-bc-alphaGeneral}
\end{equation}
and
\begin{equation}
  \LStarMat{\UFieldComp_3} =
  \frac{
    \micoeffinvR \RMat{\UFieldComp_3} + \micoeffinvL \LMat{\UFieldComp_3} - \RMat{\alphaGeneral} + \LMat{\alphaGeneral}
  }{
    \micoeffinvL + \micoeffinvR
    } , \label{eq:interface-bc-V3}
\end{equation}
where $\micoeffinv \equiv \micoeff^{-1}$.

\subsection{Material Interfaces}
The expressions for the jump of the solution at a material interface are given by $\Jump{\E} = \LStarMat{\E} - \LMat{\E}$ and $\Jump{\H} = \LStarMat{\H} - \LMat{\H}$ for the left element and $\Jump{\E} = \LStarMat{\E} - \RMat{\E}$ and $\Jump{\H} = \LStarMat{\H} - \RMat{\H}$ for the right element.
% TODO - do I need a jump J? as well
Solving the linear system given by~\eqref{eq:jump-condition-resulting-equation-system-3D-2} and~\eqref{eq:jump-condition-resulting-equation-system-3D-2} results in
\begin{align}
 \outnormalvector \times \LStarMat{\E} = \outnormalvector \times \frac{
  \left( 
\speedoflightleft \epsL \LMat{\E} - \outnormalvector \times \LMat{\H}
 \right)
  +
  \left( 
\speedoflightright \epsR \RMat{\E} + \outnormalvector \times \RMat{\H}
 \right)
}{
  \speedoflightright \epsR + \speedoflightleft \epsL
} \label{eq:material-interfaces-1} \\
 \outnormalvector \times \LStarMat{\H} = \outnormalvector \times \frac{
  \left( 
\speedoflightleft \epsL \LMat{\H} + \outnormalvector \times \LMat{\E}
 \right)
  +
  \left( 
\speedoflightright \epsR \RMat{\H} - \outnormalvector \times \RMat{\E}
 \right)
}{
  \speedoflightright \epsR + \speedoflightleft \epsL
} \label{eq:material-interfaces-1}
\end{align}
The resulting expression for $\outnormalvector \times \LStarMat{\E}$ and $\outnormalvector \times \LStarMat{\H}$ are substituted into~\eqref{eq:AnMinuU-expression-3D} in order obtain an expression for numerical flux on the interface.

For the $\TEz$ mode solving the system of equations given by \eqref{eq:interfce-bc-alphaGeneral} and~\eqref{eq:interface-bc-V3} results in the conditions
% divide by c* coeff then 1-2 again
\begin{equation}
\LStarMat{H_3} = \frac{\speedoflightright \muR \RMat{H_3} + \speedoflightleft \muL \LMat{H_3} - \left(\RMat{\alphaGeneral} - \LMat{\alphaGeneral} \right)}{\speedoflightright \muR + \speedoflightleft \muL } \label{eq:interface-bc-H3-TE}
\end{equation}
and
\begin{equation}
\LStarMat{\alphaGeneral} = \frac{\speedoflightright \epsR \RMat{\alphaGeneral} + \speedoflightleft \epsL \LMat{\alphaGeneral} - \left( \RMat{H_3} - \LMat{H_3} \right) }{\speedoflightright \epsR + \speedoflightleft \epsL} \label{eq:interface-bc-alpha-TE}
\end{equation}
where
$$ \alphaGeneral = n_1 E_2 - n_2 E_1. $$
An expression for the numerical flux is obtained by substitution of the expressions for $\Jump{H_3}$ and $\Jump{\alphaGeneral}$ resulting from~\eqref{eq:interface-bc-H3-TE} and ~\eqref{eq:interface-bc-alpha-TE} into~\eqref{eq:AnMinuU-expression-TE}
Similarly for the $\TMz$ this results in 
% divide by c* coeff then 1-2 again
\begin{equation}
\LStarMat{E_3} = \frac{\speedoflightright \epsR \RMat{E_3} + \speedoflightleft \epsL \LMat{E_3} - \left(\RMat{\alphaGeneral} - \LMat{\alphaGeneral} \right)}{\speedoflightright \epsR + \speedoflightleft \epsL }
\end{equation}
and
\begin{equation}
\LStarMat{\alphaGeneral} = \frac{\speedoflightright \muR \RMat{\alphaGeneral} + \speedoflightleft \muL \LMat{\alphaGeneral} - \left( \RMat{E_3} - \LMat{E_3} \right) }{\speedoflightright \muR + \speedoflightleft \muL}
\end{equation}
where
$$ \alphaGeneral = - n_1 H_2 + n_2 H_1. $$

Again expressions for $\Jump{E_3}$ and $\Jump{\alphaGeneral}$ resulting from~\eqref{eq:interface-bc-H3-TE} and ~\eqref{eq:interface-bc-alpha-TE} are substituted into~\eqref{eq:AnMinuU-expression-TM} to compute the numerical flux term.
% TODO - do I need a jump J? as well
\subsection{PEC}
Along a PEC boundary only the tangential components of the electric field is known, the tangential component of the magnetic field is determined from the Rankine-Hugoniot jump condition. For a PEC boundary the jump in the fields at the interface corresponds to $\JumpU = \SurfMat{\USoltn} - \USoltn$ , where $\SurfMat{\USoltn}$ is the value of the solution on the surface, which corresponds to the $\LStarSymbol$ material. In three dimensions, the tangential component of electric field is known and from~\eqref{eq:material-interfaces-tangentalcondition-H-PEC} and the tangential component of the magnetic field can be from~\eqref{eq:jump-condition-resulting-equation-system-3D-1} 
\begin{align}
  \outnormalvector \times \SurfMat{\H} = \outnormalvector \times \H - \sqrt{\frac{\eps}{\mu}} \outnormalvector \times \left( \outnormalvector \times \SurfMat{\E} - \outnormalvector \times \E \right)
\end{align}

For the $\TEz$ mode, the condition~\eqref{eq:material-interfaces-tangentalcondition-E-PEC} implies in $ \SurfMat{\alphaGeneral} = 0 $, which from~\eqref{RHL-sys-3}, by identifying $\LMat{\USoltn}$ with $\USoltn$, and $\LStarMat{\USoltn}$ with $\SurfMat{\USoltn}$, the expression
\begin{equation*}
\SurfMat{H_3} 
  =  H_3
- \sqrt{\frac{\eps}{\mu}} \Jump{ \alphaGeneral},
\end{equation*}
is obtained. By substitution into~\eqref{eq:AnMinuU-expression-TE} results in the expression,
$$
\AnMinusU =
\alphaGeneral
\sqrt{\frac{\eps}{\mu}}
\begin{pmatrix}
 - n_2 \\
n_1 \\
- \sqrt{\frac{\mu}{\eps} }
\end{pmatrix}
.
$$
% TODO - not verified

Similarly for the $TMz$ mode ~\eqref{eq:material-interfaces-tangentalcondition-E-PEC} implies in $ E_3 = 0$, which from~\eqref{RHL-sys-3}, by identifying $\LMat{\USoltn}$ with $\USoltn$, and $\LStarMat{\USoltn}$ with $\SurfMat{\USoltn}$, an expression for $\SurfMat{\alphaGeneral}$ is obtained as
\begin{equation*}
\SurfMat{\alphaGeneral} 
  =  \alphaGeneral
+ \sqrt{\frac{\eps}{\mu}} \Jump{ E_3}
.
\end{equation*}
By substitution of $\SurfMat{\alphaGeneral}$ into~\eqref{eq:AnMinuU-expression-TE}, the expression
$$
\AnMinusU =
E_3
\begin{pmatrix}
n_2 \\
- n_1 \\
\sqrt{\frac{\mu}{\eps} }
\end{pmatrix}
$$
% TODO - not verified
is obtained, which is then used in computation of the numerical flux.

% TODO - reference equations
\subsection{Absorbing boundary condition}
Many problems are posed on infinite domains, however this presents computational
difficulties. In practice computation is done on a truncated domain with
boundary conditions set in such a way as to approximate the infinite domain.
This is done by introducing an artificial outer boundary condition which absorbs
incident radiation known as an absorbing boundary condition (ABC)\cite{GivoliBook}.
This is achieved by a modified numerical flux which contains outgoing flux only,
\begin{align}
\NumFlux(\Ue,\Uout) = \NormalFluxPositiveEigenvalues(\USoltn) = \AnPlus \USoltn
\end{align}
% TODO - U here shouldn't have element subscript
in which case~\eqref{eq:AnMinusUderivation} becomes
\begin{align}
  \NumFlux(\Ue,\Uout) - \NormalFlux(\Ue) = \AnPlus \Uout - \NormalFlux(\Ue) = - \AnMinus \Ue ,
\end{align}
or equivalently
$$
\AnMinus \JumpU = - \AnMinus \USoltn,
$$
which corresponds to a first order approximation of the \SilverMuller condition.
% TODO - This can be used without need for a PML, to dissipate waves as they get to boundary.
In practice this is often used in conjunction with a coarsening of the mesh around the truncated boundary to further dissipate outgoing waves\cite{Hall2004140}.

\subsection{PML}
....possibly haven't actually used this yet...!?


\section{Meshing}
high-order curvilinear meshe for isoparametric FEM (Roman) \cite{HO-Meshing}
meshes for NEFEM, technique proposed in (Luke) \cite{NEFEMmeshes}

\section{Errors and convergence}
MENTION IN RESULTS
\begin{itemize}
  \item expected rates of convergence for time-domain (interpolation error) and freq domain (dispersion error)
\end{itemize}

%%% Local Variables:
%%% mode: latex
%%% TeX-master: "../Thesis"
%%% End: