% Chapter 1

\chapter{Notes} % Write in your own chapter title
\label{Appendix.2D_Plate_notes}
\lhead{Appendix. \emph{2D Plate Notes}} % Write in your own chapter title to set the page header

Progress [Code Changes]
\begin{itemize}
\item plate 2D, beta=zero, delta function excitations
\item plate 2D, gaussian excitiation (narrow)
\item plate 2D, gaussian excitiation (wide)
\item plate 2D, beta non-zero -> same study
\item plate 2D, FFT vs Harminv (vs..?)
\item Scattering 2D real - benchmarking
\item Study of minimum dt and minimum periodetc (time it takes maybe...?)
\item Scattering 2D complex refractive index - issues (causality)
\item How much faster is harminv for same dataset...?
\end{itemize}

Numerics
\begin{itemize}
\item Maxwells Eqtns
\item DG discretisation
\item Jump Condition
\item ADE and dispersion
\item LITERATURE REVIEW  + include some references here...!
\end{itemize}

Examples to run
\begin{itemize}
\item A fuckload of super-short gaussian initial conditions (20 mins each) - showing the effect of the initial condition on the wave and more specifically on convergence. Show lack of convergence for wider gaussians :)
\item Use optimised gaussian and do the same thing for MULTIPLE gaussians in the initial conditions - conclude with convergence.
\item compare current (beta=0) to previous 2D (i.e. without beta/TE or TM modes)
\end{itemize}

Things to Discuss - 2D plate FFT
\begin{itemize}
\item Compact 2D formulation with Beta = 0 (which reduces to TE/TM mode decoupling)
\item Comparison of FFT and harminv
\item Why 2D-plate doesn't work
\item Obtaining a good spectrum (how)
\item Timestep convergence for a spectrum
\item Period Convergence
\item Blackman window functions (with without window functions)
\item How timestep and period converges - how do spectrums look at a lower convergence
\item Show some BAD spectrums - show examples of BAD components. What happens if I only excite ONE component of my field...does that produce a better result?
\item Wave comparisons - DO WAVE COMPARISONS AT super-high p (e.g. p1->3 etc)
\item Show effect of changing the gaussian profile on the convergence (i.e. do a fuckload of gaussian initial conditions and show convergence)
\item Lorentz Fittings. Improvement in results (careful to not make improvement TOOOO good...! LOL! DO THIS LAST BECAUSE IT ISN'T THAT IMPORTANT)
\item Difference in components obtained (i.e. some are missing the larger frequency waves...!)
\item Cut-off point analysis - i.e. how do I choose a cut-off point, what if I choose the wrong cutoff point? How to identify peaks? Other methods for identifying peaks.
\item Conclusions - how long and how closely sampled, is this good for us? Does problem complexity affect this? There are lots of dependence on dt, period,cut-off points and initial conditions. Lots of factors affecting a good spectrum. We would like to use a more efficient method for finding resonances - but at the same time
\end{itemize}

Things to Discuss - 2D plate Harminv
\begin{itemize}
\item HARMINV - convergence plots + comparison to previous (i.e. improvement in convergence for a bunch of cases)
\item Harminv window functions
\item Harminv comparions of results...! :)
\item Dependence on initial conditions (Gaussian, Delta function, Gaussian support)
\item Dependence on time step
\end{itemize}

Things to Discuss - 2D plate Beta non-zero
\begin{itemize}
\item show some spectrums obtained. Maybe a comparison to analytical. Just give a slight motivation that it works...In theory would be nice to do all the above analysis with beta different from zero - but just for now do something to show that it works. Just show convergence of a wave maybe using the init conditions used above. Maybe use multiple gaussians to get a good convergence? Who knows...?
\end{itemize}

Things to Discuss - Complex
\begin{itemize}
\item 2d-plate with complex (MAYBE?? Probably not)
\item 2D-scattering real [benchmarks] - obtaining RCS, obtaining RCS convergence, limitations etc.
\item 2D-scattering real/complex comparison...why doesn't this work - some EXPLAINATION that doesn't make me look toooo bad (i.e. Kramer-Kronig relationshps which INCLUDE a freq dependence). Try to draw a logical argument between the comparison with real values, the instability and the conclusion of the method being incorrect.
\item 2D-scattering diagnostics - things going wrong, comparion of waves to real. Comparison of RCS to real. DONT TELL THEM THIS WAS A WASTE OF TIME....!
\item what is needed to make complex work (i.e. dispersion) + initial comparison
\item Why 2D-scattering complex doesn't work + Dispersion + blowing up
\item NEED TO DEMONSTRATE I UNDERSTAND DG and NUMERICAL SIDE OF THINGS TOO...this is where they might catch me out. FIND SOME CLEVER LITTLE SENTENCES TO SAY SHOWING A DEEPER UNDERSTANDING.
\item ADE, Dispersion and why other options. Why ADE.
\end{itemize}

Things to Research and Include
\begin{itemize}
\item Quality factor calculations
\item Jump condition - and suitability
\end{itemize}

\section{2D plate}

Future
\begin{itemize}
\item differnt excitiation types - limitations of excitiations
\end{itemize}

\subsection{Spectral Analysis (Beta equal zero)}
Content
\begin{itemize}
\item comparison of method(s) for different problems
\item Attempt to quantify what affects the error frequencies obtained.
\item Effects of Period
\item Effects of Timestep
\item Dependence on initial conditions
\item how to improve spectrum...?
\item need a sentance like "FFT runs are 4 times shorter (based on a paramtrised signal with changing parameters)"
\end{itemize}

\begin{itemize}
\item convergence plots for timestep and period etc etc
\item convergence plots for wave shapes - depending on initial conditions (for given point excitation)
\item Effects of Period
\item Effects of Timestep
\item Dependence on initial conditions
\item how to improve spectrum...?
\item need a sentance like "FFT runs are 4 times shorter (based on a paramtrised signal with changing parameters)"
\end{itemize}

\subsection{2D plate with/without z-dependence}
\begin{itemize}
\item which equations are used
\item convergence graphs
\item comparison of running for different T
\end{itemize}
\section{2D scattering}
\section{2D complex parameters}