\chapter{Discontinuous Galerkin Method for Maxwells Equations} % Write in your own chapter title
\label{Chapter3}
\lhead{Chapter 3. \emph{Discontinuous Galerkin}} % Write in your own chapter title to set the page header

\subsection{Formulation}

The Discontinuous Galerkin method was first introduced to solve the neutron transport problem by Reed and Hill \cite{} in 1973.
** Some background/literature review here Cockburn, Shui for solving hyperbolic equations etc ***
% PhDJesusAlvarez good for motivation - not so much for the method

As seen in Chapter~\ref{PhysicalProblemChapter}, given a suitable choice of initial conditions the evolution of the system in time can be described by Maxwell's curl equations in strong form,~\eqref{eq:maxwell-curl-equations-conservation-form}. Consider that the problem is defined on a physical domain, $\Omega$, which is discretised by an unstructured mesh of $\elemtot$ nonoverlapping and body-conforming simplices, $\elem$, such that $ \Omega \approx \Omega_h = \mathop{\bigcup}_{k=1}^{K} \elem $.
% discuss the discretisation more - duplication of nodes?
Following the method of weighted residuals on a single element we multiply the strong form,~\eqref{strong-form-DG}, by a vector of test functions $\TF$ and integration over an element $\elem$,
$$
\int_{\elem} \TF \cdot \uet \delem  + \int_{\elem} \TF \cdot \dpart{\Flux_{k}}{x_{k}}= \int_{\elem} \TF \cdot \maxwellSource \delem,
$$
where $\Ue$ denotes the restriction of the solution, $\USoltn$, to the element $\elem$. After integration by parts the weak form is obtained as

\begin{equation}
\int_{\elem} \TF \cdot \uet \delem  - \int_{\elem} \dpart{\TF}{x_k} \cdot
\Flux_{k}(\Ue) \delem + \int_{\elemtrace} \TF \cdot \NormalFlux(\Ue) \delemtrace
= \int_{\elem} \TF \cdot \maxwellSource(\Ue) \delem,
\label{eq:weak-form-with-physical-flux}
\end{equation}
where $\mathbf{F_n}$, the outward normal flux, is given by $ \mathbf{F_n}(\mathbf{U}) = \outnormalcoeffk \mathbf{F}_k(\mathbf{U}) $, with $\outnormalcoeffk$ being the $\outnormalcoeffcomp$th component of the outward unit normal vector of $\elemtrace$.

Since the weak form stated in~\eqref{eq:weak-form-with-physical-flux} is specified on the element $\elem$, this does not constitute a scheme suitable for solving the global problem. In order to recover the global solution the continuity of the solution between elements is weakly enforced by replacing the physical normal flux, $\NormalFlux(\Ue)$, with a consistent numerical flux, $\NumFlux(\Ue,\Uout)$. This flux along the element trace of $\elem$ is evaluated in terms of the solution, $\mathbf{U}_e$, in the element $\elem$, and the solution in the neighbouring element, $\mathbf{U}^{out}$. The DG weak formulation is this written as
\begin{equation}
\int_{\elem} \TF \cdot \uet \delem  - \int_{\elem} \dpart{\TF}{x_k} \cdot
\Flux_{k}(\Ue) \delem + \int_{\elemtrace} \TF \cdot \NumFlux(\Ue) \delemtrace
= \int_{\elem} \TF \cdot \maxwellSource(\Ue) \delem,
\label{eq:weak-form-with-numerical-flux}
\end{equation}

% TODO: What is the definition of Uout in mathematical language
\subsection{Choice of Numerical Flux}
Flux splitting technique...Donna and Huerta....this is the change
We write the physical normal flux as
$$
\NormalFlux(\USoltn) = \An \USoltn
$$
where $\An = \outnormalcoeffk \Ak$, which can be decomposed into incoming and outgoing fluxes
\begin{align}
\NormalFlux(\USoltn) = \NormalFluxPositiveEigenvalues(\USoltn) + \NormalFluxNegativeEigenvalues(\USoltn)
\label{eq:phys-flux-splitting}
\end{align}
where $ \NormalFluxPositiveEigenvalues = \AnPlus \USoltn$, $\NormalFluxNegativeEigenvalues = \AnMinus \USoltn, $ and the matrices $\AnMinus$ and $\AnPlus$ denote respectively the matrices of the positive and negative eigenvalues of $\An$. These can be written conveniently as
\begin{align}
  \label{eq:AnMinus-AnPlus-Definition}
\AnPlus &= \left( \An + \AnMod \right) / 2   \\
\AnMinus &= \left( \An - \AnMod \right) / 2
\end{align}
A choise of an upwind numerical flux [***] corresponds to
% TODO: citation
\begin{align}
\NumFlux(\USoltn) = \NormalFluxPositiveEigenvalues(\USoltn) + \NormalFluxNegativeEigenvalues(\Uout).
\label{eq:num-flux-splitting}
\end{align}
After integration by parts,~\eqref{eq:weak-form-with-numerical-flux} can be written as
\begin{align*}
\int_{\elem} \TF \cdot \uet \delem  + \int_{\elem} \TF \cdot \dpart{\Flux_{k}(\Ue)}{x_k} \delem + \int_{\elemtrace} \TF \cdot \left[ \NumFlux(\Ue,\Uout) - \NormalFlux(\Ue) \right] \delemtrace \\
= \int_{\elem} \TF  \cdot \maxwellSource(\Ue) \delem,
% the second term changes from + to - from prev weak form
\label{eq:weak-form-upwind-splitting-fluxes}
\end{align*}
By substitution of~\eqref{eq:num-flux-splitting} and~\eqref{eq:phys-flux-splitting} into~\eqref{eq:AnMinusU-defn} we note that
\begin{align*}
  \NumFlux(\Ue,\Uout) - \NormalFlux(\Ue) = \AnMinus \Uout - \AnMinus \USoltn = \AnMinusU,
\end{align*}
where the jump operator has been defined as $\JumpU = \Uout - \USoltn$. By substitution into~\eqref{eq:weak-form-upwind-splitting-fluxes}, the weak form with upwind flux splitting is written as
\begin{equation}
\int_{\elem} \TF \cdot \uet \delem  + \int_{\elem} \TF \cdot \dpart{\Flux_{k}(\Ue)}{x_k} \delem + \int_{\elemtrace} \TF \cdot \AnMinusU \delemtrace = \int_{\elem} \TF  \cdot \maxwellSource(\Ue) \delem,
\label{eq:weak-form-with-physical-flux}
\end{equation}
% TODO: missing some stuff on diagonalisation of A here....is it necessary, also the form of the
% positive and negative eigenvalues.
% TODO: Missing the form of the numerical flux for DG!!
% *** conditions to be satisfied by numerical flux + form of numerical flux ***
% TODO: upwind flux: effect for wave domination problems - flow of information, upwind flux

\subsubsection{Internal Element Boundaries}
We consider the boundary conditions at an internal boundary between elements. We recall the form of the physical outward normal flux
$$\An = \sum_{k=1}^{\nsd} \outnormalcoeffk \Ak ,$$
where $n_k$ are the directional cosines of the outward normal vector, $\outnormalvector$, on the element boundary. We recall from~\ref{sec:conservation-form} that this expression may be written as
$$
  \An =
  \begin{pmatrix}
 & \zerom , & \mu^{-1} \RTotNorm, & \zerom \\
 & - \eps^{-1} \RTotNorm & \zerom & \zerom \\
 & \zerom & \zerom & \zerom 
 & \end{pmatrix}
$$
where,
$$
  \RTotNorm =
  \begin{pmatrix}
 & 0 & n_3 & -n_2 \\
 & -n_3 & 0 & n_1 \\
& n_2 & -n_1 & 0 
 & \end{pmatrix} .
$$
The modulus of $\An$ is given by
\begin{align*}
\AnMod = \speedoflight
\begin{pmatrix}
  \modAnSubMatrix & \zerom & \zerom \\
  \zerom  & \modAnSubMatrix & \zerom \\
   \zerom & \zerom & \zerom 
\end{pmatrix}
\end{align*}
where $\speedoflight = \left( \epsilon \mu  \right)^{-\frac{1}{2}}$ is the speed of light in the medium and
\begin{align*}
  \modAnSubMatrix = 
\begin{pmatrix}
\outnormalcoeff_2^2 + \outnormalcoeff_3^2 &      -\outnormalcoeff_1 \outnormalcoeff_2 &      -\outnormalcoeff_1 \outnormalcoeff_3 \\
-\outnormalcoeff_1 \outnormalcoeff_2 & \outnormalcoeff_1^2 + \outnormalcoeff_3^2 &      -\outnormalcoeff_2 \outnormalcoeff_3 \\
-\outnormalcoeff_1 \outnormalcoeff_3 &      -\outnormalcoeff_2 \outnormalcoeff_3 & \outnormalcoeff_1^2 + \outnormalcoeff_2^2 \\
\end{pmatrix} .
\end{align*}
Note that the identity $\sqrt{\sum_{\outnormalcoeffcomp} \outnormalcoeffk^2} = 1$, for the unit vector $\outnormalvector$, has been used to simplify this expression. We therefore write
\begin{align*}
\AnMinus = \speedoflight
\begin{pmatrix}
  -\modAnSubMatrix & \RTotNorm & \zerom \\
  -\RTotNorm  & -\modAnSubMatrix & \zerom \\
   \zerom & \zerom & \zerom 
\end{pmatrix} .
\end{align*}

By noting that for any vector $\RTotNorm \anyVector = \outnormalvector \times \anyVector$ and $\modAnSubMatrix \anyVector = \outnormalvector \times \left(  \outnormalvector
  \times \anyVector \right)$,for any vector $\anyVector$, we note that
\begin{align*}
\AnMinusU = \frac{1}{2}
\begin{pmatrix}
  -\nvect \times \left( \JumpH + \sqrt{\frac{\epsilon}{\mu}} \nvect \times \JumpE \right) \\
   \nvect \times \left( \JumpE + \sqrt{\frac{\mu}{\epsilon}} \nvect \times \JumpH \right) \\
  \zerov
\end{pmatrix} .
\end{align*}
% Should I write out A_n for the TEz and TMz modes?
Similarily, for the $\TEz$ mode we obtain
\begin{align*}
\AnMinusU =
  \frac{1}{2}
  \left[
    \Jump{H_3} + \sqrt{\frac{\epsilon}{\mu}}
    \left(
      \outnormalcoeff_1 \Jump{E_2} - 
      \outnormalcoeff_2 \Jump{E_1}
    \right)
  \right]
\begin{pmatrix}
   -\outnormalcoeff_2 \\
   \outnormalcoeff_1 \\
   - \sqrt{ \frac{\mu}{\epsilon} } \\
   0  \\
   0 
\end{pmatrix} .
\end{align*}
and for the $\TMz$ mode
\begin{align*}
\AnMinusU =
  \frac{1}{2}
  \left[
    \Jump{E_3} + \sqrt{\frac{\mu}{\epsilon}}
    \left(
      \outnormalcoeff_2 \Jump{H_1} - 
      \outnormalcoeff_1 \Jump{H_2}
    \right)
  \right]
\begin{pmatrix}
   \outnormalcoeff_2 \\
   -\outnormalcoeff_1 \\
   -\sqrt{ \frac{\epsilon}{\mu} } \\
   0 
\end{pmatrix} .
\end{align*}

% VIVA: calculated with matlab script:
% modAn = sqrtm(An*An') and knowing that
% make sure I can do sqrtm by hand
\subsection{Spatial Discretisation}
The system is discretised by approximating the solution using a nodal basis as
\begin{align}
\USoltn_e(\xbf,\t) \simeq \sum_{j=1}^{\nen} \SF_{j} (\xbf) \mathbf{\ucoeff}_{j}(\t) ,
\label{eq:nodal-basis-defn}
\end{align}
where $N_{i}$ are Lagrangian shape functions, $\nen$ is the number of nodal
points, $u_{i}$ is the nodal solution at the $i$th node and $U^{k}_e$ denotes
the $k$th component of the vector, $\Ue$. Following the Galerkin method, the
vector of test functions, $\TF$, is chosen to be the same basis as the shape
functions, with the $k$th component given by
\begin{align}
\TFComp^{k} = \sum_{i=1}^{n} \SF_{i}(\xbf).
\label{eq:test-function-comp-defn}
\end{align}
By substitution of~\eqref{eq:nodal-basis-defn} and~\eqref{eq:test-function-comp-defn} into~\eqref{eq:weak-form-final} we obtain the discretised weak form
\begin{align*}
\sum_{i,j} \left[  \dodet{\ucoeff_i} \int_{\elem} \SF_i \SF_j \delem   +
\ucoeff_i \Ak \int_{\elem} \SF_i
  \dode{\SF_j}{x_k} \delem  +
\AnMinus \JumpUCoeff \int_{\elemtrace} \SF_i \SF_j
  \delemtrace 
-
\Asource \int_{\elem} \SF_i \SF_j \delem \right]  = 0,
\end{align*}
% TODO - this is wrong...I'm mixing vectors and scalars...
which can be written as a system of ordinary differential equations

$$
\MassMatrix \dodet{\UVect} + \FluxDiv + \FluxNumFlux - \Asource \MassMatrix = 0
$$
% TODO - are A_s and M in wrong order? order matters heres. Same for A_k. Also matters below.
% TODO - A_s has nor been defined....!
where $\UVect$ is a vector of the solution coefficients $u_{i}$. The block diagonal mass matrix, $\MassMatrix$, and the divergence and flux vectors, $\FluxDiv$ and $\FluxNumFlux$, are then given by
\begin{align*}
\MassMatrixComponent_{ij} &= \int_{\elem} \SF_i \SF_j \delem \\
\FluxDiv_{i} &=  \sum_{k=1}^{\nsd} a_k \sum_{j=1}^{\nen} \int_{\elem} \left( \dpart{\SF_i}{x_k} \dpart{\SF_j}{x_k} \delem \right) \ucoeff_j \\
\FluxNumFlux_{i} &= \sum_{j=1}^{\nsd} \an^{-}
\left( 
\int_{\elemtrace} \SF_i \SF_j
  \delemtrace
 \right)
\JumpUCoeffSpecifyIndex{j} \\
\end{align*}
% TODO - can I just use a scalar a_k here? or is this rubbish
The resulting system of equations can be written as
$$
\dodet{\UVect} + \Residual(\UVect) = 0,
$$
where $\Residual = \MassMatrix^{-1} \left[  \FluxDiv + \FluxNumFlux \right] - \Asource$, is the residual vector.

The solution is advanced in time with an explicit fourth order Runge-Kutta (RK4) method. Implicit schemes which allow larger time steps may be employed to obtain the final solution of the system of equations in a shorter computational time. However as we will be seen shortly, in~\autoref{Ch:SignalAnalysis}, the highest frequency which can be resolved is inversely proportional by time step length, in which case a short time step may be desirable.
% TODO - justify high order time integration....!
% TODO: RK4 Stability condition 
% TODO: Also RK4 method
% TODO: Local matrix + broken space
\subsection{Jump Conditions}
\begin{itemize}
	\item material interfaces (PEC,ABC)
  \item will need to mention PML -> another section
\end{itemize}

For interfaces which intersect the domain boundary, $\partial \Omega$, the not all components of $\mathbf{U}^{out}$ are determined by the boundary conditions on the interface. For a system of conservation laws Rankine-Hugoniot jump conditions of the form

$$
\Jump{ \mathbf{F}_n } = \lambda_j \Jump{ \mathbf{U} }
$$

where $\lambda_j$ are the eigenvalues of the jacobian matrix $\mathbf{A}_n$.
This condition should be satisfied along the characteristics in the phase plane.
For the 3 dimensions these are $ \lambda_{ 1,2 } = - \speedoflightleft $, $
\lambda_{ 3,4 } = \speedoflightright $ and $\lambda_{5..9} = 0 $, where the
$\speedoflightleft$ and $\speedoflightright$ are the velocities of the
electromagnetic wave in media on the left and right side of the interface
respectively. This condition should be satisfied along the phase plane
characteristics, as shown in \ref{fig:phase-plane-characteristics}.

\begin{figure}[h]
  \centering
  
  \caption{Phase plane diagram showing the characteristics for Maxwells' equations}
  \label{fig:phase-plane-characteristics}
\end{figure}

\subsection{Rankine-Hugoniot Jump Conditions in 2D}
For the $\TEz$ and $\TMz$ modes, by setting..., the normal physical flux can be
written in the form
$$
\mathbf{F}_n = 
\begin{pmatrix}
  - n_2 \UFieldComp_3 \\
  n_1 \UFieldComp_3 \\
  \alphaGeneral
\end{pmatrix}
$$
where $ \alphaGeneral = n_1 \UFieldComp_2 - n_2 \UFieldComp_1 $, and the vector $\UField$ is given by $
\UField = 
  \begin{pmatrix} 
    E_1 \; E_2 \; H_3
  \end{pmatrix}^T
$ for the $\TEz$ mode, and $
\UField = 
  -
  \begin{pmatrix} 
    H_1 \; H_2 \; E_3
  \end{pmatrix}^T
  $ for the $\TMz$ mode.
For dispersive media this will be expanded with zeros...
% TODO - CHECK OUT THE A_n vectors and check that these F_n come nicely from there...
% TODO - should I have a numbered lambda here? and would that be different for
% TE and TM modes
By applying the Rankine-Hugoniot condition along $\frac{\partial x}{\parital t}
= 0$, with $ \Jump{ \mathbf{F}_n } = 0 $, the normal flux in region $\LStar$
and $\RStar$ are equal, we denote the flux in these region as
$$
\mathbf{F}_n^{out} = 
\begin{pmatrix}
  - n_2 \UFieldComp_3^{out} \\
  n_1 \UFieldComp_3^{out} \\
  \alphaGeneral^{out}
\end{pmatrix} .
$$

Along $\frac{\partial x}{\parital t} = c_R$, the Rankine-Hugoniot condition takes the form
$$
\Jump{ \mathbf{F}_n } = \speedoflightright \Jump{ \mathbf{U} }
$$
which results in
\begin{align}
-n_2 \left(V_3^{R} - V_3^{out} \right) &= \micoeff_R \left( V_1^{R} - V_1^{out} \right) \\ \label{RHR-sys-1}
n_1 \left(V_3^{R} - V_3^{out} \right) &= \micoeff_R  \left( V_2^{R} - V_2^{out} \right) \\  \label{RHR-sys-2}
\micoeff_R \left(\alphaGeneral^{R} - \alphaGeneral^{out} \right) &= \left( V_3^{R} - V_3^{out} \right) \label{RHR-sys-3}
\end{align}
where $\micoeff = \sqrt{\epsilon/\mu}$ for $\TEz$ mode and $\micoeff =
\sqrt{\mu/\epsilon}$ for $\TMz$ mode. It can be easily verified that
\eqref{RHR-sys-1} and \eqref{RHR-sys-2} imply \eqref{RHR-sys-3}.

Along the $\frac{\partial x}{\parital t} = - \speedoflightleft$, the Rankine-Hugoniot condition
takes the form
$$
\Jump{ \mathbf{F}_n } = - \speedoflightleft \Jump{ \mathbf{U} }
$$
which results in
\begin{align}
-n_2 \left(V_3^{L} - V_3^{out} \left) &= - \micoeff_L \left( V_1^{L} - V_1^{out} \left) \\  \label{RHL-sys-1}
n_1 \left(V_3^{L} - V_3^{out} \left) &= - \micoeff_L  \left( V_2^{L} - V_2^{out} \left) \\ \label{RHL-sys-2}
- \micoeff_L \left( \alphaGeneral^{L} - \alphaGeneral^{out} \right) &= \left( V_3^{L} - V_3^{out} \left) \label{RHL-sys-3}
\end{align}
As above, \eqref{RHL-sys-1} and \eqref{RHL-sys-2} imply \eqref{RHL-sys-3}.
For the $\TEz$ and $\TMz$ modes solving the Ranking-Hugoniot condition is equivalent to solving the system
\begin{align}
\micoeff_R \left(\alphaGeneral^{R} - \alphaGeneral^{out} \right) &= \left( V_3^{R} - V_3^{out} \right) \\
- \micoeff_L \left( \alphaGeneral^{L} - \alphaGeneral^{out} \right) &= \left( V_3^{L} - V_3^{out} \left)
\end{align}
Solving the linear system results given by \eqref{RHL-sys-3} and
\eqref{RHR-sys-3} results in
% divide by c* coeff then 1-2 again
\begin{equation}
\UFieldComp_3^{out} = \frac{\micoeff_R^{-1} \UFieldComp_3^R + \micoeff_L^{-1} \UFieldComp_3^L - \left(\alphaGeneral^R - \alphaGeneral^L \right)}{\micoeff_L^{-1} + \micoeff_R^{-1}} \label{eq:interface-bc-V3}
\end{equation}
and
% 1 - 2 and rearrange
\begin{equation}
\alphaGeneral^{out} = \frac{\UFieldComp_3^R - \UFieldComp_3^L - \micoeff_R \alphaGeneral^R - \micoeff_L \alphaGeneral^L}{\micoeff_L + \micoeff_R}\label{eq:interfce-bc-alphaGeneral}
\end{equation}


\subsection{Material Interfaces}
...what is a material interface for 3D...
%TODO ...

material interfaces for 3D...by solving linear system...
For the $\TEz$ mode this results in the conditions
% divide by c* coeff then 1-2 again
\begin{equation}
H_3^{out} = \frac{\speedoflightright \mu_R H_3^R + \speedoflightleft \mu_L H_3^L - \left(\alphaGeneral^R - \alphaGeneral^L \right)}{\speedoflightright \mu_R + \speedoflightleft \mu_L } \label{eq:interface-bc-V3}
\end{equation}
and
\begin{equation}
\alphaGeneral^{out} = \frac{\speedoflightright \epsilon_R \alphaGeneral^R + \speedoflightleft \epsilon_L \alphaGeneral^L - \left( H_3^R - H_3^L \right) }{\speedoflightright \epsilon_R + \speedoflightleft \epsilon_L}
\end{equation}
with
$$ \alphaGeneral = n_1 E_2 - n_2 E_1 $$
and for the $\TMz$ this results in 
% divide by c* coeff then 1-2 again
\begin{equation}
E_3^{out} = \frac{\speedoflightright \epsilon_R E_3^R + \speedoflightleft \epsilon_L E_3^L - \left(\alphaGeneral^R - \alphaGeneral^L \right)}{\speedoflightright \epsilon_R + \speedoflightleft \epsilon_L }
\end{equation}
and
\begin{equation}
\alphaGeneral^{out} = \frac{\speedoflightright \mu_R \alphaGeneral^R + \speedoflightleft \mu_L \alphaGeneral^L - \left( E_3^R - E_3^L \right) }{\speedoflightright \mu_R + \speedoflightleft \mu_L}
\end{equation}
with $$ \alphaGeneral = n_1 E_2 - n_2 E_1 $$

\subsection{Absorbing boundary condition}
...infinite
domain....truncation....finite....computation....
ABC...is...artificial.....EM radiation absorbed....

This is obtained by...Only flux out....
numerical flux becomes...
\begin{align}
\NumFlux(\USoltn,\Uout) = \NormalFluxPositiveEigenvalues(\USoltn) = \AnPlus \USoltn
\end{align}
% TODO - U here shouldn't have element subscript
or equivalently....there is no inward flux....
$$
\AnMinus \JumpU = - \AnMinus \USoltn
$$
% TODO I don't get this expression...
no incoming flux equiv to 1st order S-M condition

can be used with no PML, to dissipate waves as they get to boundary...etc...
Kabakian et al (2004)

\subsection{Residual Vector}
recall
$$
\Residual = \MassMatrix^{-1} \left[  \FluxDiv + \FluxNumFlux \right] - \Asource
$$

need to integrate over element interior for div term (and source term)
need to integrate over element faces for boundary terms


\subsubsection{Spatial Discretisation}
  
\begin{align*}
\MassMatrixComponent_{ij} &= \int_{\elem} \SF_i \SF_j \delem \\
\FluxDiv_{i} &=  \sum_{k=1}^{\nsd} a_k \sum_{j=1}^{\nen} \int_{\elem} \left( \dpart{\SF_i}{x_k} \dpart{\SF_j}{x_k} \delem \right) \ucoeff_j \\
\FluxNumFlux_{i} &= \sum_{j=1}^{\nsd} \an^{-}
\left( 
\int_{\elemtrace} \SF_i \SF_j
  \delemtrace
 \right)
\JumpUCoeffSpecifyIndex{j} \\
\end{align*}
\subsubsection{Planar Elements}

...with planar faces
...define a linear mapping ref elem (I) -> physical element.
isoparametric mapping given by coords of the vertices of Omega
...then $\FluxNumFlux_{i}$ integral 
\begin{align*}
\FluxDiv_{i} &=  \sum_{k=1}^{\nsd} a_k \sum_{j=1}^{\nsd} \int_{\elem} \left( \dpart{\SF_i}{x_k} \cdot \dpart{\SF_j}{x_k} \delem \right) \ucoeff_j ,
\end{align*}
is transformed to the reference element (i.e computed in the ref elem as
\begin{align*}
\FluxDiv_{i} &= |J| \sum_{k=1}^{\nsd} a_k \sum_{j=1}^{\nsd} J_{l,k}^{-1} \sum_{l=1}^{\nsd}\int_{\elem} \left( \dpart{\SF_i}{x_k} \cdot \dpart{\SF_j}{x_k} \delem \right) \ucoeff_j ,
\end{align*}
% TODO - this is not complete....I need to know how to transform to reference
% element. Generally just do this by chain rule...simples...think about 1D first
a note here about how the jacobian |J| (of the linear mapping I -> Omega) is a constant
and can be taken out of calculation
Then the 'convection elemental matricies' $ C_{i,j}^{\xi_l} $ (i.e. the stuff in brackets) is
precomputed once only per reference element
and the whole thing is computed as
\begin{align*}
\FluxDiv_{i} &= |J| \sum_{k=1}^{\nsd} a_k \sum_{j=1}^{\nsd} J_{l,k}^{-1} \sum_{l=1}^{\nsd} C_{i,j}^{\xi_l} \ucoeff_j ,
\end{align*}
i.e. now we're doing the same thing without any integration

The flux term (i.e. the one on faces) similar...
This time with a reference face...
mapping R_I between reference face and physical face R_e
Then compute the integral $\FluxNumFlux_{i}$ as

$$
\FluxNumFlux_{i} &=
\an^{-}
\sum_{j=1}^{\nfn}
\left( 
\int_{\elemtrace} \SF_i \SF_j
  \delemtrace
 \right)
\JumpUCoeffSpecifyIndex{j} \\
$$

a_n^{-} now outside the integral - because its a constant for a given face
(the normal is constant over face)
sum is over face nodes, since for non-face nodes the thing vanishes
% TODO - why does the thingy vanish
Ruben quote
'because other shape functions vanish over the complete planar face'

so I can do the same thing
$$
\FluxNumFlux_{i} &=
\an^{-}
\sum_{j=1}^{\nfn}
m_{ij}
\JumpUCoeffSpecifyIndex{j} \\
$$
note: ruben uses Gamma as the superscript for this part of num flux

with
$$
m_{ij} =
\int_{\elemtrace} \SF_i \SF_j
  \delemtrace
$$
m_ij == 'mass matrixfor the reference face'
computed only once (not once per face or anything like that....apparently!??)
'and used in the computation of each physical face'

this is all ``quadrature free'' implementation of DG methods (Atkins and Shu,
1998) -> can be done because: I can actually reference Rubens paper here
The use of hybrid meshes to improve the efficiency of a discontinuous Galerkin
method for the solution of Maxwell's equations

A_n^- is constant (i.e. linear eqtns) and
using triangular or tetrahedral meshes (i.e. constant Jacobian for non-curved
elements)
---> is this is also true for Quads? Or just affine/planar quads?

\subsubsection{Curved Elements}

I stopped reading Rubens Thesis at this point and switched to the paper...

There is a lot of stuff in the thesis about curved elements...

need to justify why NEFEM is actually pretty quick if elements are
STORED...combine this with parallel....awesome...

Fekette nodal distribution



\begin{itemize}
	\item discuss types of element - (non-)affine, planar, different shapes etc.
	\item condition number of a matrix - moving nodes around %reference!
\end{itemize}

\subsection{Errors and convergence}
\begin{itemize}
  \item expected rates of convergence for time-domain (interpolation error) and freq domain (dispersion error)
\end{itemize}

%%% Local Variables:
%%% mode: latex
%%% TeX-master: "../Thesis"
%%% End:
