\chapter{Discontinuous Galerkin Method for Maxwells Equations} % Write in your own chapter title
\label{Chapter3}
\lhead{Chapter 3. \emph{Discontinuous Galerkin}} % Write in your own chapter title to set the page header

\subsection{Formulation}

The Discontinuous Galerkin method was first introduced to solve the neutron transport problem by Reed and Hill \cite{} in 1973.
** Some background/literature review here Cockburn, Shui for solving hyperbolic equations etc ***
% PhDJesusAlvarez good for motivation - not so much for the method

As seen in Chapter~\ref{PhysicalProblemChapter}, given a suitable choice of initial conditions the evolution of the system in time can be described by Maxwell's curl equations in strong form,~\eqref{eq:maxwell-curl-equations-conservation-form}. Consider that the problem is defined on a physical domain, $\Omega$, which is discretised by an unstructured mesh of $\elemtot$ nonoverlapping and body-conforming simplices, $\elem$, such that $ \Omega \approx \Omega_h = \mathop{\bigcup}_{k=1}^{K} \elem $.
% discuss the discretisation more - duplication of nodes?
Following the method of weighted residuals on a single element we multiply the strong form,~\eqref{strong-form-DG}, by a vector of test functions $\TF$ and integration over an element $\elem$,
$$
\int_{\elem} \TF \cdot \uet \delem  + \int_{\elem} \TF \cdot \dpart{\Flux_{k}}{x_{k}}= \int_{\elem} \TF \cdot \maxwellSource \delem,
$$
where $\Ue$ denotes the restriction of the solution, $\USoltn$, to the element $\elem$. After integration by parts the weak form is obtained as

\begin{equation}
\int_{\elem} \TF \cdot \uet \delem  - \int_{\elem} \dpart{\TF}{\xk} \cdot
\Flux_{k}(\Ue) \delem + \int_{\elemtrace} \TF \cdot \NormalFlux(\Ue) \delemtrace
= \int_{\elem} \TF \cdot \maxwellSource(\Ue) \delem,
\label{eq:weak-form-with-physical-flux}
\end{equation}
where $\mathbf{F_n}$, the outward normal flux, is given by $ \mathbf{F_n}(\mathbf{U}) = \outnormalcoeffk \mathbf{F}_k(\mathbf{U}) $, with $\outnormalcoeffk$ being the $\outnormalcoeffcomp$th component of the outward unit normal vector of $\elemtrace$.

Since the weak form stated in~\eqref{eq:weak-form-with-physical-flux} is specified on the element $\elem$, this does not constitute a scheme suitable for solving the global problem. In order to recover the global solution the continuity of the solution between elements is weakly enforced by replacing the physical normal flux, $\NormalFlux(\Ue)$, with a consistent numerical flux, $\NumFlux(\Ue,\Uout)$. This flux along the element trace of $\elem$ is evaluated in terms of the solution, $\mathbf{U}_e$, in the element $\elem$, and the solution in the neighbouring element, $\mathbf{U}^{out}$. The DG weak formulation is this written as
\begin{equation}
\int_{\elem} \TF \cdot \uet \delem  - \int_{\elem} \dpart{\TF}{\xk} \cdot
\Flux_{k}(\Ue) \delem + \int_{\elemtrace} \TF \cdot \NumFlux(\Ue) \delemtrace
= \int_{\elem} \TF \cdot \maxwellSource(\Ue) \delem,
\label{eq:weak-form-with-numerical-flux}
\end{equation}

% TODO: What is the definition of Uout in mathematical language
% TODO:
% Ruben/Oubay: This numerical flux is evaluated in terms of the trace of
% the solution on element Xe and the trace of the solution, Uout, on the
% other element adjacent to Ce
\subsection{Choice of Numerical Flux}
Flux splitting technique~\cite{donea2003finite} this is the change


QUOTE: R/O: A natural choice, for the linear hyperbolic
system of interest here, is to employ a flux splitting technique
~\cite{donea2003finite}, which corresponds to an upwind approximation \cite{chen2005high}

We write the physical normal flux as
$$
\NormalFlux(\USoltn) = \An \USoltn
$$
where $\An = \outnormalcoeffk \Ak$, which can be decomposed into incoming
(superscript -) and outgoing (superscript +) fluxes
\begin{align}
\NormalFlux(\USoltn) = \NormalFluxPositiveEigenvalues(\USoltn) + \NormalFluxNegativeEigenvalues(\USoltn)
\label{eq:phys-flux-splitting}
\end{align}
where $ \NormalFluxPositiveEigenvalues = \AnPlus \USoltn$, $\NormalFluxNegativeEigenvalues = \AnMinus \USoltn, $ and the matrices $\AnMinus$ and $\AnPlus$ denote respectively the matrices of the positive and negative eigenvalues of $\An$. These can be written conveniently as
\begin{align}
  \label{eq:AnMinus-AnPlus-Definition}
\AnPlus &= \left( \An + \AnMod \right) / 2   \\
\AnMinus &= \left( \An - \AnMod \right) / 2
\end{align}
A choise of an upwind numerical flux [***] corresponds to
% TODO: citation
\begin{align}
\NumFlux(\USoltn) = \NormalFluxPositiveEigenvalues(\USoltn) + \NormalFluxNegativeEigenvalues(\Uout).
\label{eq:num-flux-splitting}
\end{align}
After integration by parts,~\eqref{eq:weak-form-with-numerical-flux} can be written as
\begin{align*}
\int_{\elem} \TF \cdot \uet \delem  + \int_{\elem} \TF \cdot \dpart{\Flux_{k}(\Ue)}{\xk} \delem + \int_{\elemtrace} \TF \cdot \left[ \NumFlux(\Ue,\Uout) - \NormalFlux(\Ue) \right] \delemtrace \\
= \int_{\elem} \TF  \cdot \maxwellSource(\Ue) \delem,
% the second term changes from + to - from prev weak form
\label{eq:weak-form-upwind-splitting-fluxes}
\end{align*}
We note that % TODO - I've lost some text here...!
\begin{align*}
  \NumFlux(\Ue,\Uout) - \NormalFlux(\Ue) = \AnMinus \Uout - \AnMinus \USoltn = \AnMinusU,
\end{align*}
where the jump operator has been defined as $\JumpU = \Uout - \USoltn$. By substitution into~\eqref{eq:weak-form-upwind-splitting-fluxes}, the weak form with upwind flux splitting is written as
\begin{equation}
\int_{\elem} \TF \cdot \uet \delem  + \int_{\elem} \TF \cdot \dpart{\Flux_{k}(\Ue)}{\xk} \delem + \int_{\elemtrace} \TF \cdot \AnMinusU \delemtrace = \int_{\elem} \TF  \cdot \maxwellSource(\Ue) \delem,
\label{eq:weak-form-final}
\end{equation}

% TODO: missing some stuff on diagonalisation of A here....is it necessary, also the form of the
% positive and negative eigenvalues.
% TODO: Missing the form of the numerical flux for DG!!
% *** conditions to be satisfied by numerical flux + form of numerical flux ***
% TODO: upwind flux: effect for wave domination problems - flow of information, upwind flux

\subsubsection{Internal Element Boundaries}
We consider the boundary conditions at an internal boundary between elements. We recall the form of the physical outward normal flux
$$\An = \sum_{k=1}^{\nsd} \outnormalcoeffk \Ak ,$$
where $n_k$ are the directional cosines of the outward normal vector, $\outnormalvector$, on the element boundary. We recall from~\ref{sec:conservation-form} that this expression may be written as
$$
  \An =
  \begin{pmatrix}
 & \zerom , & \mu^{-1} \RTotNorm, & \zerom \\
 & - \eps^{-1} \RTotNorm & \zerom & \zerom \\
 & \zerom & \zerom & \zerom 
 & \end{pmatrix}
$$
where,
$$
  \RTotNorm =
  \begin{pmatrix}
 & 0 & n_3 & -n_2 \\
 & -n_3 & 0 & n_1 \\
& n_2 & -n_1 & 0 
 & \end{pmatrix} .
$$
The modulus of $\An$ is given by
\begin{align*}
\AnMod = \speedoflight
\begin{pmatrix}
  \modAnSubMatrix & \zerom & \zerom \\
  \zerom  & \modAnSubMatrix & \zerom \\
   \zerom & \zerom & \zerom 
\end{pmatrix}
\end{align*}
where $\speedoflight = \left( \epsilon \mu  \right)^{-\frac{1}{2}}$ is the speed of light in the medium and
\begin{align*}
  \modAnSubMatrix = 
\begin{pmatrix}
\outnormalcoeff_2^2 + \outnormalcoeff_3^2 &      -\outnormalcoeff_1 \outnormalcoeff_2 &      -\outnormalcoeff_1 \outnormalcoeff_3 \\
-\outnormalcoeff_1 \outnormalcoeff_2 & \outnormalcoeff_1^2 + \outnormalcoeff_3^2 &      -\outnormalcoeff_2 \outnormalcoeff_3 \\
-\outnormalcoeff_1 \outnormalcoeff_3 &      -\outnormalcoeff_2 \outnormalcoeff_3 & \outnormalcoeff_1^2 + \outnormalcoeff_2^2 \\
\end{pmatrix} ,
\end{align*}
where the identity $\sqrt{\sum_{\outnormalcoeffcomp} \outnormalcoeffk^2} = 1$, for the unit vector $\outnormalvector$, has been used. We therefore write
\begin{align*}
\AnMinus = \speedoflight
\begin{pmatrix}
  -\modAnSubMatrix & \RTotNorm & \zerom \\
  -\RTotNorm  & -\modAnSubMatrix & \zerom \\
   \zerom & \zerom & \zerom 
\end{pmatrix} .
\end{align*}

Noting that $\RTotNorm \anyVector = \outnormalvector \times \anyVector$ and $\modAnSubMatrix \anyVector = \outnormalvector \times \left(  \outnormalvector
  \times \anyVector \right)$,for any vector $\anyVector$, results in the expression
\begin{align*}
\AnMinusU = \frac{1}{2}
\begin{pmatrix}
  -\nvect \times \left( \JumpH + \sqrt{\frac{\epsilon}{\mu}} \nvect \times \JumpE \right) \\
   \nvect \times \left( \JumpE + \sqrt{\frac{\mu}{\epsilon}} \nvect \times \JumpH \right) \\
  \zerov
\end{pmatrix} .
\end{align*}
% Should I write out A_n for the TEz and TMz modes?
A similar procedure results in
\begin{align*}
\AnMinusU =
  \frac{1}{2}
  \left[
    \Jump{H_3} + \sqrt{\frac{\epsilon}{\mu}}
    \left(
      \outnormalcoeff_1 \Jump{E_2} - 
      \outnormalcoeff_2 \Jump{E_1}
    \right)
  \right]
\begin{pmatrix}
   -\outnormalcoeff_2 \\
   \outnormalcoeff_1 \\
   - \sqrt{ \frac{\mu}{\epsilon} } \\
   0  \\
   0 
\end{pmatrix} .
\end{align*}
for the $\TEz$ mode and
\begin{align*}
\AnMinusU =
  \frac{1}{2}
  \left[
    \Jump{E_3} + \sqrt{\frac{\mu}{\epsilon}}
    \left(
      \outnormalcoeff_2 \Jump{H_1} - 
      \outnormalcoeff_1 \Jump{H_2}
    \right)
  \right]
\begin{pmatrix}
   \outnormalcoeff_2 \\
   -\outnormalcoeff_1 \\
   -\sqrt{ \frac{\epsilon}{\mu} } \\
   0 
\end{pmatrix} .
\end{align*}
for the $\TMz$ mode

% VIVA: calculated with matlab script:
% modAn = sqrtm(An*An') and knowing that
% make sure I can do sqrtm by hand
\subsection{Spatial Discretisation}

R/O Quote: 'Triangles and quadrilaterals are employed to provide a consistent
discretisation of the spatial solution domain, X, for two
dimensional problems. In three dimensions, consistent meshes
consisting of tetrahedra, hexahedra, prisms and pyramids are used.
Apart from the pyramid, which requires special attention, optimal
nodal finite elements of arbitrary order are readily defined for all
these shapes. For the pyramid, a recently proposed approximation
space [28] is adopted. This space is well suited for both continuous
and discontinuous approximations and is optimal, i.e. the a priori
error estimate is Oðhpþ1 Þ in the L2ðXÞ norm, where p denotes the
order of the approximation. The approximation spaces that are employed
are summarised in Table 1' <----

The system is discretised by approximating the solution using a nodal basis as
\begin{align}
\USoltn_e(\xbf,\t) \simeq \sum_{j=1}^{\nen} \SF_{j} (\xbf) \UVect_{j}(\t) ,
\label{eq:nodal-basis-defn}
\end{align}
% VIVA: the second U_j is the vector of coefficients
where $N_{j}$ are $j$ nodal Lagrangian shape functions spanning the approximation space,
$\UVect_{j}$ is the value of the solution at the $j$th node, $\nen$ is the number of nodal
points. Following the Galerkin method, the
vector of test functions, $\TF$, is chosen to be the same basis as the shape
functions
\begin{align}
\TF = \sum_{i=1}^{\nen} \SF_{i}(\xbf).
\label{eq:test-function-comp-defn}
\end{align}
By substitution of~\eqref{eq:nodal-basis-defn} and~\eqref{eq:test-function-comp-defn} into~\eqref{eq:weak-form-final} we obtain the discretised weak form
\begin{align*}
\sum_{i,j} \left[
  % term 1
  \left(
    \int_{\elem} \SF_i \SF_j \delem
  \right)
  \dodet{\UVect_j}
+
  % term 2
  \left(
    \int_{\elem} \SF_i \dpart{\SF_j}{x^k} \delem
  \right)
  \Ak \UVect_j
+
  % term 3
  \left(
  \int_{\elemtrace} \SF_i \SF_j \delemtrace 
  \right)
  \AnMinus \JumpUCoeffVectUnknownsWithIndex{j}
  % term 4
-
  \left(
  \int_{\elem} \SF_i \SF_j \delem
  \right)
  \Asource 
  \right]  = 0,
\end{align*}
% TODO - ok...so the second term actually also has an implicit sum over k in
% it...but this isn't true Einstein notation - where is the double index?
which can be written as a system of $\nen$ ordinary differential equations,
$$
\sum_{j=1}^{\nen}
\left[
\MassMatrix_{ij} \IdentityMatrix \dodet{\UVect_j} +
\left( \ConvectionMatrix^{k}_{ij} \Ak \right) \UVect_j -
\MassMatrix_{ij} \IdentityMatrix \Asource
\right]
- \sum_{j=1}^{\nfn} \MassMatrixFace_{ij} \AnMinus \Jump{\UVect_j}
= 0
$$
% TODO - are A_s and M in wrong order? order matters heres. Same for A_k. Also matters below.
% TODO - A_s has not been defined....! Also am I missing a U to multiply A_s?
for each node $i$, where $\UVect_j$ is a vector of the solution coefficients at
the $j$th node, with
\begin{align*}
\MassMatrixComponent_{ij} &= \int_{\elem} \SF_i \SF_j \delem \\
\ConvectionMatrix^{k}_{ij} &= \int_{\elem} \SF_i \dpart{\SF_j}{\xk}\delem \\
\MassMatrixFace_{ij} &= \int_{\elemtrace} \SF_i \SF_j
  \delemtrace
\end{align*}
where $\MassMatrix$ is the mass matrix, $\ConvectionMatrix_{ij}$ is the
convection matrix in the direction $x^{k}$, $\MassMatrixFace$ is the face
mass matrix and $\IdentityMatrix$ is the identity matrix.
% TODO - introduce \nfn variable here, and more importantly elaborate on the fact
% why nfn is used not nen! Why (zeros)
The resulting system of equations can be written as
$$
\dodet{\UVect} + \Residual(\UVect) = 0.
$$
where the residual vector is given by $\Residual = ( \MassMatrix_{ij} \IdentityMatrix )^{-1} \left[  \FluxDiv + \FluxNumFlux \right] - \Asource$, with $ \FluxDiv &= \sum_{j}^{\nen} \ConvectionMatrix^{k}_{ij} \Ak \UVect_j $ and $ \FluxNumFlux &= \sum_{j}^{\nfn} \MassMatrixFace_{ij} \An \Jump{\UVect_j} $.
% TODO: what is all this crap about M_{ij} and how does it correspond to residual?
% TODO: how should I write residual vector now?


The solution is advanced in time with an explicit fourth order Runge-Kutta (RK4)
method. The time step should be selected to be sufficiently small that the
spatial discretisation error dominates numerical error.
% TODO - maybe this should go in the results section
Implicit schemes which allow larger time steps may be employed to obtain the final solution of the system of equations in a shorter computational time. However as we will be seen shortly, in~\autoref{Ch:SignalAnalysis}, the highest frequency which can be resolved is inversely proportional by time step length, in which case a short time step may be desirable.
% TODO - justify high order time integration....!
% TODO: RK4 Stability condition 
% TODO: Also RK4 method
% TODO: Local matrix + broken space
\subsection{Jump Conditions}
\begin{itemize}
	\item material interfaces (PEC,ABC)
  \item will need to mention PML -> another section
\end{itemize}

For interfaces which intersect the domain boundary, $\partial \Omega$, the not all components of $\mathbf{U}^{out}$ are determined by the boundary conditions on the interface. For a system of conservation laws Rankine-Hugoniot jump conditions of the form

$$
\Jump{ \mathbf{F}_n } = \lambda_j \Jump{ \mathbf{U} }
$$

where $\lambda_j$ are the eigenvalues of the jacobian matrix $\mathbf{A}_n$.
This condition should be satisfied along the characteristics in the phase plane.
For the 3 dimensions these are $ \lambda_{ 1,2 } = - \speedoflightleft $, $
\lambda_{ 3,4 } = \speedoflightright $ and $\lambda_{5..9} = 0 $, where the
$\speedoflightleft$ and $\speedoflightright$ are the velocities of the
electromagnetic wave in media on the left and right side of the interface
respectively. This condition should be satisfied along the phase plane
characteristics, as shown in \ref{fig:phase-plane-characteristics}.

\begin{figure}[h]
  \centering
  
  \caption{Phase plane diagram showing the characteristics for Maxwells' equations}
  \label{fig:phase-plane-characteristics}
\end{figure}

\subsection{Rankine-Hugoniot Jump Conditions in 2D}
For the $\TEz$ and $\TMz$ modes, by setting..., the normal physical flux can be
written in the form
$$
\mathbf{F}_n = 
\begin{pmatrix}
  - n_2 \UFieldComp_3 \\
  n_1 \UFieldComp_3 \\
  \alphaGeneral
\end{pmatrix}
$$
where $ \alphaGeneral = n_1 \UFieldComp_2 - n_2 \UFieldComp_1 $, and the vector $\UField$ is given by $
\UField = 
  \begin{pmatrix} 
    E_1 \; E_2 \; H_3
  \end{pmatrix}^T
$ for the $\TEz$ mode, and $
\UField = 
  -
  \begin{pmatrix} 
    H_1 \; H_2 \; E_3
  \end{pmatrix}^T
  $ for the $\TMz$ mode.
For dispersive media this will be expanded with zeros...
% TODO - CHECK OUT THE A_n vectors and check that these F_n come nicely from there...
% TODO - should I have a numbered lambda here? and would that be different for
% TE and TM modes
By applying the Rankine-Hugoniot condition along $\dpart{x}{t} = 0$, with $ \Jump{ \mathbf{F}_n } = 0 $, the normal flux in region $\LStar$
and $\RStar$ are equal, we denote the flux in these region as
$$
\mathbf{F}_n^{out} = 
\begin{pmatrix}
  - n_2 \UFieldComp_3^{out} \\
  n_1 \UFieldComp_3^{out} \\
  \alphaGeneral^{out}
\end{pmatrix} .
$$

Along $\dpart{x}{t} = \speedoflightright$, the Rankine-Hugoniot condition takes the form
$$
\Jump{ \mathbf{F}_n } = \speedoflightright \Jump{ \mathbf{U} }
$$
which results in
\begin{align}
-n_2 \left(V_3^{R} - V_3^{out} \right) &= \micoeff_R \left( V_1^{R} - V_1^{out} \right) \label{RHR-sys-1} \\
n_1 \left(V_3^{R} - V_3^{out} \right) &= \micoeff_R  \left( V_2^{R} - V_2^{out} \right) \label{RHR-sys-2} \\
\micoeff_R \left(\alphaGeneral^{R} - \alphaGeneral^{out} \right) &= \left( V_3^{R} - V_3^{out} \right) \label{RHR-sys-3}
\end{align}
where $\micoeff = \sqrt{\eps/\mu}$ for $\TEz$ mode and $\micoeff =
\sqrt{\mu/\epsilon}$ for $\TMz$ mode. It can be easily verified that
\eqref{RHR-sys-1} and~\eqref{RHR-sys-2} imply~\eqref{RHR-sys-3}.

Along the $\dpart{x}{t} = - \speedoflightleft$, the Rankine-Hugoniot condition
takes the form
$$
\Jump{ \mathbf{F}_n } = - \speedoflightleft \Jump{ \mathbf{U} }
$$
which results in
\begin{align}
-n_2 \left(V_3^{L} - V_3^{out} \right) &= - \micoeff_L \left( V_1^{L} - V_1^{out} \right) \label{RHL-sys-1} \\
n_1 \left(V_3^{L} - V_3^{out} \right) &= - \micoeff_L  \left( V_2^{L} - V_2^{out} \right) \label{RHL-sys-2} \\
- \micoeff_L \left( \alphaGeneral^{L} - \alphaGeneral^{out} \right) &= \left( V_3^{L} - V_3^{out} \right) \label{RHL-sys-3}
\end{align}
As above, \eqref{RHL-sys-1} and \eqref{RHL-sys-2} imply \eqref{RHL-sys-3}.
For the $\TEz$ and $\TMz$ modes solving the Ranking-Hugoniot condition is equivalent to solving the system
\begin{align}
\micoeff_R \left(\alphaGeneral^{R} - \alphaGeneral^{out} \right) &= \left( V_3^{R} - V_3^{out} \right) \\
- \micoeff_L \left( \alphaGeneral^{L} - \alphaGeneral^{out} \right) &= \left( V_3^{L} - V_3^{out} \right)
\end{align}
Solving the linear system results given by \eqref{RHL-sys-3} and
\eqref{RHR-sys-3} results in
% divide by c* coeff then 1-2 again
% 1 - 2 and rearrange
\begin{equation}
\alphaGeneral^{out} = \frac{\UFieldComp_3^R - \UFieldComp_3^L - \micoeff_R \alphaGeneral^R - \micoeff_L \alphaGeneral^L}{\micoeff_L + \micoeff_R}\label{eq:interfce-bc-alphaGeneral}
\end{equation}
and
\begin{equation}
  \UFieldComp_3^{out} =
  \frac{
    \micoeffinv_R \UFieldComp_3^R + \micoeffinv_L \UFieldComp_3^L - \left(\alphaGeneral^R - \alphaGeneral^L \right)
  }{
    \micoeffinv_L + \micoeffinv_R
    } , \label{eq:interface-bc-V3}
\end{equation}
where $\micoeffinv = 1 / \micoeff$.


\subsection{Material Interfaces}
...what is a material interface for 3D...
%TODO ...

material interfaces for 3D...by solving linear system...
For the $\TEz$ mode this results in the conditions
% divide by c* coeff then 1-2 again
\begin{equation}
H_3^{out} = \frac{\speedoflightright \mu_R H_3^R + \speedoflightleft \mu_L H_3^L - \left(\alphaGeneral^R - \alphaGeneral^L \right)}{\speedoflightright \mu_R + \speedoflightleft \mu_L } \label{eq:interface-bc-V3}
\end{equation}
and
\begin{equation}
\alphaGeneral^{out} = \frac{\speedoflightright \epsilon_R \alphaGeneral^R + \speedoflightleft \epsilon_L \alphaGeneral^L - \left( H_3^R - H_3^L \right) }{\speedoflightright \epsilon_R + \speedoflightleft \epsilon_L}
\end{equation}
with
$$ \alphaGeneral = n_1 E_2 - n_2 E_1 $$
and for the $\TMz$ this results in 
% divide by c* coeff then 1-2 again
\begin{equation}
E_3^{out} = \frac{\speedoflightright \epsilon_R E_3^R + \speedoflightleft \epsilon_L E_3^L - \left(\alphaGeneral^R - \alphaGeneral^L \right)}{\speedoflightright \epsilon_R + \speedoflightleft \epsilon_L }
\end{equation}
and
\begin{equation}
\alphaGeneral^{out} = \frac{\speedoflightright \mu_R \alphaGeneral^R + \speedoflightleft \mu_L \alphaGeneral^L - \left( E_3^R - E_3^L \right) }{\speedoflightright \mu_R + \speedoflightleft \mu_L}
\end{equation}
with $$ \alphaGeneral = n_1 E_2 - n_2 E_1 $$

\subsection{Absorbing boundary condition}
...infinite
domain....truncation....finite....computation....
ABC...is...artificial.....EM radiation absorbed....

Ruben/Oubay: 'For problems posed on unbounded domains, the computational
domain is truncated and a non-reflecting boundary
condition is imposed at the truncated boundary. This is achieved
by the addition of an uniaxial perfectly matched layer (UPML)'

This is obtained by...Only flux out....
numerical flux becomes...
\begin{align}
\NumFlux(\USoltn,\Uout) = \NormalFluxPositiveEigenvalues(\USoltn) = \AnPlus \USoltn
\end{align}
% TODO - U here shouldn't have element subscript
or equivalently....there is no inward flux....
$$
\AnMinus \JumpU = - \AnMinus \USoltn
$$
% TODO I don't get this expression...
no incoming flux equiv to 1st order S-M condition

can be used with no PML, to dissipate waves as they get to boundary...etc...
Kabakian et al (2004)

\subsection{Residual Vector}
recall
$$
\Residual = \MassMatrix^{-1} \left[  \FluxDiv + \FluxNumFlux \right] - \Asource
$$

need to integrate over element interior for div term (and source term)
need to integrate over element faces for boundary terms


\subsubsection{Spatial Discretisation}
  
\subsubsection{Planar Elements}

...with planar faces
...define a linear mapping ref elem (I) -> physical element.
isoparametric mapping given by coords of the vertices of Omega
...then $\FluxNumFlux_{i}$ integral 
\begin{align*}
\FluxDiv_{i} &=  \sum_{k=1}^{\nsd} a_k \sum_{j=1}^{\nsd} \int_{\elem} \left( \dpart{\SF_i}{\xk} \cdot \dpart{\SF_j}{\xk} \delem \right) \UVect_j ,
\end{align*}
is transformed to the reference element (i.e computed in the ref elem as
\begin{align*}
\FluxDiv_{i} &= |J| \sum_{k=1}^{\nsd} a_k \sum_{j=1}^{\nsd} J_{l,k}^{-1} \sum_{l=1}^{\nsd}\int_{\elem} \left( \dpart{\SF_i}{\xk} \cdot \dpart{\SF_j}{\xk} \delem \right) \UVect_j ,
\end{align*}

or alternatively
\begin{align*}
\MassMatrixComponent_{ij} &= \int_{\refelem} \SF_i \SF_j |\Jacobian| \drefelem \\
\ConvectionMatrix^{k}_{ij} &= \int_{\refelem} \SF_i
                             \left(
                               \sum_{l=1}^{\nsd}
                               \Jacobian_{kl}^{-1}
                               \dpart{\SF_j}{\xi_{l}}
                             \right)
                             |\Jacobian|
                             \drefelem
  \\
\MassMatrixFace_{ij} &= \int_{\refelemtrace} \SF_i \SF_j |\JacobianFace|
  \drefelemtrace
\end{align*}
So far this is pretty general...all I've done is introduce the isoparametric mapping.
Note what J is (the jacobian of the mapping) and Jf is 'the jacobian of the
restriction of the isoparametric mapping to the face f' <- actually, I don't see
any face
% TODO - this is not complete....I need to know how to transform to reference
% element. Generally just do this by chain rule...simples...think about 1D first
next note here about how the jacobian |J| (of the linear mapping I -> Omega) is a constant
and can be taken out of calculation

Also note how $\J = \dpart{\xbf}{\mathbf{\xi}}$
Then the 'convection elemental matricies' $ C_{i,j}^{\xi_l} $ (i.e. the stuff in brackets) is
precomputed once only per reference element
and the whole thing is computed as
\begin{align*}
\FluxDiv_{i} &= |J| \sum_{k=1}^{\nsd} a_k \sum_{j=1}^{\nsd} J_{l,k}^{-1} \sum_{l=1}^{\nsd} C_{i,j}^{\xi_l} \UVect_j ,
\end{align*}
i.e. now we're doing the same thing without any integration

Another thing to mention now is about the choices of quadrature (i.e. summations
over gauss points)

R/O Quote:
 For quadrilateral and hexahedral elements, quadrature
based on the tensor product of well known one–dimensional Gauss–Legendre
rules is readily implemented for any order of approximation. Note, however,
that other quadrature formulae, with fewer integration points, exist [31, 32].
For triangles, specific quadrature rules, such as the symmetric quadrature
proposed in [33, 34], are used. Analogously, efficient specific quadrature
rules are used for tetrahedra, prisms and pyramids [34, 35].

The flux term (i.e. the one on faces) similar...
This time with a reference face...
mapping $\Gamma_I$ between reference face and physical face $\Gamma_e$
Then compute the integral $\FluxNumFlux_{i}$ as

$$
\FluxNumFlux_{i} =
\an^{-}
\sum_{j=1}^{\nfn}
\left( 
\int_{\elemtrace} \SF_i \SF_j
  \delemtrace
 \right)
\JumpUCoeffVectUnknownsWithIndex{j} \\
$$

$a_n^{-}$ now outside the integral - because its a constant for a given face
(the normal is constant over face)
sum is over face nodes, since for non-face nodes the thing vanishes
% TODO - why does the thingy vanish
Ruben quote
'because other shape functions vanish over the complete planar face'

so I can do the same thing
$$
\FluxNumFlux_{i} =
\an^{-}
\sum_{j=1}^{\nfn}
m_{ij}
\JumpUCoeffVectUnknownsWithIndex{j} \\
$$
note: ruben uses Gamma as the superscript for this part of num flux

with
$$
m_{ij} =
\int_{\elemtrace} \SF_i \SF_j
  \delemtrace
$$
$m_ij ==$ 'mass matrixfor the reference face'
computed only once (not once per face or anything like that....apparently!??)
'and used in the computation of each physical face'

this is all ``quadrature free'' implementation of DG methods (Atkins and Shu,
1998) -> can be done because: I can actually reference Rubens paper here
The use of hybrid meshes to improve the efficiency of a discontinuous Galerkin
method for the solution of Maxwell's equations

$A_n^-$ is constant (i.e. linear eqtns) and
using triangular or tetrahedral meshes (i.e. constant Jacobian for non-curved
elements)
---> is this is also true for Quads? Or just affine/planar quads?

\subsubsection{Curved Elements}

I stopped reading Rubens Thesis at this point and switched to the paper...

There is a lot of stuff in the thesis about curved elements...

need to justify why NEFEM is actually pretty quick if elements are
STORED...combine this with parallel....awesome...

Fekette nodal distribution:
R/O Quote: In two dimensions, a Fekete nodal distribution is adopted for the triangle
[29] and a tensor product of one dimensiona In three dimensions, the nodal distributions proposed in [30] for the tetrahedron and
in [28] for the pyramid are used. A tensor product of one dimensional Fekete
nodal distributions is used for the hexahedron and a tensor product of triangular
and one dimensional Fekete nodal distributions is used for the prism.l Fekete nodal distributions for the quadrilateral.

Show some figures of Fekette distributions in reference element?

Wiki: 'In geometry, an affine transformation, affine map[1] or an affinity (from the Latin, affinis, "connected with") is a function between affine spaces which preserves points, straight lines and planes.'

What is the definition of an affine shape??




\begin{itemize}
	\item discuss types of element - (non-)affine, planar, different shapes etc.
	\item condition number of a matrix - moving nodes around %reference!
\end{itemize}

\subsection{Errors and convergence}
\begin{itemize}
  \item expected rates of convergence for time-domain (interpolation error) and freq domain (dispersion error)
\end{itemize}

%%% Local Variables:
%%% mode: latex
%%% TeX-master: "../Thesis"
%%% End: