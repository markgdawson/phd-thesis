\chapter{Discontinuous Galerkin Method for Maxwells Equations} % Write in your own chapter title
\label{Chapter3}
\lhead{Chapter 3. \emph{Discontinuous Galerkin}} % Write in your own chapter title to set the page header

\subsection{Formulation}

The Discontinuous Galerkin method was first introduced to solve the neutron transport problem by Reed and Hill \cite{} in 1973.
** Some background/literature review here Cockburn, Shui for solving hyperbolic equations etc ***
% PhDJesusAlvarez good for motivation - not so much for the method

As seen in Chapter~\ref{PhysicalProblemChapter}, given a suitable choice of initial conditions the evolution of the system in time can be described by Maxwell's curl equations in strong form,~\eqref{eq:maxwell-curl-equations-conservation-form}. Consider that the problem is defined on a physical domain, $\Omega$, which is discretised by an unstructured mesh of $\elemtot$ nonoverlapping and body-conforming simplices, $\elem$, such that $ \Omega \approx \Omega_h = \mathop{\bigcup}_{k=1}^{K} \elem $.
% discuss the discretisation more - duplication of nodes?
Following the method of weighted residuals for a single element we multiply the strong form,~\eqref{eq:maxwell-curl-equations-conservation-form}, by a vector of test functions $\TF$ and integration over an element $\elem$,
$$
\int_{\elem} \TF \cdot \uet \delem  + \int_{\elem} \TF \cdot \dpart{\Flux_{k}}{x_{k}}= \int_{\elem} \TF \cdot \maxwellSource \delem,
$$
where $\Ue$ denotes the restriction of the solution, $\USoltn$, to the element $\elem$. After integration by parts the weak form is obtained as

\begin{equation}
\int_{\elem} \TF \cdot \uet \delem  - \int_{\elem} \dpart{\TF}{\xk} \cdot
\Flux_{k}(\Ue) \delem + \int_{\elemtrace} \TF \cdot \NormalFlux(\Ue) \delemtrace
= \int_{\elem} \TF \cdot \maxwellSource(\Ue) \delem,
\label{eq:weak-form-with-physical-flux}
\end{equation}
where $\mathbf{F_n}$, the outward normal flux, is given by $ \mathbf{F_n}(\mathbf{U}) = \outnormalcoeffk \mathbf{F}_k(\mathbf{U}) $, with $\outnormalcoeffk$ being the $\outnormalcoeffcomp$th component of the outward unit normal vector of $\elemtrace$.

Since the weak form stated in~\eqref{eq:weak-form-with-physical-flux} is specified on the element $\elem$, this does not constitute a scheme suitable for solving the global problem. In order to recover the global solution the continuity of the solution between elements is weakly enforced by replacing the physical normal flux, $\NormalFlux(\Ue)$, with a consistent numerical flux, $\NumFlux(\Ue,\Uout)$. This flux along the element trace of $\elem$ is evaluated in terms of the solution, $\mathbf{U}_e$, in the element $\elem$, and the solution in the neighbouring element, $\mathbf{U}^{out}$. The DG weak formulation is this written as
\begin{equation}
\int_{\elem} \TF \cdot \uet \delem  - \int_{\elem} \dpart{\TF}{\xk} \cdot
\Flux_{k}(\Ue) \delem + \int_{\elemtrace} \TF \cdot \NumFlux(\Ue) \delemtrace
= \int_{\elem} \TF \cdot \maxwellSource(\Ue) \delem,
\label{eq:weak-form-with-numerical-flux}
\end{equation}

% TODO: What is the definition of Uout in mathematical language
% TODO:
% Ruben/Oubay: This numerical flux is evaluated in terms of the trace of
% the solution on element Xe and the trace of the solution, Uout, on the
% other element adjacent to Ce
\subsection{Choice of Numerical Flux}
The flux splitting technique proposed in~\cite{donea2003finite} for linear hyperbolic systems is employed.
% TODO: this is a lot like Rubens/Obays paper, rewrite it. QUOTE: A natural choice, for the linear hyperbolic system of interest here, is to employ a flux splitting technique ~\cite{donea2003finite}, which corresponds to an upwind approximation \cite{chen2005high}
The physical normal flux is therefore written as
$$
\NormalFlux(\USoltn) = \An \USoltn
$$
where $\An = \outnormalcoeffk \Ak$, which can be decomposed into incoming
(superscript -) and outgoing (superscript +) fluxes
\begin{align}
\NormalFlux(\USoltn) = \NormalFluxPositiveEigenvalues(\USoltn) + \NormalFluxNegativeEigenvalues(\USoltn)
\label{eq:phys-flux-splitting}
\end{align}
where $ \NormalFluxPositiveEigenvalues = \AnPlus \USoltn$, $\NormalFluxNegativeEigenvalues = \AnMinus \USoltn, $ and the matrices $\AnMinus$ and $\AnPlus$ denote respectively the matrices of the positive and negative eigenvalues of $\An$. These can be written conveniently as
\begin{align}
  \label{eq:AnMinus-AnPlus-Definition}
\AnPlus &= \left( \An + \AnMod \right) / 2   \\
\AnMinus &= \left( \An - \AnMod \right) / 2
\end{align}
Choosing of an upwind approximation\cite{chen2005high} results in
\begin{align}
\NumFlux(\USoltn) = \NormalFluxPositiveEigenvalues(\USoltn) + \NormalFluxNegativeEigenvalues(\Uout).
\label{eq:num-flux-splitting}
\end{align}
After integration by parts,~\eqref{eq:weak-form-with-numerical-flux} can be written as
\begin{align*}
\int_{\elem} \TF \cdot \uet \delem  + \int_{\elem} \TF \cdot \dpart{\Flux_{k}(\Ue)}{\xk} \delem + \int_{\elemtrace} \TF \cdot \left[ \NumFlux(\Ue,\Uout) - \NormalFlux(\Ue) \right] \delemtrace \\
= \int_{\elem} \TF  \cdot \maxwellSource(\Ue) \delem, \label{eq:weak-form-upwind-splitting-fluxes}
% the second term changes from + to - from prev weak form
\end{align*}
We note that % TODO - I've lost some text here...!
\begin{equation}
  \NumFlux(\Ue,\Uout) - \NormalFlux(\Ue) = \AnMinus \Uout - \AnMinus \USoltn = \AnMinusU, \label{eq:AnMinusUderivation}
\end{equation}
where the jump operator has been defined as $\JumpU = \Uout - \USoltn$. By substitution into~\eqref{eq:weak-form-upwind-splitting-fluxes}, the weak form with upwind flux splitting is written as
\begin{equation}
\int_{\elem} \TF \cdot \uet \delem  + \int_{\elem} \TF \cdot \dpart{\Flux_{k}(\Ue)}{\xk} \delem + \int_{\elemtrace} \TF \cdot \AnMinusU \delemtrace = \int_{\elem} \TF  \cdot \maxwellSource(\Ue) \delem,
\label{eq:weak-form-final}
\end{equation}

% TODO: missing some stuff on diagonalisation of A here....is it necessary, also the form of the
% positive and negative eigenvalues.
% TODO: Missing the form of the numerical flux for DG!!
% *** conditions to be satisfied by numerical flux + form of numerical flux ***
% TODO: upwind flux: effect for wave domination problems - flow of information, upwind flux

\subsubsection{Internal Element Boundaries}
We consider the boundary conditions at an internal boundary between elements. We recall the form of the physical outward normal flux
$$\An = \sum_{k=1}^{\nsd} \outnormalcoeffk \Ak ,$$
where $n_k$ are the directional cosines of the outward normal vector, $\outnormalvector$, on the element boundary. We recall from~\ref{sec:conservation-form} that this expression may be written as
$$
  \An =
  \begin{pmatrix}
 & \zerom , & \mu^{-1} \RTotNorm, & \zerom \\
 & - \eps^{-1} \RTotNorm & \zerom & \zerom \\
 & \zerom & \zerom & \zerom 
 & \end{pmatrix}
$$
where,
$$
  \RTotNorm =
  \begin{pmatrix}
 & 0 & n_3 & -n_2 \\
 & -n_3 & 0 & n_1 \\
& n_2 & -n_1 & 0 
 & \end{pmatrix} .
$$
The modulus of $\An$ is given by
\begin{align*}
\AnMod = \speedoflight
\begin{pmatrix}
  \modAnSubMatrix & \zerom & \zerom \\
  \zerom  & \modAnSubMatrix & \zerom \\
   \zerom & \zerom & \zerom 
\end{pmatrix}
\end{align*}
where $\speedoflight = \left( \eps \mu  \right)^{-\frac{1}{2}}$ is the speed of light in the medium and
\begin{align*}
  \modAnSubMatrix = 
\begin{pmatrix}
\outnormalcoeff_2^2 + \outnormalcoeff_3^2 &      -\outnormalcoeff_1 \outnormalcoeff_2 &      -\outnormalcoeff_1 \outnormalcoeff_3 \\
-\outnormalcoeff_1 \outnormalcoeff_2 & \outnormalcoeff_1^2 + \outnormalcoeff_3^2 &      -\outnormalcoeff_2 \outnormalcoeff_3 \\
-\outnormalcoeff_1 \outnormalcoeff_3 &      -\outnormalcoeff_2 \outnormalcoeff_3 & \outnormalcoeff_1^2 + \outnormalcoeff_2^2 \\
\end{pmatrix} ,
\end{align*}
where the identity $\sqrt{\sum_{\outnormalcoeffcomp} \outnormalcoeffk^2} = 1$, for the unit vector $\outnormalvector$, has been used. We therefore write
\begin{align*}
\AnMinus = \speedoflight
\begin{pmatrix}
  -\modAnSubMatrix & \RTotNorm & \zerom \\
  -\RTotNorm  & -\modAnSubMatrix & \zerom \\
   \zerom & \zerom & \zerom 
\end{pmatrix} .
\end{align*}

Noting that $\RTotNorm \anyVector = \outnormalvector \times \anyVector$ and $\modAnSubMatrix \anyVector = \outnormalvector \times \left(  \outnormalvector
  \times \anyVector \right)$,for any vector $\anyVector$, results in the expression
\begin{align}
\AnMinusU = \frac{1}{2}
\begin{pmatrix}
  -\nvect \times \left( \JumpH + \sqrt{\frac{\eps}{\mu}} \nvect \times \JumpE \right) \\
   \nvect \times \left( \JumpE + \sqrt{\frac{\mu}{\eps}} \nvect \times \JumpH \right) \\
  \zerov
\end{pmatrix} .
  \label{eq:AnMinuU-expression-3D}
\end{align}
% Should I write out A_n for the TEz and TMz modes?
A similar procedure results in
\begin{align}
\AnMinusU =
  \frac{1}{2}
  \left[
    \Jump{H_3} - \sqrt{\frac{\eps}{\mu}} \Jump{\alphaGeneral}
  \right]
\begin{pmatrix}
   -\outnormalcoeff_2 \\
   \outnormalcoeff_1 \\
   - \sqrt{ \frac{\mu}{\eps} } \\
   0  \\
   0 
\end{pmatrix} . \label{eq:AnMinuU-expression-TE}
\end{align}
with $ \alphaGeneral = n_1 E_2 - n_2 E_1. $ for the $\TEz$ mode and
\begin{align}
\AnMinusU =
  \frac{1}{2}
  \left[
    \Jump{E_3} - \sqrt{\frac{\mu}{\eps}} \Jump{\alphaGeneral}
  \right]
\begin{pmatrix}
   \outnormalcoeff_2 \\
   -\outnormalcoeff_1 \\
   -\sqrt{ \frac{\eps}{\mu} } \\
   0 
\end{pmatrix} . \label{eq:AnMinuU-expression-TM}
\end{align}
for the $\TMz$ mode with $ \alphaGeneral = - n_1 H_2 + n_2 H_1. $

% VIVA: calculated with matlab script:
% modAn = sqrtm(An*An') and knowing that
% make sure I can do sqrtm by hand
\subsection{Spatial Discretisation}
The system is discretised by approximating the solution using a nodal basis
\begin{align}
\USoltn_e(\xbf,\t) \simeq \sum_{j=1}^{\nen} \SF_{j} (\xbf) \UVect_{j}(\t) ,
\label{eq:nodal-basis-defn}
\end{align}
% VIVA: the second U_j is the vector of coefficients
where $N_{j}$ are $j$ nodal Lagrangian shape functions spanning the approximation space,
$\UVect_{j}$ is the value of the solution at the $j$th node, $\nen$ is the number of nodal
points. Following the Galerkin method, the
vector of test functions, $\TF$, is chosen from the same basis as the shape
functions
\begin{align}
\TF = \sum_{i=1}^{\nen} \SF_{i}(\xbf).
\label{eq:test-function-comp-defn}
\end{align}
By substitution of~\eqref{eq:nodal-basis-defn} and~\eqref{eq:test-function-comp-defn} into~\eqref{eq:weak-form-final} we obtain the discretised weak form
\begin{align*}
\sum_{i,j} \left[
  % term 1
  \left(
    \int_{\elem} \SF_i \SF_j \delem
  \right)
  \dodet{\UVect_j}
+
  % term 2
  \left(
    \int_{\elem} \SF_i \dpart{\SF_j}{x^k} \delem
  \right)
  \Ak \UVect_j
+
  % term 3
  \left(
  \int_{\elemtrace} \SF_i \SF_j \delemtrace 
  \right)
  \AnMinus \JumpUCoeffVectUnknownsWithIndex{j}
  % term 4
-
  \left(
  \int_{\elem} \SF_i \SF_j \delem
  \right)
  \Asource 
  \right]  = 0,
\end{align*}
% NOTE - the second term actually also has an implicit sum over k in (Einstein notation)
which can be written as a system of $\nen$ ordinary differential equations,
$$
\sum_{j=1}^{\nen}
\left[
\MassMatrix \dodet{\UVect_j} +
\left( \ConvectionMatrix^{k} \Ak \right) \UVect_j -
\MassMatrix \Asource
\right]
- \sum_{\faceindex=1}^{\nfn}
\left( 
\int_{\elemtrace} \SF_i \SF_j \AnMinus \delemtrace
 \right)
\Jump{\UVect_j}
= 0
$$
% NOTE - again, second has implicit sum
% NOTE - in Ruben/Oubay paper MI is used instead of M
% TODO - A_s has not been defined....! Also am I missing a U to multiply A_s?
for each node $i$, where $\UVect_j$ is a vector of the solution coefficients at
the $j$th node, with
\begin{align*}
\MassMatrixComponent &= \int_{\elem} \SF_i \SF_j \delem \\
\ConvectionMatrixComponent^{k} &= \int_{\elem} \SF_i \dpart{\SF_j}{\xk}\delem
\end{align*}
where $\MassMatrix$ is the elemental mass matrix, $\ConvectionMatrix$ is the
convection matrix in the direction $x^{k}$, $\MassMatrixFace$ is the face
mass matrix and $\nfn$ denotes the number of face nodes. Note that a choice of
Lagrangian, nodal shape functions results in a block diagonal elemental mass
matrix. Additionally, an isoparametric mapping results in the restriction of the index $\faceindex$ to face nodes only, since
other terms are zero.
% TODO - this happens because by definition other face nodes are zero on all nodes
% except the ones not on a face...
The resulting system of equations can also be written as
$$
\dodet{\UVect} + \Residual(\UVect) = 0.
$$
where the residual vector is given by
$$
\Residual = \FluxDiv + \FluxNumFlux ,
$$
with
\begin{align}
  \FluxDiv &= \MassMatrix^{-1} \sum_{j}^{\nen} \ConvectionMatrix^{k} \Ak \UVect_j - \Asource \\
  \FluxNumFlux &= \MassMatrix_{ij}^{-1} \sum_{\faceindex}^{\nfn} \int_{\elemtrace} \SF_i \SF_j \AnMinus \delemtrace \Jump{\UVect_j} .
\end{align}
% TODO: what is all this crap about M_{ij} and how does it correspond to residual?
% TODO: how should I write residual vector now?

The solution is advanced in time with an explicit fourth order Runge-Kutta (RK4)
method. The time step is selected to be sufficiently small that the numerical
error is dominated by the error in spatial discretisation. Implicit schemes which allow larger time steps may be employed to obtain the final solution of the system of equations in a shorter computational time. However, as shown in~\autoref{Ch:SignalAnalysis}, methods to extract frequency domain information from a time domain signal have an upper frequency limit inversely proportional to time step.
% TODO: RK4 Stability condition 
% TODO - justify high order time integration....!

% TODO: R/O Quote: 'Triangles and quadrilaterals are employed to provide a consistent
%% discretisation of the spatial solution domain, X, for two
%% dimensional problems. In three dimensions, consistent meshes
%% consisting of tetrahedra, hexahedra, prisms and pyramids are used.
%% Apart from the pyramid, which requires special attention, optimal
%% nodal finite elements of arbitrary order are readily defined for all
%% these shapes. For the pyramid, a recently proposed approximation
%% space [28] is adopted. This space is well suited for both continuous
%% and discontinuous approximations and is optimal, i.e. the a priori
%% error estimate is Oðhpþ1 Þ in the L2ðXÞ norm, where p denotes the
%% order of the approximation. The approximation spaces that are employed
%% are summarised in Table 1' <----



\subsection{Residual Vector}
Computation of the residual vector requires integration over the element
interior, to compute $\FluxDiv$ and over the element trace to compute $\FluxNumFlux$.
% TODO - more comments here on the fact I'm going to use isoparametric mapping
% (and maybe refer to NEFEM etc also)

\subsubsection{Isoparametric Mapping}
% TODO - 'isoparametric mapping given by the coords of the vertices of Omega
In order to perform the integration over the element interior an isoparametric
mapping, $\IsoMapping$, is introduced from a reference element $\refelem$ to the
physical element $\elem$. The integrals in~\eqref{}, once transformed to the
reference element, become
\begin{align*}
\MassMatrixComponent_{ij} &= \int_{\refelem} \SF_i \SF_j |\Jacobian| \drefelem \\
\ConvectionMatrix^{k}_{ij} &= 
                               \sum_{l=1}^{\nsd}
                                \int_{\refelem} \SF_i
                               \Jacobian_{lk}^{-1}
                               \dpart{\SF_j}{\xi_{l}}
                             |\Jacobian|
                             \drefelem
\end{align*}
% TODO - ruben has lk not lk on Jacobian matrix
% VIVA - make sure I know what this definition of J means
where $\J = \dpart{\xbf}{\mathbf{\xi}}$ is the Jacobian of the mapping $\IsoMapping$. Similarly to compute the integrals from $\FluxNumFlux$, integration over a physical element face. An isoparametric mapping between the reference face $\refface$ and the physical face $\face$, is introduced. Following transformation to the reference face, the face mass matrix becomes

$$
 \FluxNumFlux = \MassMatrix^{-1}
 \sum_{\faceindex=1}^{\nfn}
 \left( 
 \int_{\refelemtrace}
\SF_i \SF_j \AnMinus |\JacobianFace| \drefelemtrace
 \right)
\Jump{\UVect_j}
$$
where $\JacobianFace$ is the Jacobian of the restriction of the isoparametric
mapping to the face $\face$.
% TODO - easier way of saying this

\subsubsection{Affine Elements}
% Wikipedia: 'In geometry, an affine transformation, affine map[1] or an affinity (from the Latin, affinis, "connected with") is a function between affine spaces which preserves points, straight lines and planes.'
For elements with planar faces, $\Jacobian$ and $\JacobianFace$ are constant,
and therefore the mass and convection matrices become
\begin{align*}
\MassMatrix &= |\Jacobian| \MassMatrixAffine \\
\ConvectionMatrix^{k} &= |\Jacobian|
                               \sum_{l=1}^{\nsd}
                               \Jacobian_{kl}^{-1}
                               \ConvectionMatrixAffine^{k}
\end{align*}
with
\begin{align*}
\MassMatrixComponentAffine &= \int_{\refelem} \SF_i \SF_j \drefelem \\
\ConvectionMatrixComponentAffine^{k} &= \int_{\refelem} \SF_i
                             \left(
                               \dpart{\SF_j}{\xi_{l}}
                             \right)
                             \drefelem .
\end{align*}

and $\FluxNumFlux$ becomes
$$
\FluxNumFlux = \MassMatrix^{-1}
|\JacobianFace| 
\AnMinus \Jump{\UVect_j}
 \sum_{\faceindex=1}^{\nfn}
 \MassMatrixFaceAffine
$$
with
$$
 \MassMatrixFaceComponentAffine =
 \int_{\refelemtrace}
\SF_i \SF_j \drefelemtrace,
$$
where $\AnMinus$ has been factored out of the integrand, since $\outnormalvector$, and therefore $\AnMinus$, are constant over a planar face. Note that matrices $\MassMatrixAffine^{-1}$, $\ConvectionMatrixAffine^{k}$ can be computed and stored once only for each reference element, and the matrix $\MassMatrixFaceAffine$ once for each reference face. This results in an implementation of the DG method where all numerical integration is performed \textit{a priori}, this is known as the \textit{quadrature free} implementation\cite{Atkins and Shu 1998}.
% TODO - also cite Rubens paper here: The use of hybrid meshes to improve the efficiency of a discontinuous Galerkin
% TODO - what is this whole thing with triangular/tetrahedral meshes -> always constant jacobian unless curved, Quads -> not always constant jacobian unless curved, should I elabourate?

%TODO - Another thing to mention now is about the choices of quadrature (i.e. summations over gauss points)
%%  R/O Quote:
%%   For quadrilateral and hexahedral elements, quadrature
%%  based on the tensor product of well known one–dimensional Gauss–Legendre
%%  rules is readily implemented for any order of approximation. Note, however,
%%  that other quadrature formulae, with fewer integration points, exist [31, 32].
%%  For triangles, specific quadrature rules, such as the symmetric quadrature
%%  proposed in [33, 34], are used. Analogously, efficient specific quadrature
%%  rules are used for tetrahedra, prisms and pyramids [34, 35].


\subsubsection{Curved Elements}
For isoparametric curved elements, since $\Jacobian$ and $\JacobianFace$ are not constant, the computation of the matrices $\MassMatrix$, $\ConvectionMatrix^{k}$ and $\FluxNumFlux$ require a separate numerical integration for each element. In many applications the number of curved elements is small, for example to capture a curved boundary. In such cases the matrices are computed and stored \textit{a priori}, once for each curved element.

%  * isoparametric only capture *roughly* the geometry....
%  * curved elements used both for higher accuracy or for curved boundaries...
%     -> can use planar high order elements...dont confuse the two
%  * p-extension of FEM reference (Ruben has Szabo and Babuska,1991)
%  
%  planar mesh -> 'poly order of the approximation is increased' -> to get to the desired error
%  
%  This is great...but in some cases...
%  but... geometric accuracy 'deteriorates the solution'...
%  isoparametric causes this...i.e the nodes are correct but between them is interpolation...
%  LOADS OF REFERENCES HERE (Sevilla stuff)
% Fekette + matrix condition number
%  higher geom accuracy
%  There is a lot of stuff in Rubens thesis about curved elements...what is he on about??
% VIVA: parallel is also cool for storage...possibly...


% TODO: Fekette nodal distribution: Show some figures of Fekette distributions in reference element?
%% R/O Quote: In two dimensions, a Fekete nodal distribution is adopted for the triangle
%% [29] and a tensor product of one dimensiona In three dimensions, the nodal distributions proposed in [30] for the tetrahedron and
%% in [28] for the pyramid are used. A tensor product of one dimensional Fekete
%% nodal distributions is used for the hexahedron and a tensor product of triangular
%% and one dimensional Fekete nodal distributions is used for the prism.l Fekete nodal distributions for the quadrilateral.

\subsection{Jump Conditions}
% TODO - references -> LeVeque/Donea + Huerta(2005)
For interfaces which intersect the domain boundary, $\partial \Omega$, the not all components of $\mathbf{U}^{out}$ are determined by the boundary conditions on the interface. For a system of conservation laws Rankine-Hugoniot jump conditions of the form
\begin{align}
\Jump{ \mathbf{F}_n } = \lambda_j \Jump{ \mathbf{U} } \label{eq:rankine-hugoniot}
\end{align}
where $\lambda_j$ are the eigenvalues of the jacobian matrix $\mathbf{A}_n$.
This condition should be satisfied along the characteristics in the phase plane.
For the 3 dimensions these are $ \lambda_{ 1,2 } = - \speedoflightleft $, $
\lambda_{ 3,4 } = \speedoflightright $ and $\lambda_{5..9} = 0 $, where the
$\speedoflightleft$ and $\speedoflightright$ are the velocities of the
electromagnetic wave in media on the left and right side of the interface
respectively. This condition should be satisfied along the phase plane
characteristics, as shown in \ref{fig:phase-plane-characteristics}.

\begin{figure}[h]
  \centering
  
  \caption{Phase plane diagram showing the characteristics for Maxwells' equations}
  \label{fig:phase-plane-characteristics}
\end{figure}

\subsection{Rankine-Hugoniot Jump Conditions in 3D}
For the three dimensional case the normal flux can be written as
\begin{align*}
\NormalFlux =
\begin{pmatrix}
- \outnormalvector \times \H \\
\outnormalvector \times \E
\end{pmatrix}
\end{align*}
Applying the condition~\eqref{eq:rankine-hugoniot} and solving the resulting linear system results in
\begin{align}
  \mu^{out} \H^{out} - \mu_L \H^{L} = - \frac{1}{\speedoflightleft} \outnormalvector \times \left( \E^{out} - \E^{L} \right)
  \label{eq:jump-condition-resulting-equation-system-3D-1} \\
  \mu^{R} \H^{R} - \mu_{out} \H^{out} = \frac{1}{\speedoflightleft} \outnormalvector \times \left( \E^{R} - \E^{*} \right)
  \label{eq:jump-condition-resulting-equation-system-3D-2} \\
\end{align}
% TODO - I got this from Mar thesis, have not verified it
\subsection{Rankine-Hugoniot Jump Conditions in 2D}
For the $\TEz$ and $\TMz$ modes, by setting..., the normal physical flux can be
written in the form
$$
\NormalFlux = 
\begin{pmatrix}
  - n_2 \UFieldComp_3 \\
  n_1 \UFieldComp_3 \\
  \alphaGeneral
\end{pmatrix}
$$
where $ \alphaGeneral = n_1 \UFieldComp_2 - n_2 \UFieldComp_1 $, and the vector $\UField$ is given by $
\UField = 
  \begin{pmatrix} 
    E_1 \; E_2 \; H_3
  \end{pmatrix}^T
$ for the $\TEz$ mode, and $
\UField = 
  -
  \begin{pmatrix} 
    H_1 \; H_2 \; E_3
  \end{pmatrix}^T
  $ for the $\TMz$ mode.
For dispersive media this will be expanded with zeros...
% TODO - CHECK OUT THE A_n vectors and check that these F_n come nicely from there...
% TODO - should I have a numbered lambda here? and would that be different for
% TE and TM modes
By applying the Rankine-Hugoniot condition along $\dpart{x}{t} = 0$, with $ \Jump{ \mathbf{F}_n } = 0 $, the normal flux in region $\LStar$
and $\RStar$ are equal, we denote the flux in these region as
$$
\mathbf{F}_n^{out} = 
\begin{pmatrix}
  - n_2 \UFieldComp_3^{out} \\
  n_1 \UFieldComp_3^{out} \\
  \alphaGeneral^{out}
\end{pmatrix} .
$$

Along $\dpart{x}{t} = \speedoflightright$, the Rankine-Hugoniot condition takes the form
$$
\Jump{ \mathbf{F}_n } = \speedoflightright \Jump{ \mathbf{U} }
$$
which results in
\begin{align}
-n_2 \left(V_3^{R} - V_3^{out} \right) &= \micoeff_R \left( V_1^{R} - V_1^{out} \right) \label{RHR-sys-1} \\
n_1 \left(V_3^{R} - V_3^{out} \right) &= \micoeff_R  \left( V_2^{R} - V_2^{out} \right) \label{RHR-sys-2} \\
\micoeff_R \left(\alphaGeneral^{R} - \alphaGeneral^{out} \right) &= \left( V_3^{R} - V_3^{out} \right) \label{RHR-sys-3}
\end{align}
where $\micoeff = \sqrt{\eps/\mu}$ for $\TEz$ mode and $\micoeff =
\sqrt{\mu/\eps}$ for $\TMz$ mode. It can be easily verified that
\eqref{RHR-sys-1} and~\eqref{RHR-sys-2} imply~\eqref{RHR-sys-3}.

Along the $\dpart{x}{t} = - \speedoflightleft$, the Rankine-Hugoniot condition
takes the form
$$
\Jump{ \mathbf{F}_n } = - \speedoflightleft \Jump{ \mathbf{U} }
$$
which results in
\begin{align}
-n_2 \left(V_3^{L} - V_3^{out} \right) &= - \micoeff_L \left( V_1^{L} - V_1^{out} \right) \label{RHL-sys-1} \\
n_1 \left(V_3^{L} - V_3^{out} \right) &= - \micoeff_L  \left( V_2^{L} - V_2^{out} \right) \label{RHL-sys-2} \\
- \micoeff_L \left( \alphaGeneral^{L} - \alphaGeneral^{out} \right) &= \left( V_3^{L} - V_3^{out} \right) \label{RHL-sys-3}
\end{align}
As above, \eqref{RHL-sys-1} and \eqref{RHL-sys-2} imply \eqref{RHL-sys-3}.
For the $\TEz$ and $\TMz$ modes solving the Ranking-Hugoniot condition is equivalent to solving the system
\begin{align}
\micoeff_R \left(\alphaGeneral^{R} - \alphaGeneral^{out} \right) &= \left( V_3^{R} - V_3^{out} \right) \\
- \micoeff_L \left( \alphaGeneral^{L} - \alphaGeneral^{out} \right) &= \left( V_3^{L} - V_3^{out} \right)
\end{align}
Solving the linear system results given by \eqref{RHL-sys-3} and
\eqref{RHR-sys-3} results in
% divide by c* coeff then 1-2 again
% 1 - 2 and rearrange
\begin{equation}
\alphaGeneral^{out} = \frac{\UFieldComp_3^R - \UFieldComp_3^L - \micoeff_R \alphaGeneral^R - \micoeff_L \alphaGeneral^L}{\micoeff_L + \micoeff_R}\label{eq:interfce-bc-alphaGeneral}
\end{equation}
and
\begin{equation}
  \UFieldComp_3^{out} =
  \frac{
    \micoeffinv_R \UFieldComp_3^R + \micoeffinv_L \UFieldComp_3^L - \left(\alphaGeneral^R - \alphaGeneral^L \right)
  }{
    \micoeffinv_L + \micoeffinv_R
    } , \label{eq:interface-bc-V3}
\end{equation}
where $\micoeffinv = 1 / \micoeff$.

\subsection{Material Interfaces}
The expressions for the jump of the solution at a material interface are given by $\Jump{\E} = \E^{out} - \E^{L}$ and $\Jump{\H} = \H^{out} - \H^{L}$ for the left element and $\Jump{\E} = \E^{out} - \E^{R}$ and $\Jump{\H} = \H^{out} - \H^{R}$ for the right element.
% TODO - do I need a jump J? as well
Solving the linear system given by~\eqref{eq:jump-condition-resulting-equation-system-3D-2} and~\eqref{eq:jump-condition-resulting-equation-system-3D-2} results in
\begin{align}
 \outnormalvector \times \E^{out} = \outnormalvector \times \frac{
  \left( 
\speedoflightleft \eps_L \E^{L} - \outnormalvector \times \H^{L}
 \right)
  +
  \left( 
\speedoflightright \eps_R \E^{R} + \outnormalvector \times \H^{R}
 \right)
}{
  \speedoflightright \eps_R + \speedoflightleft \eps_L
} \label{eq:material-interfaces-1} \\
 \outnormalvector \times \H^{out} = \outnormalvector \times \frac{
  \left( 
\speedoflightleft \eps_L \H^{L} + \outnormalvector \times \E^{L}
 \right)
  +
  \left( 
\speedoflightright \eps_R \H^{R} - \outnormalvector \times \E^{R}
 \right)
}{
  \speedoflightright \eps_R + \speedoflightleft \eps_L
} \label{eq:material-interfaces-1}
\end{align}
The resulting expression for $\outnormalvector \times \Jump{E}$ and $\outnormalvector \times \Jump{H}$ are substituted into~\eqref{eq:AnMinuU-expression-3D} in order obtain an expression for numerical flux on the interface.
For the $\TEz$ mode solving the system of equations given by \eqref{eq:interfce-bc-alphaGeneral} and~\eqref{eq:interface-bc-V3} results in the conditions
% divide by c* coeff then 1-2 again
\begin{equation}
H_3^{out} = \frac{\speedoflightright \mu_R H_3^R + \speedoflightleft \mu_L H_3^L - \left(\alphaGeneral^R - \alphaGeneral^L \right)}{\speedoflightright \mu_R + \speedoflightleft \mu_L } \label{eq:interface-bc-H3-TE}
\end{equation}
and
\begin{equation}
\alphaGeneral^{out} = \frac{\speedoflightright \eps_R \alphaGeneral^R + \speedoflightleft \eps_L \alphaGeneral^L - \left( H_3^R - H_3^L \right) }{\speedoflightright \eps_R + \speedoflightleft \eps_L} \label{eq:interface-bc-alpha-TE}
\end{equation}
with
$$ \alphaGeneral = n_1 E_2 - n_2 E_1. $$
An expression for the numerical flux is obtained by substitution of the expressions for $\Jump{H_3}$ and $\Jump{\alphaGeneral}$ resulting from~\eqref{eq:interface-bc-H3-TE} and ~\eqref{eq:interface-bc-alpha-TE} into~\eqref{eq:AnMinuU-expression-TE}
Similarly for the $\TMz$ this results in 
% divide by c* coeff then 1-2 again
\begin{equation}
E_3^{out} = \frac{\speedoflightright \eps_R E_3^R + \speedoflightleft \eps_L E_3^L - \left(\alphaGeneral^R - \alphaGeneral^L \right)}{\speedoflightright \eps_R + \speedoflightleft \eps_L }
\end{equation}
and
\begin{equation}
\alphaGeneral^{out} = \frac{\speedoflightright \mu_R \alphaGeneral^R + \speedoflightleft \mu_L \alphaGeneral^L - \left( E_3^R - E_3^L \right) }{\speedoflightright \mu_R + \speedoflightleft \mu_L}
\end{equation}
with
$$ \alphaGeneral = - n_1 H_2 + n_2 H_1. $$

Again expressions for $\Jump{E_3}$ and $\Jump{\alphaGeneral}$ resulting from~\eqref{eq:interface-bc-H3-TE} and ~\eqref{eq:interface-bc-alpha-TE} are substituted into~\eqref{eq:AnMinuU-expression-TM} to obtain the numerical flux.
% TODO - do I need a jump J? as well



\subsection{Absorbing boundary condition}
Many problems are posed on infinite domains, however this presents computational difficulties. In practice computation is done on a truncated domain with boundary conditions set in such a way as to approximate the infinite domain. This is done by introducing an artificial outer boundary condition which absorbs incident radiation known as an absorbing boundary condition (ABC)\cite[].
% TODO - what about PML
This is achieved by a modified numerical flux which contains outgoing flux only,
\begin{align}
\NumFlux(\Ue,\Uout) = \NormalFluxPositiveEigenvalues(\USoltn) = \AnPlus \USoltn
\end{align}
% TODO - U here shouldn't have element subscript
in which case~\eqref{eq:AnMinusUderivation} becomes
\begin{align}
  \NumFlux(\Ue,\Uout) - \NormalFlux(\Ue) = \AnPlus \Uout - \NormalFlux(\Ue) = - \AnMinus \Ue ,
\end{align}
or equivalently
$$
\AnMinus \JumpU = - \AnMinus \USoltn,
$$
which corresponds to a first order approximation of the \SilverMuller condition.
% TODO - This can be used without need for a PML, to dissipate waves as they get to boundary.
In practice this is often used in conjunction with a coarsening of the mesh around the truncated boundary to further dissipate outgoing waves\cite{Hall2004140}.

\subsection{PML}
....possibly haven't actually used this yet...!

\subsection{Errors and convergence}
\begin{itemize}
  \item expected rates of convergence for time-domain (interpolation error) and freq domain (dispersion error)
\end{itemize}

%%% Local Variables:
%%% mode: latex
%%% TeX-master: "../Thesis"
%%% End: