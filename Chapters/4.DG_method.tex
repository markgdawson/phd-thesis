\chapter{Discontinuous Galerkin Method for Maxwells Equations} % Write in your own chapter title
\label{Chapter3}
\lhead{Chapter 3. \emph{Discontinuous Galerkin}} % Write in your own chapter title to set the page header

\subsection{Formulation}

The Discontinuous Galerkin method was first introduced to solve the neutron transport problem by Reed and Hill \cite{} in 1973.
** Some background/literature review here Cockburn, Shui for solving hyperbolic equations etc ***
% PhDJesusAlvarez good for motivation - not so much for the method

As seen in Chapter~\ref{PhysicalProblemChapter}, given a suitable choice of initial conditions the evolution of the system in time can be described by Maxwell's curl equations in strong form,~\eqref{eq:maxwell-curl-equations-conservation-form}. Consider that the problem is defined on a physical domain, $\Omega$, which is discretised by an unstructured mesh of $\elemtot$ nonoverlapping and body-conforming simplices, $\elem$, such that $ \Omega \approx \Omega_h = \mathop{\bigcup}_{k=1}^{K} \elem $.
% discuss the discretisation more - duplication of nodes?
Following the method of weighted residuals on a single element we multiply the strong form,~\eqref{strong-form-DG}, by a vector of test functions $\TF$ and integration over an element $\elem$,
$$
\int_{\elem} \TF \cdot \uet \delem  + \int_{\elem} \TF \cdot \dpart{\Flux_{k}}{x_{k}}= \int_{\elem} \TF \cdot \maxwellSource \delem,
$$
where $\Ue$ denotes the restriction of the solution, $\USoltn$, to the element $\elem$. After integration by parts the weak form is obtained as

\begin{equation}
\int_{\elem} \TF \cdot \uet \delem  - \int_{\elem} \dpart{\TF}{x_k} \cdot
\Flux_{k}(\Ue) \delem + \int_{\elemtrace} \TF \cdot \NormalFlux(\Ue) \delemtrace
= \int_{\elem} \TF \cdot \maxwellSource(\Ue) \delem,
\label{eq:weak-form-with-physical-flux}
\end{equation}
where $\mathbf{F_n}$, the outward normal flux, is given by $ \mathbf{F_n}(\mathbf{U}) = \outnormalcoeffk \mathbf{F}_k(\mathbf{U}) $, with $\outnormalcoeffk$ being the $\outnormalcoeffcomp$th component of the outward unit normal vector of $\elemtrace$.

Since the weak form stated in~\eqref{eq:weak-form-with-physical-flux} is specified on the element $\elem$, this does not constitute a scheme suitable for solving the global problem. In order to recover the global solution the continuity of the solution between elements is weakly enforced by replacing the physical normal flux, $\NormalFlux(\Ue)$, with a consistent numerical flux, $\NumFlux(\Ue,\Uout)$. This flux along the element trace of $\elem$ is evaluated in terms of the solution, $\mathbf{U}_e$, in the element $\elem$, and the solution in the neighbouring element, $\mathbf{U}^{out}$. The DG weak formulation is this written as
\begin{equation}
\int_{\elem} \TF \cdot \uet \delem  - \int_{\elem} \dpart{\TF}{x_k} \cdot
\Flux_{k}(\Ue) \delem + \int_{\elemtrace} \TF \cdot \NumFlux(\Ue) \delemtrace
= \int_{\elem} \TF \cdot \maxwellSource(\Ue) \delem,
\label{eq:weak-form-with-numerical-flux}
\end{equation}

% TODO: What is the definition of Uout in mathematical language
\subsection{Choice of Numerical Flux}
Flux splitting technique...Donna and Huerta....this is the change
We write the physical normal flux as
$$
\NormalFlux(\USoltn) = \An \USoltn
$$
where $\An = \outnormalcoeffk \Ak$, which can be decomposed into incoming and outgoing fluxes
\begin{align}
\NormalFlux(\USoltn) = \NormalFluxPositiveEigenvalues(\USoltn) + \NormalFluxNegativeEigenvalues(\USoltn)
\label{eq:phys-flux-splitting}
\end{align}
where $ \NormalFluxPositiveEigenvalues = \AnPlus \USoltn$, $\NormalFluxNegativeEigenvalues = \AnMinus \USoltn, $ and the matrices $\AnMinus$ and $\AnPlus$ denote respectively the matrices of the positive and negative eigenvalues of $\An$. These can be written conveniently as
\begin{align}
  \label{eq:AnMinus-AnPlus-Definition}
\AnPlus &= \left( \An + \AnMod \right) / 2   \\
\AnMinus &= \left( \An - \AnMod \right) / 2
\end{align}
A choise of an upwind numerical flux [***] corresponds to
% TODO: citation
\begin{align}
\NumFlux(\USoltn) = \NormalFluxPositiveEigenvalues(\USoltn) + \NormalFluxNegativeEigenvalues(\Uout).
\label{eq:num-flux-splitting}
\end{align}
After integration by parts,~\eqref{eq:weak-form-with-numerical-flux} can be written as
\begin{align*}
\int_{\elem} \TF \cdot \uet \delem  + \int_{\elem} \dpart{\TF}{x_k} \cdot
\Flux_{k}(\Ue) \delem + \int_{\elemtrace} \TF \cdot \left[ \NumFlux(\Ue,\Uout) - \NormalFlux(\Ue) \right] \delemtrace \\
= \int_{\elem} \TF  \cdot \maxwellSource(\Ue) \delem,
% the second term changes from + to - from prev weak form
\label{eq:weak-form-upwind-splitting-fluxes}
\end{align*}
By substitution of~\eqref{eq:num-flux-splitting} and~\eqref{eq:phys-flux-splitting} into~\eqref{eq:AnMinusU-defn} we note that
\begin{align*}
  \NumFlux(\Ue,\Uout) - \NormalFlux(\Ue) = \AnMinus \Uout - \AnMinus \USoltn = \AnMinusU,
\end{align*}
where the jump operator has been defined as $\JumpU = \Uout - \USoltn$. By substitution into~\eqref{eq:weak-form-upwind-splitting-fluxes}, the weak form with upwind flux splitting is written as
\begin{equation}
\int_{\elem} \TF \cdot \uet \delem  + \int_{\elem} \dpart{\TF}{x_k} \cdot
\Flux_{k}(\Ue) \delem + \int_{\elemtrace} \TF \cdot \AnMinusU \delemtrace
= \int_{\elem} \TF \cdot \maxwellSource(\Ue) \delem,
\label{eq:weak-form-with-physical-flux}
\end{equation}
% TODO: missing some stuff on diagonalisation of A here....is it necessary, also the form of the
% positive and negative eigenvalues.
% TODO: Missing the form of the numerical flux for DG!!
% *** conditions to be satisfied by numerical flux + form of numerical flux ***
% TODO: upwind flux: effect for wave domination problems - flow of information, upwind flux

\subsubsection{Internal Element Boundaries}
We consider the boundary conditions at an internal boundary between elements. We recall the form of the physical outward normal flux
$$\An = \sum_{k=1}^{\nsd} \outnormalcoeffk \Ak ,$$
where $n_k$ are the directional cosines of the outward normal vector, $\outnormalvector$, on the element boundary. We recall from~\ref{sec:conservation-form} that this expression may be written as
$$
  \An =
  \begin{pmatrix}
 & \zerom , & \mu^{-1} \RTotNorm, & \zerom \\
 & - \eps^{-1} \RTotNorm & \zerom & \zerom \\
 & \zerom & \zerom & \zerom 
 & \end{pmatrix}
$$
where,
$$
  \RTotNorm =
  \begin{pmatrix}
 & 0 & n_3 & -n_2 \\
 & -n_3 & 0 & n_1 \\
& n_2 & -n_1 & 0 
 & \end{pmatrix} .
$$
The modulus of $\An$ is given by
\begin{align*}
\AnMod = \speedoflight
\begin{pmatrix}
  \modAnSubMatrix & \zerom & \zerom \\
  \zerom  & \modAnSubMatrix & \zerom \\
   \zerom & \zerom & \zerom 
\end{pmatrix}
\end{align*}
where $\speedoflight = \left( \epsilon \mu  \right)^{-\frac{1}{2}}$ is the speed of light in the medium and
\begin{align*}
  \modAnSubMatrix = 
\begin{pmatrix}
\outnormalcoeff_2^2 + \outnormalcoeff_3^2 &      -\outnormalcoeff_1 \outnormalcoeff_2 &      -\outnormalcoeff_1 \outnormalcoeff_3 \\
-\outnormalcoeff_1 \outnormalcoeff_2 & \outnormalcoeff_1^2 + \outnormalcoeff_3^2 &      -\outnormalcoeff_2 \outnormalcoeff_3 \\
-\outnormalcoeff_1 \outnormalcoeff_3 &      -\outnormalcoeff_2 \outnormalcoeff_3 & \outnormalcoeff_1^2 + \outnormalcoeff_2^2 \\
\end{pmatrix} .
\end{align*}
Note that the identity $\sqrt{\sum_{\outnormalcoeffcomp} \outnormalcoeffk^2} = 1$, for the unit vector $\outnormalvector$, has been used to simplify this expression. We therefore write
\begin{align*}
\AnMinus = \speedoflight
\begin{pmatrix}
  -\modAnSubMatrix & \RTotNorm & \zerom \\
  -\RTotNorm  & -\modAnSubMatrix & \zerom \\
   \zerom & \zerom & \zerom 
\end{pmatrix} .
\end{align*}

By noting that for any vector $\RTotNorm \anyVector = \outnormalvector \times \anyVector$ and $\modAnSubMatrix \anyVector = \outnormalvector \times \left(  \outnormalvector
  \times \anyVector \right)$,for any vector $\anyVector$, we note that
\begin{align*}
\AnMinusU = \frac{1}{2}
\begin{pmatrix}
  -\nvect \times \left( \JumpH + \sqrt{\frac{\epsilon}{\mu}} \nvect \times \JumpE \right) \\
   \nvect \times \left( \JumpE + \sqrt{\frac{\mu}{\epsilon}} \nvect \times \JumpH \right) \\
  \zerov
\end{pmatrix} .
\end{align*}
Similarily, for the $\TEz$ mode we obtain
\begin{align*}
\AnMinusU =
  \frac{1}{2}
  \left[
    \Jump{H_3} + \sqrt{\frac{\epsilon}{\mu}}
    \left(
      \outnormalcoeff_1 \Jump{E_2} - 
      \outnormalcoeff_2 \Jump{E_1}
    \right)
  \right]
\begin{pmatrix}
   -\outnormalcoeff_2 \\
   \outnormalcoeff_1 \\
   - \sqrt{ \frac{\mu}{\epsilon} } \\
   0  \\
   0 
\end{pmatrix} .
\end{align*}
and for the $\TMz$ mode
\begin{align*}
\AnMinusU =
  \frac{1}{2}
  \left[
    \Jump{E_3} + \sqrt{\frac{\mu}{\epsilon}}
    \left(
      \outnormalcoeff_2 \Jump{H_1} - 
      \outnormalcoeff_1 \Jump{H_2}
    \right)
  \right]
\begin{pmatrix}
   \outnormalcoeff_2 \\
   -\outnormalcoeff_1 \\
   -\sqrt{ \frac{\epsilon}{\mu} } \\
   0 
\end{pmatrix} .
\end{align*}

% VIVA: calculated with matlab script:
% modAn = sqrtm(An*An') and knowing that
% make sure I can do sqrtm by hand
\subsection{Spatial Discretisation}
The system is discretised by approximating the solution using a nodal basis as
\begin{align}
U^{k}_e(\xbf,\t) \simeq \sum_{i=1}^{\nbasis} \SF_{i} (\xbf) \ucoeff_{i}(\t) ,
\label{eq:nodal-basis-defn}
\end{align}
where $N_{i}$ are Lagrangian shape functions, $\nbasis$ is the number of nodal points, $u_{i}$ is the nodal solution at the $i$th node and $U^{k}_e$ denotes the $k$th component of the vector, $\Ue$. Following the Galerkin method, the vector of test functions, $\TF$, is chosen to be the same basis as the shape functions, with components given by
\begin{align}
\TFComp^{k} = \sum_{i=1}^{n} \SF_{i}(\xbf).
\label{eq:test-function-comp-defn}
\end{align}
By substitution of~\eqref{eq:nodal-basis-defn} and~\eqref{eq:test-function-comp-defn} into~\eqref{eq:weak-form-final} we obtain the discretised weak form
\begin{align*}
\sum_{i,j} \left[  \dodet{\ucoeff_i} \int_{\elem} \SF_i \SF_j \delem   +
\ucoeff_i \Ak \int_{\elem} \dode{\SF_i}{x_k}
  \dode{\SF_j}{x_k} \delem  +
\AnMinus \JumpUCoeff \int_{\elemtrace} \SF_i \SF_j
  \delemtrace 
-
\Asource \int_{\elem} \SF_i \SF_j \delem \right]  = 0,
\end{align*}
% TODO - more detail here - also is this correct?
which can be written as a system of ordinary differential equations

$$
\MassMatrix \dodet{\UVect} + \FluxDiv + \FluxNumFlux - \Asource \MassMatrix = 0
$$
% TODO - are A_s and M in wrong order? order matters heres. Same for A_k. Also matters below.
where $\UVect$ is a vector of the solution coefficients $u_{i}$. The block diagonal mass matrix, $\MassMatrix$, and the divergence and flux vectors, $\FluxDiv$ and $\FluxNumFlux$, are then given by
\begin{align*}
\MassMatrixComponent_{ij} &= \int_{\elem} \SF_i \SF_j \delem \\
\FluxDiv_{i} &= \ucoeff_i \Ak \int_{\elem} \dode{\SF_i}{x_k}
  \dode{\SF_j}{x_k} \delem  \\
\FluxNumFlux_{i} &= \AnMinus \JumpUCoeff \int_{\elemtrace} \SF_i \SF_j
  \delemtrace  \\
\end{align*}
The resulting system of equations can be written as
$$
\dodet{\UVect} + \Residual(\UVect) = 0,
$$
where $\Residual = \MassMatrix^{-1} \left[  \FluxDiv + \FluxNumFlux \right] - \Asource$, is the residual vector.

The solution is advanced in time with an explicit fourth order Runge-Kutta (RK4) method. Implicit schemes which allow larger time steps may be employed to obtain the final solution of the system of equations in a shorter computational time. However as we will be seen shortly, in~\autoref{Ch:SignalAnalysis}, the highest frequency which can be resolved is inversely proportional by time step length, in which case a short time step may be desirable.
% TODO - justify high order time integration....!
% TODO: RK4 Stability condition 
% TODO: Also RK4 method
% TODO: Local matrix + broken space
\subsection{Jump Conditions}
\begin{itemize}
	\item material interfaces (PEC,ABC)
  \item will need to mention PML -> another section
\end{itemize}

For interfaces which intersect the domain boundary, $\partial \Omega$, the not all components of $\mathbf{U}^{out}$ are determined by the boundary conditions on the interface. For a system of conservation laws Rankine-Hugoniot jump conditions of the form

$$
\Jump{ \mathbf{F}_n } = \lambda_j \Jump{ \mathbf{U} }
$$

where $\lambda_j$ are the eigenvalues of the jacobian matrix $\mathbf{A}_n$. This condition should be satisfied along the characteristics in the phase plane. For the 3 dimensions these are $ \lambda_{ 1,2 } = - \speedoflightleft $, $ \lambda_{ 3,4 } = \speedoflightright $ and $\lambda_{5..9} = 0 $, where the $\speedoflightleft$ and $\speedoflightright$ are the velocities of the electromagnetic wave in media on the left and right side of the interface respectively.

\subsubsection{Spatial Discretisation}
\begin{itemize}
	\item discuss types of element - (non-)affine, planar, different shapes etc.
	\item condition number of a matrix - moving nodes around %reference!
\end{itemize}

\subsection{Errors and convergence}
\begin{itemize}
  \item expected rates of convergence for time-domain (interpolation error) and freq domain (dispersion error)
\end{itemize}

%%% Local Variables:
%%% mode: latex
%%% TeX-master: "../Thesis"
%%% End:
