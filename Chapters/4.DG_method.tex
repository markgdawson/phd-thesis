\chapter{Discontinuous Galerkin for Maxwells Equations} % Write in your own chapter title
\label{Chapter3}
\lhead{Chapter 3. \emph{Discontinuous Galerkin}} % Write in your own chapter title to set the page header

\subsection{Formulation}

The Discontinuous Galerkin method was first introduced to solve the neutron transport problem by Reed and Hill \cite{} in 1973.

** Some background/literature review here Cockburn, Shui for solving hyperbolic equations etc ***
% PhDJesusAlvarez good for motivation - not so much for the method

As seen in Chapter \ref{PhysicalProblemChapter}, given a suitable choice of initial conditions the evolution of the system in time can be described by Maxwell's curl equations \eqref{maxwell-curl-equations-conservation-form}, given again for convenience

$$
\ut + \fk  = \mathbf{S(U)}
\label{strong-form-DG}
$$

We assume that the domain on which we wish to solve the equations, $\Omega$ can be discretised by an unstructured mesh of $K$ element such that

\begin{equation}
  \Omega \approx \Omega_h = \mathop{\bigcup}_{k=1}^{K} \Omega_e^k
\end{equation}
where $\Omega_e^k$ are the elements in the discretisation.  *** discuss the discretisation more - discontinuous elements etc ***

Following the method of weighted residuals the \eqref{strong-form-DG} is multiplied by a vector of test functions $\mathbf{W}$ and integrated over an element $\Omega_e^k$. Following integration by parts the following weak form is obtained

$$
\int_{\Omega_e^k} \mathbf{W} \ut d\Omega_e^k  - \int_{\Omega_e^k} \frac{\partial \mathbf{W}}{ \partial x_k} \mathbf{F}_k(\mathbf{U}) d\Omega + \int_{\partial \Omega_e^k} \mathbf{W} \cdot \mathbf{F_n}(\mathbf{U_e}) d\Gamma = \int_{\Omega_e^k} \mathbf{W} \cdot \mathbf{S}(\mathbf{U_e}) d\Omega
\label{maxwell-DG-weak-form}
$$

where $n_k$ is the $k$th component of the element outward normal to $\partial \Omega_e^k$ and $\mathbf{F_n}$, the normal flux, is given by

$$
\mathbf{F_n}(\mathbf{U}) = \mathbf{F}_k(\mathbf{U}) n_k
$$

This weak form is specified on an element, however this is not yet a scheme suitable for solving the global problem. In order to recover the global solution the continuity of the solution between elements is weakly enforced by replacing the physical normal flux $\mathbf{F_n}(\mathbf{U_e})$ with a consistent numerical flux $\mathbf{\tilde{F}}_n(\mathbf{U}_e,\mathbf{U}_e^{out})$, evaluated in terms of the solution on an element, $\mathbf{U}_e$, and the is the value of the solution along a given face in adjoining element sharing that face, $\mathbf{U}^{out}$.

*** conditions to be satisfied by numerical flux + form of numerical flux ***

The system is discretised by choosing the solution approximated by
$$
U_e \simeq \sum_{i=1}^{n} u_{i} N_{i}
$$
where $N_{i}$ are Lagrangian shape functions and $u_{i}$ are nodal solution values. Following the Galerkin method test functions $W$ are then chosen with the same basis of shape functions:

$$
W = \sum_{i=1}^{n} N_{i}
$$

The resulting discretised system of equations can be written as a system of ordinary differentail equations

$$
\mathbf{M} \frac{d \mathbf{U}} {dt} + \mathbf{R}(\mathbf{U}) = 0
$$

where $\mathbf{U}$ is a vector of the coefficients $u_{i}$, $M$ is the mass matrix which is block diagonal and $\mathbf{R}$ is the residual vector.

\subsubsection{Local Element Equations}
\begin{itemize}
  \item problem in conservation form -> GMWR -> weak form
  \item discretised versions of relevant maxwells equations
	\item local matrix + broken space
	\item Nodal (or Modal) representation
\end{itemize}
\subsubsection{Numerical Flux}

\begin{itemize}
	\item recover solution with numerical flux
  \item numerical flux used
	\item upwind flux: effect for wave domination problems - flow of information, upwind flux
\end{itemize}

\subsection{Time Integration}
\begin{itemize}
	\item explicit RK4
	\item breif mention explicit justification in terms of delta t limit (and refer to chapter)
	\item justify high-order time integration
\end{itemize}

\subsection{Implementing Boundary Conditions}
\begin{itemize}
	\item material interfaces (PEC,ABC)
  \item will need to mention PML -> another section
\end{itemize}
\subsubsection{Spatial Discretisation}
\begin{itemize}
	\item discuss types of element - (non-)affine, planar, different shapes etc.
	\item condition number of a matrix - moving nodes around %reference!
\end{itemize}
\subsection{Errors and convergence}
\begin{itemize}
  \item expected rates of convergence for time-domain (interpolation error) and freq domain (dispersion error)
\end{itemize}